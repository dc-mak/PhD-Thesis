\chapter{\kl{CN} Comparsion and Feedback}

\margintoc{}

In this chapter, I will discuss the reality and expectations of how usable a
verification tool for C \emph{can} be at this stage. I will start with my brief
and informal comparison between \kl{CN} and similar tools, based on my
experience of verifying \kl{pKVM}'s simpler \emph{early} allocator, in \kl{CN},
\kl{VeriFast}~\sidecite{jacobs2011verifast} and
\kl{Frama-C}.\sidecite{baudin2021dogged}\sidenote{\url{https://github.com/rems-project/CN-pKVM-early-allocator-case-study}}
I will also discuss a \kl{RefinedC} verification of the same, which was
completed by the developers~\sidecite{sammler2021refinedc}. Part of this
comparison was featured in~\sidetextcite{pulte2023cn}, including the annotation
overhead and execution times.

After that, I will discuss some feedback given by some industry partners on
\kl{CN}. As an academic, these seem borderline unreasonable, especially when
compared to the aforementioned state-of-the-art research tools, but as an
ex-industry professional, it is much easier to concede. Given our goal of
targeting kernel hackers, we as academics want industry adoption for these
tools, in which case taking industry feedback seriously is crucial.

\section{Comparison}

The early allocator in \kl{pKVM} during intialisation, before switching over to
the aforementioned buddy allocator. It is a simple allocator which just bumps a
pointer, with no support for reclaiming memory. It features functions to
initialise, get the number of pages allocated, and allocate new zeroed pages.

\subsection{Early allocator in CN}

The proof for the early allocator was not part of CI for \kl{CN}, and so is
considerably out-of-date.\sidenote{TODO Update this, add to CI, check how much
slower it is with bit vectors.} Nevertheless, I will show and discuss a sample
of the code and specifications, because because I expect updated specifications
to look recognisably similar.

The first function is one to zero all the bytes in a page. This is implemented
in assembly in \kl{pKVM}, but for the purposes of comparison, I implemented a
version in C. The precondition specifies that the function expects an array of
\cninline{Bytes}, (not to be confused with memory bytes from
\cref{sec:mem-bytes-use}), which is defined as a wrapper around
\cninline{Owned<char>}. The postcondition specifies the function returns an
array of bytes with value zero, expressed this way to avoid the use of the byte
array equality lemma mentioned in \cref{fig;byte-array-eq-lemma}.

The loop invariant states that ownership moves from the former to the latter at
the loop index. This version of \kl{CN} required manually folding and unfolding
predicates, but inferred indices, and so the body of the loop features
annotations to do the former.

\cfile[fontsize=\footnotesize,breaklines,lastline=18]{code/cn_early_alloc.c}

The second version is to allocate a zeroed page. It accesses two global
variables, \cinline{cur} which marks the current pointer of the allocator, and
\cinline{end} which marks the exclusive end of the allocator. It requires there
is at least a page size difference between the two, and of ownership of the
bytes from \cinline{cur} to \cinline{end}, defined in the \cninline{EarlyAlloc}
predicate. It ensures that it retains ownership of the same predicate, but with
new bounds \textemdash{} \cinline{cur} in the postcondition used to mean the
its value at the \emph{end} of the function, using \cninline|{cur}@start| to
refer to the value at the start of the function. And it ensures that the
returned value is the base address of an array of bytes. The body of the
function simply unfolds and folds the \cninline{EarlyAlloc} predicate,
incrementing the pointer by the page size in between.

\cfile[fontsize=\footnotesize,breaklines,firstline=20]{code/cn_early_alloc.c}

\subsection{Early allocator in VeriFast}

The verification of the early allocator in \kl{VeriFast} is similar to the
\kl{CN} one. The main difference is that since \kl{VeriFast} does not support
iterated separating conjunction, it may require lemmas to manipulate the
inductive predicates involved instead. \kl{VeriFast} supports non-precise
assertions (\cref{subsec:precise-assertion}) as well, but does not support
ordinary disjunction for impure assertions for the same reason \kl{CN} does
not: to avoid backtracking in symbolic
execution.\sidenote{\url{https://verifast.github.io/verifast-docs/faq.html\#how-to-express-a-disjunction-p-or-q-in-an-assertion}}

In this example, the \cninline{characters_zeroed} predicate is recursively
defined; and is marked as precise. This allows \kl{VeriFast} to unfold
(\cninline{open}) and fold (\cninline{close}) the predicate automatically. The
loop invariant is expressed in Tuerk-style,\sidecite{tuerk2010local} but
regular loop-invariants are supported too.

\cfile[fontsize=\footnotesize,breaklines,lastline=13]{code/verifast_early_alloc.c}

If \kl{VeriFast} does not support iterated separating conjunctions, then how
does it support the indexing in this example without lemmas? The answer lies in
a surprising difference between how \kl{VeriFast} handles the semantically
equivalent subscripting \cinline{e1[e2]} and \cinline{*(e1 + e2)}.\sidenote{\url{https://github.com/verifast/verifast/issues/259}} % chktex 36

\begin{quote}
    \emph{Indeed, \kl{VeriFast} symbolically evaluates \cinline{*(start + i)} and % chktex 36
    \cinline{start[i]} differently. \cinline{*(start + i)} is symbolically % chktex 36
    evaluated just like any other dereference of a pointer to int, whereas
    evaluation of \cinline{start[i]} first looks for an
    \cninline{ints(start, ?n, ?vs)} chunk where \cninline{i <= n} and, if it % chktex 26 chktex 36
    finds one, returns \cninline{nth(i, vs)}. This is convenient for ``random % chktex 36
    access'' to ints-encoded arrays. If it does not find such a chunk, it falls
    back to looking for an \cninline{integer(start + i, _)} chunk, but the % chktex 36
    \cninline{start + i} computation is indeed not checked for overflow. This
    seems sound because if the integer chunk exists, it implies that the
    address is within the limits.}
\end{quote}

%This could end up being a useful heuristic.

This particular example was also a good lesson in how phrasing assertions
differently can lead to drastically different performance outcomes. Whilst the
final version uses the \cninline{character} predicate (which is a points-to at
character type), initially I used a more general \cninline{chars} predicate
that generalised \cninline{character} relating a current and end pointer to a
\emph{list} of characters. \cninline{create(count, item)} is a fix-point % chktex 36
function that I defined, which creates a list of only \cninline{item} of length
\cninline{count}.

\cfile[fontsize=\footnotesize,breaklines,lastline=2]{code/verifast_alt_loop.c}

\kl{VeriFast} generally has good support for automation, and the ability to
state and prove lemmas (pure and resourceful) from within the system itself.
For pure lemmas, \cninline{lemma_auto}, marks it as available for use by
automation all the time, whereas \cinline{lemma_auto (expr)} automatically
applies the lemma when a precise predicate \cninline{expr} is in the context.

We need a lemma because the loop exit condition \cninline{i == 4096} means that
\cninline{length(unzeroed) == 0} and so \cninline{unzeroed == nil == create(0,0)}, % chktex 36
and this is too much for automation to deduce. Because of this, it was not clear
to me where to place an annotation to manually instantiate the lemma. At the
same time, I was not adept enough with triggers to figure out how to fire them
exactly when needed. Marking it as an automatic lemma without a trigger worked,
but it slowed down verification considerably: usable in batch mode, a not usable
interactively. There is no fallback to external assistants.

\cfile[fontsize=\footnotesize,breaklines,firstline=3]{code/verifast_alt_loop.c}

Like \kl{CN}, \kl{VeriFast} also requires annotations to read (and write)
global variables. The rest of the specification is very similar to the \kl{CN}
one; the \cninline{earlyAlloc} predicate could be marked as precise, this would
have removed the need for the unfolding and folding statements.

\cfile[fontsize=\footnotesize,breaklines,firstline=15]{code/verifast_early_alloc.c}

\kl{VeriFast} has a similar (but slightly smaller) annotation overhead to
\kl{CN}\@. Though I did not use them, fractional permissions are also
supported. As its names implies, it is indeed very fast, about 10 times faster
than \kl{CN} in this case (50 milliseconds vs 500 milliseconds). It has a
useful graphical user-interface which provides syntax highlighting for
specifications, excellent visibility into the proof state, and highlights the
annotation it cannot prove directly in the source code. Not only that, it
supports replaying the steps of the proof leading up to that state.

There are some aspect of \kl{VeriFast} which I do not understand, and cannot
find any documentation or explanation for, namely its handling of structs.
First off, it provides no in built predicates to do the equivalent of claiming
ownership of a whole struct, so users have to write such predicates themselves,
using auto-generated predicates for each members. With this one can write the
following. The semicolon is to mark the predicate as precise, and to separate
input arguments from output ones. Confusingly, it fails, with an error `no
matching heap chunks s2\_x(..)'. % chktex 36

\cfile[fontsize=\footnotesize,breaklines,lastline=15]{code/verifast_structs.c}

However, changing the predicate definition to explicitly mention the fields in
the \cinline{inner} struct allows it to pass.
\cfile[fontsize=\footnotesize,breaklines,firstline=17,lastline=18]{code/verifast_structs.c}

This strikes me as unusual because any changes to the representation of
\cinline{struct s2} are no longer encapsulated fields of that type. In \kl{CN},
the following is true (modulo padding),
\cninline[breaklines]|s1_inner(p, inner) <=> s2_x(&p->inner, inner.x)|,% chktex 36
\sidenote{`s1\_inner' and `s2\_x' are auto-generated predicates, expressing
points-to facts for struct fields.} but in \kl{VeriFast} this does not seem
to be the case.

Furthermore, the specification itself is weak, because it does not connect the
values in the inner struct to the outer struct of which it is a field. Yet,
adjusting the specification to link the output value \cninline|?val| of the
field to the output value of its member, causes the verification to fail again,
this time saying that it cannot prove \cninline|in.x == 1|.

\cfile[fontsize=\footnotesize,breaklines,firstline=19,lastline=20]{code/verifast_structs.c}

To be clear: I know I am at fault, and specifying things incorrectly. I am
quite confident such an example can be verified in \kl{VeriFast}. My point is
to draw attention to the (a) unexpected properties of nested structs and (b)
the difficulty of understanding any modes/dataflow implicit in the syntax. The
issue with structs may be an intentional design choice to sidestep handling
exploding and imploding structs, which does require extra inference and thus
reduce performance.

Stepping back, \kl{VeriFast} uses an ad-hoc semantics of C the developers
believe to be sound. During this brief experiment, I noted a few missing
features.
\begin{itemize}
    \item Struct literals/initialisers.
    \item Implicit type promotions (e.g.\ from \cinline{long long} to
        \cinline{unsigned long long}).
    \item Function types in struct fields.
    \item Taking the address of local variables.
    \item Expressive/accurate pointer provenance (has basic support).
    \item \kl{UB} or unspecified values for uninitialised reads.
\end{itemize}

\subsection{Early allocator in Frama-C and RefinedC}

Whereas \kl{VeriFast} is most similar to \kl{CN}, \kl{Frama-C} and
\kl{RefinedC} are one step removed. Both use undecidable logics, falling back
to \kl{Rocq} for manual proof. For \kl{Frama-C}, the logic is a Hoare logic,
not separation, whereas in \kl{RefinedC}, the logic is an \kl{Iris} instance.

Frama-C has a smaller annotation overhead, though its lack of support for
separation logic means the end guarantee is weaker than with the other tools.
It works by translating C programs into CIL~\sidecite{necula2002cil}. Its Hoare
logic verifier, WP, uses a custom semantics CIL, which is parametric in the
memory model, allowing the user to select trade-off increased performance for
pointer manipulation expressiveness. Its includes support for multiple
specifications per function, ghost parameters, arguments and code, and runtime
assertion checks, which amongst other things, can be used to error on
uninitialised reads. \kl{Frama-C} was noticeably slower than \kl{CN} for the
early allocator (3.5s).

\kl{RefinedC} is implemented inside \kl{Rocq} atop \kl{Iris}, making its
\kl[TCB]{trusted computing base} (TCB) the smallest out of all the tools
mentioned so far. This trust is undercut by its custom ad hoc semantics for C
based on the \emph{Caesium} kernel over which its typing and automation
framework \emph{Lithium} operates. The rules are intended to be heuristic
rather than decidable, with the main mode of operation inside a \kl{Rocq}
session. Its annotation overhead is similar to \kl{CN}, but the performance is
much slower (16.7s).

\section{Industry feedback}

Lorem ipsum dolor sit amet,\sidenote{Under review from industry, hence
placeholder for formatting and word count.} consectetuer adipiscing elit.
Aenean commodo ligula eget dolor. Aenean massa. Cum sociis natoque penatibus et
magnis dis parturient montes, nascetur ridiculus mus. Donec quam felis,
ultricies nec, pellentesque eu, pretium quis, sem. Nulla consequat massa quis
enim. Donec pede justo, fringilla vel, aliquet nec, vulputate eget, arcu. In
enim justo, rhoncus ut, imperdiet a, venenatis vitae, justo. Nullam dictum
felis eu pede mollis pretium. Integer tincidunt. Cras dapibus. Vivamus
elementum semper nisi. Aenean vulputate eleifend tellus.

Aenean leo ligula, porttitor eu, consequat vitae, eleifend ac, enim. Aliquam
lorem ante, dapibus in, viverra quis, feugiat a, tellus. Phasellus viverra
nulla ut metus varius laoreet. Quisque rutrum. Aenean imperdiet. Etiam
ultricies nisi vel augue. Curabitur ullamcorper ultricies nisi. Nam eget dui.
Etiam rhoncus. Maecenas tempus, tellus eget condimentum rhoncus, sem quam
semper libero, sit amet adipiscing sem neque sed ipsum. Nam quam nunc, blandit
vel, luctus pulvinar, hendrerit id, lorem. Maecenas nec odio et ante tincidunt
tempus.

Donec vitae sapien ut libero venenatis faucibus. Nullam quis ante. Etiam sit
amet orci eget eros faucibus tincidunt. Duis leo. Sed fringilla mauris sit amet
nibh. Donec sodales sagittis magna. Sed consequat, leo eget bibendum sodales,
augue velit cursus nunc, quis gravida magna mi a libero. Fusce vulputate
eleifend sapien. Vestibulum purus quam, scelerisque ut, mollis sed, nonummy id,
metus. Nullam accumsan lorem in dui. Cras ultricies mi eu turpis hendrerit
fringilla.

Vestibulum ante ipsum primis in faucibus orci luctus et ultrices posuere
cubilia Curae; In ac dui quis mi consectetuer lacinia. Nam pretium turpis et
arcu. Duis arcu tortor, suscipit eget, imperdiet nec, imperdiet iaculis, ipsum.
Sed aliquam ultrices mauris. Integer ante arcu, accumsan a, consectetuer eget,
posuere ut, mauris. Praesent adipiscing. Phasellus ullamcorper ipsum rutrum
nunc. Nunc nonummy metus. Vestibulum volutpat pretium libero. Cras id dui.
Aenean ut eros et nisl sagittis vestibulum. Nullam nulla eros, ultricies sit
amet, nonummy id, imperdiet feugiat, pede. Sed lectus.

Donec mollis hendrerit risus. Phasellus nec sem in justo pellentesque
facilisis. Etiam imperdiet imperdiet orci. Nunc nec neque. Phasellus leo dolor,
tempus non, auctor et, hendrerit quis, nisi. Curabitur ligula sapien, tincidunt
non, euismod vitae, posuere imperdiet, leo. Maecenas malesuada. Praesent congue
erat at massa. Sed cursus turpis vitae tortor. Donec posuere vulputate arcu.
Phasellus accumsan cursus velit. Vestibulum ante ipsum primis in faucibus orci
luctus et ultrices posuere cubilia Curae; Sed aliquam, nisi quis porttitor
congue, elit erat euismod orci, ac placerat dolor lectus quis orci.

Phasellus consectetuer vestibulum elit. Aenean tellus metus, bibendum sed,
posuere ac, mattis non, nunc. Vestibulum fringilla pede sit amet augue. In
turpis. Pellentesque posuere. Praesent turpis. Aenean posuere, tortor sed
cursus feugiat, nunc augue blandit nunc, eu sollicitudin urna dolor sagittis
lacus. Donec elit libero, sodales nec, volutpat a, suscipit non, turpis. Nullam
sagittis. Suspendisse pulvinar, augue ac venenatis condimentum, sem libero
volutpat nibh, nec pellentesque velit pede quis nunc.

Vestibulum ante ipsum primis in faucibus orci luctus et ultrices posuere
cubilia Curae; Fusce id purus. Ut varius tincidunt libero. Phasellus dolor.
Maecenas vestibulum mollis diam. Pellentesque ut neque. Pellentesque habitant
morbi tristique senectus et netus et malesuada fames ac turpis egestas. In dui
magna, posuere eget, vestibulum et, tempor auctor, justo. In ac felis quis
tortor malesuada pretium. Pellentesque auctor neque nec urna. Proin sapien
ipsum, porta a, auctor quis, euismod ut, mi. Aenean viverra rhoncus pede.

Pellentesque habitant morbi tristique senectus et netus et malesuada fames ac
turpis egestas. Ut non enim eleifend felis pretium feugiat. Vivamus quis mi.
Phasellus a est. Phasellus magna. In hac habitasse platea dictumst. Curabitur
at lacus ac velit ornare lobortis. Curabitur a felis in nunc fringilla
tristique. Morbi mattis ullamcorper velit. Phasellus gravida semper nisi.
Nullam vel sem. Pellentesque libero tortor, tincidunt et, tincidunt eget,
semper nec, quam. Sed hendrerit. Morbi ac felis. Nunc egestas, augue at
pellentesque laoreet, felis eros vehicula leo, at malesuada velit leo quis
pede.

Donec interdum, metus et hendrerit aliquet, dolor diam sagittis ligula, eget
egestas libero turpis vel mi. Nunc nulla. Fusce risus nisl, viverra et, tempor
et, pretium in, sapien. Donec venenatis vulputate lorem. Morbi nec metus.
Phasellus blandit leo ut odio. Maecenas ullamcorper, dui et placerat feugiat,
eros pede varius nisi, condimentum viverra felis nunc et lorem. Sed magna
purus, fermentum eu, tincidunt eu, varius ut, felis. In auctor lobortis lacus.
Quisque libero metus, condimentum nec, tempor a, commodo mollis, magna.
Vestibulum ullamcorper mauris at ligula. Fusce fermentum.

Nullam cursus lacinia erat. Praesent blandit laoreet nibh. Fusce convallis
metus id felis luctus adipiscing. Pellentesque egestas, neque sit amet
convallis pulvinar, justo nulla eleifend augue, ac auctor orci leo non est.
Quisque id mi. Ut tincidunt tincidunt erat. Etiam feugiat lorem non metus.
Vestibulum dapibus nunc ac augue. Curabitur vestibulum aliquam leo. Praesent
egestas neque eu enim. In hac habitasse platea dictumst. Fusce a quam. Etiam ut
purus mattis mauris sodales aliquam. Curabitur nisi. Quisque malesuada placerat
nisl. Nam ipsum risus, rutrum vitae, vestibulum eu, molestie vel, lacus.

Sed augue ipsum, egestas nec, vestibulum et, malesuada adipiscing, dui.
Vestibulum facilisis, purus nec pulvinar iaculis, ligula mi congue nunc, vitae
euismod ligula urna in dolor. Mauris sollicitudin fermentum libero. Praesent
nonummy mi in odio. Nunc interdum lacus sit amet orci. Vestibulum rutrum, mi
nec elementum vehicula, eros quam gravida nisl, id fringilla neque ante vel mi.
Morbi mollis tellus ac sapien. Phasellus volutpat, metus eget egestas mollis,/
lacus lacus blandit dui, id egestas quam mauris ut lacus. Fusce vel dui. Sed in
libero ut nibh placerat accumsan. Proin faucibus arcu quis ante. In
consectetuer turpis ut velit. Nulla sit amet est. Praesent metus tellus,
elementum eu, semper a, adipiscing nec, purus. Cras risus ipsum, faucibus ut,
ullamcorper id, varius ac, leo. Suspendisse feugiat. Suspendisse enim turpis,
dictum sed, iaculis a, condimentum nec, nisi. Praesent nec nisl a purus blandit
viverra. Praesent ac massa at ligula laoreet iaculis. Nulla neque dolor,
sagittis eget, iaculis quis, molestie non, velit. Mauris turpis nunc, blandit

\section{Summary}

Here, I summarise the many (incompatible, or at the very least, impractical)
directions \kl{CN} could pursue based on the comparison with other tools
and the industry feedback. I leave evaluating and synthesising these points
to the next chapter.
\begin{itemize}
    \item Improve performance by 10x.
    \item Trace type checking steps for replay.
    \item Improve syntax concision and approachability.
    \item Improve error messages.
    \item Infer/suggest specifications.
    \item Reduce the \kl{TCB}.
    \item Support using \kl{CN} `out-of-the-box'.
    \item Focus on industry-favoured use cases, such as mathematics,
        input-validation, memory safety beyond current tools, and concurrency.
\end{itemize}

\chapter{Not-so-great expectations}

In the last chapter, I compared \kl{CN} to some relevant tools in the
space,\sidenote{A notable omission was \kl{VST}, mostly due to lack of time and
expertise for a fair comparison.} and I presented industry feedback on \kl{CN}.
Based on my experience of having worked on programming language tooling both in
industry and in academia, I will synthesise the two and lay out what I believe
are achievable standards of usability for \kl{CN}, with an emphasis on how new
projects can avoid several pitfalls.

\section{Work backwards from examples}

\kl{CN}'s guiding star during its development was the \kl{pKVM buddy
allocator}. Whilst this was understandable, and indeed its unique selling
point, this \emph{also} guided its design to a large degree. Features were only
considered and implemented insofar as they got \kl{CN} closer to verifying the
entire allocator.

Whilst this produced a successful paper, this over-fitted \kl{CN} to the example.
More concretely, it meant that we had accrued significant technical and design
debt, which became more obvious by the time it came to specifying and verifying
more pedestrian examples, such as those now in the \kl{CN
tutorial}.~\sidecite{pulte2024tutorial}

Many of the features mentioned in \cref{chap:inform-impl} might have been
incorporated earlier. An obvious candidate is unifying the syntax for
functions, predicates and specifications: these features were built separately
over time and as a result have ended up similar but
incompatible.\sidenote{\href{https://github.com/rems-project/cerberus/issues/288}{Cerberus
\#288}} For example, one of the many incompatibilities is the separate scopes
for predicates and functions: this results in a confusing error message when
the predicate and a function are mixed up. However, unifying the scopes at
point would require a major refactor to symbol resolution during
desugaring.\sidenote{\href{https://github.com/rems-project/cerberus/issues/288}{Cerberus
\#288}} Had we worked through enough examples and designs concretely, we might
have noticed the emerging similarities and reworked the approach.

With only slight irony, my point is that \emph{test-driven development is
exceptionally useful when implementing a verification tool}. This bore out in
my personal experience in implementing \kl{VIP}: writing examples and
informally but rigorously working through how they ought to be specified and
helped enormously in guiding my intuitions and highlighting problems early on,
and I conjecture a broader range of smaller examples escalating to the
\kl{buddy allocator} might have done similar.

\section{Design with a formalism}

Whilst there are some changes that I think would have been difficult to
pre-empt by more careful design at the start, for example, the change in
inference
scheme,\sidenote{\href{https://github.com/rems-project/cerberus/commit/7c2c0a364a4373e4eb109f32d01cc9584f51e81f}{Commit
7c2c0a36.} This was motivated by reducing the annotation burden for folding
and unfolding predicates, which turned out to be heavier than annotation
saved by inferring indices.}, or the switch to the monadic syntax
(\cref{sec:monadic-syntax}), there are few design decisions I conjecture would
would have benefited from scrutiny prior to implementation.

\subsection{Calling conventions affect syntax}

By default, \kl{Cerberus} elaborates calls to C functions by allocating and
initialising parameters \emph{at the call site} and then \emph{passing
pointers} to the temporary objects to the callee. This leads to very verbose
and awkward specifications, where functions had to (a) claim ownership of their
parameters and (b) promise they did not modify them so that the objects at the
call-site remained unaffected.

This eventually became intolerable enough to realise that what needed to change
was not the specifications but the
elaboration.\sidenote{\href{https://github.com/rems-project/cerberus/commit/772173d6432c86b029fd1bb993b8dc83b80c96c0}{Cerberus
772173d6}} Of course, this required changes to \kl{CN}
too,\sidenote{\href{https://github.com/rems-project/cerberus/commit/186e4e42a75cad2222d441ce608fc1dc84cc7b98}{Cerberus
186e4e42}} to handle the changed elaboration. Ultimately, this resulted in
duplicated effort. Had we spent some time looking at small sample core
programs, and testing out a formalism on it, we might have anticipated the
issue sooner.


\subsection{Error reporting}

Most of the time, \emph{any} programming language tool will be fed an incorrect
or incomplete program. The standard approach to dealing with this is to layer
the program's stages such that only well-formed programs pass from one stage to
the next. Within the context of \kl{CN}, the main stages are:
\begin{itemize}
    \item Parsing
    \item Desugaring
    \item Elaboration of C into \kl{Core}
    % \item Translation of \kl{Core} to $\mu$\kl{Core}
    \item Base type/well-formed specification checks
    \item Per-function constraint and resource inference and checking.
\end{itemize}

Such a strategy is understandable, indeed it makes the system more robust
because each phase assumes the invariants ensured by the previous, and
establishes new ones. It makes the implementer's job easier, but as important
as that is, it is quite likely the cumulative user time will dominate
cumulative implementer time in the long run: code is written once, read many
times, and used orders of magnitude more.

The problem is that users (rightfully) expect to be able to get feedback on
their program at multiple stages of development, and getting one error at a
time slows down the edit-check cycle. Of course, multiple errors can also
overwhelm the user, but it is easier to limit them once they are already
supported, than it is to support them if they do not exist.

In the ideal case, users should be able to execute a whole pipeline of stages
even in the presence of errors from the first; it would be very impressive if
\kl{CN} could do resource inference in the presence of parse errors
\emph{within the function it was checking}, or even just in other parts of the
program.

This is likely unrealistic for \kl{CN}, because of the amount of re-engineering
it would require in the \kl{Cerberus} front-end. A realistic goal for \kl{CN}
is to raise the lower bar of recovery, by designing each step such that:
\emph{it avoids stopping at the first error}. For example, up until relatively
recently, \kl{CN} stopped at the first error in the first failing function,
\emph{even though it was designed to support per-function checking}.

It would be ironic, if, in advocating support for multiple errors, I insisted
that the best way to achieve this was a full rewrite of everything starting
from the parser. And so, below I conjecture a process which I believe can be
applied in an incremental way to existing projects, which allows developers
to reap the benefits of multiple errors in proportion to the amount of
time they are willing to spend on it.
\begin{itemize}
    \item Pick the most important stage.
    \item For every (or each important) sort, a \emph{hole}
        construct.~\sidecite{omar2017hazelnut}
    \item Update the rest of the stage to handle this hole.
    \item If the subsequent stage supports holes, consider joining them up.
\end{itemize}
If, like in \kl{CN} proof mode, the last (semantic analysis) stage backwards is
the most interesting and expensive one, this suggests working backwards through
each stage.

To be clear, I am not talking about \emph{incrementalising} the entire type
checker, though support for holes could make this easier.

What I am saying is that for every \kl{CN} annotation, there is the opportunity
to support multiple errors through multiple stages. Take the example of failed
the bit vector update for the buddy allocator (\cref{sec:buddy-failed-bv}). One
of the major pain points of this update was (tediously) ensuring that
\emph{every datatype, function, predicate, and pre/postcondition} was updated
to switch from integer to bit vector types; without this, \kl{CN} would refuse
to move on to checking any functions. As a result, I was forced to manually
figure out the dependency structure, start at the leaves, and comment out
everything else that was irrelevant. This also made it difficult to
incrementally check my progress into a CI pipeline. Though gating the switch to
integers behind a flag would have helped, I believe multiple error propagation
would have been helpful too, \emph{especially when type checking is slow}.

I conjecture inspiration from the below two papers will prove fruitful
to investigating and guiding the theory and implementation of multiple
errors for \kl{CN}.
\begin{itemize}
    \item \citetitle{zhao2024total}~\sidecite{zhao2024total}.
        This paper defines a bidirectional gradual
        type system for types and expressions (including holes), and a total
        procedure to mark expressions with \emph{all possible} type errors. It
        then adds constraint solving on top of the type holes, so that
        inconsistent type constraints are localised \emph{exclusively to holes
        and marked types and expressions}, by tracking the program origins of
        unknown types. The relevance of this paper is its principled approach
        to all of the above.
    \item \citetitle{spies2024quiver}~\sidecite{spies2024quiver}.
        This paper also uses a two way flow of
        information familiar from bidirectional type checking, but describes it
        using terms more standard in separation logic circles: \emph{abductive
        deductive verification}. It symbolically evaluates a program using a
        context of separation logic assertions, proving everything it can and
        assuming everything it cannot. The specifications (`predicate
        transformers') they infer are analogous to \kl{CN}'s function types.
        The relevance of this paper is its similar domain to \kl{CN}.
\end{itemize}
Though designing such calculi in this way may seem like it increases the amount
of work involved for reporting errors, in actuality, it simply makes
implementation-related error reporting \emph{explicit in the formalism}.

\subsection{Elaborate with care}

Elaboration and large-scale program transformations such as A-normalisation can
affect the accuracy of source locations. On top of this, if handled without
care, generated variables can also end up not associated to any source location
the user wrote, and can be difficult to interpret in a counter-example. \kl{CN}
used to A-normalise, but does not any
more,\sidenote{\href{https://github.com/rems-project/cerberus/commit/21808139bda2ee320756c71eb22dbd57d0986f97}{Commit
21808139}.} because it was difficult to relate such variables to \kl{Core}
expressions. The situation has improved a bit with more helpful auto-generated
names and no A-normalising, but the solution is ad hoc, and the gap between
\kl{Core} and C still remains.

Whilst I have some ideas on how to improve the situation,\sidenote{Low-hanging
fruit is to associate bound variables to the location of the sub-expression
they are binding
(\href{https://github.com/rems-project/cerberus/issues/270}{Cerberus
\#270}).} a systematic treatment of source location information through
elaboration would be helpful to understand what it would mean, and what is
required for maximal source location accuracy.

\section{Software engineering}

\kl{CN} has been developed for around 4 years at the time of writing. It has
had multiple contributors, some of whom have left the project now. As such,
it benefits from good software engineering practices like any other project.
There are many such things, so I will focus on the important ones.

\subsection{Rich regression testing}

For a long time, \kl{CN} did not have any regression testing and so features
which were implemented, but accidentally broke later, went unnoticed.
Another substantial benefit to regression testing is that it enables assists
with fairly large-scale refactors. Regression testing is also very helpful when
multiple people are working on overlapping parts of the project, so that
changes which accidentally break other things are picked up \emph{before} being
merged in to the main branch.

Regression testing should not only capture the ability of the program to
execute with or without errors, but also \emph{the output} in \emph{both} of
those cases. If a test is intended to fail before a \kl{VIP}-related change,
and it continues to fail after the change, it is often good to check that it is
failing \emph{for the same reason} (we would not a division-by-zero test
changing its behaviour after a change to the memory-object model).

To assist with this, I wrote a small but increasingly sophisticated Python
script\sidenote{\href{https://github.com/rems-project/cerberus/pull/703}{Cerberus
\#703}.} to run a given program, in a particular configuration (consisting a
name, a filter on file names, and arguments to the program) and capture and
check the return codes and all the output, against a pre-existing file. If
there is a difference, the program will raise an error, and also provide a diff
the developer can apply \emph{across some or all of the affected files} if the
change is expected.

\subsection{Performance benchmarking}

A slow heavily-automated verification tool is basically an unusable
verification tool, and the lack of good performance measurement makes it
difficult to understand why \kl{CN} is slow and how to fix it.

Prior measurements were ad hoc but revealed that most of the time is spent in
the solver, which we believe to still be the case. Whilst this is useful
information, we need better granularity on the subject, to answer the following.
\begin{itemize}
    \item What is the distribution of times per call to the solver?
    \item What part of \kl{CN} generates the slowest problems?
\end{itemize}

We have some performance benchmarking now, enough to demonstrate that enabling
\kl{VIP} is slow and soundly reducing bounds checks\sidenote{By replacing them
with liveness checks.} reduces execution time (\cref{sec:vip-ref}). However, we
do not have good insight into explaining these observations. The measurements
are very noisy and coarse \textemdash{} the script measures the total execution
time for running each regression test once, when what we need are answers to
the aforementioned questions.

As I mentioned before, performance remains the main reason that it is nigh-on
impossible to update the \kl{pKVM buddy allocator} to verify after \kl{CN}'s
bit vector update (\cref{sec:buddy-failed-bv}). \kl{CN} is still an order of
magnitude slower than we would like or expect it to be, and so fixing this, and
setting up infrastructure prevent backsliding, is a key priority.

\subsection{Get source location right}

Source location information is the fundamental building block of a useful error
message, without which an error is close to useless. Unfortunately, early
versions of \kl{CN} did not give this crucial data its due leading to an
incredibly poor user experience. This is all the more disappointing because
within the OCaml ecosystem, the \kl{Menhir} parser generator makes getting this
sort of thing right a lot easier than one might initially
expect.~\sidecite{pottier2016reachability}

\paragraph{Keep lexer and parser simple.} Initially, support for \kl{CN}
annotations in comments was implemented with reference to lots of global
mutable state.\sidenote{A feature of ocamllex that seems to be
under-appreciated is that each rule is an OCaml function and can be passed
arbitrary parameters, thus avoid the need to use global mutable state.}
This prevented code re-use, complicated the code and introduced subtle
location bugs because of accidentally re-using shared mutable token
buffers. I simplified the lexer and parser to avoid this, and in the
process improved and unified the error-reporting for parsing C and
\kl{CN}.\sidenote{\href{https://github.com/rems-project/cerberus/pull/252}{Cerberus
\#252}.}

\paragraph{Investigate strange source locations.} Strange source locations are
bugs which deserve to be investigated, or at the very least, have their
triggers documented. One such
issue\sidenote{\href{https://github.com/rems-project/cerberus/issues/382}{Cerberus
\#382}.} uncovered a subtle interaction between the `lexer hack' used to parse
C11,~\sidecite{jourdan2017simple} and a token buffer feature offered by
\kl{Menhir} for more helpful error messages. Again, the issue was mutable state
which was accidentally shared across what should be separate, fresh instances
of the lexer. The solution was to stage the creation of the lexer and its
internal state, and make this explicit in its
API.\sidenote{\href{https://github.com/rems-project/cerberus/commit/c970a43b7126560b229fb55c32ce22bfbc5d23f2}{Cerberus
c970a43b}.}

\paragraph{Write parser error messages if feasible.} Initially, a parse error
in \kl{CN} would simply signal `unexpected token' and signal no other
information, except a ballpark source location which was sometimes right. This
was less than helpful for a number of reasons, most notably not telling the
user what token \emph{was} expected. It is possible to \emph{generate} very
inelegant, but informative parse error messages based on the error message
format output by
\kl{Menhir}.\sidenote{\url{https://gallium.inria.fr/~fpottier/menhir/manual.html\#sec\%3Amessages\%3Aformat}}

I shall now explain how I did
this.\sidenote{\href{https://github.com/rems-project/cerberus/commit/127758764cd6efdabaaaddb60eabf40575864117}{Cerberus
12775876}.} The first line shows the sentence used to get to the error state.
This is useful for \kl{Menhir}, but is not considered to be a good source of
information for writing an error message. This is because the information in
the comments shows the production which is being parsed, and at which point in
it the error occurs. More than half of error states (629) have exactly one
production, and in this case, I generate an error message as shown at the
bottom of the excerpt.

\begin{minted}[fontsize=\footnotesize,linenos,breaklines]{py}
cn_statements: ASSERT WHILE
##
## Ends in an error in state: 1585.
##
## cn_statement -> ASSERT . LPAREN assert_expr RPAREN SEMICOLON [ .. ]
##
## The known suffix of the stack is as follows:
## ASSERT
##

parsing "cn_statement": seen "ASSERT", expecting "LPAREN assert_expr RPAREN SEMICOLON"
\end{minted}

Of course, it is better to fill these and the remainder (593) manually, as I
demonstrated for one
case.\sidenote{\href{https://github.com/rems-project/cerberus/issues/245}{Cerberus
\#245}.} Details of how best to do so are in the \kl{Menhir} manual, along with
examples and links to how \kl{CompCert} uses it for its C parser (proof that
the approach can be scaled). For the easy cases I did not auto-generate, I
output an message which says that an error message is missing for the state
(identified by a number), and that it can be added in the stated file. In any
case, I set up \kl{Menhir} to show exactly where and between which tokens
parsing failed, which is often enough for experienced users to fix the syntax.
\kl{Menhir} also provides functionality to check and merge its auto-generated
error messages with existing ones, so that the latter can be preserved despite
changes in the grammar.

\paragraph{Consider a custom pre-processor.} The bane of any academic C tooling
seems to be the preprocessor. Whilst it faithfully preserves line numbers, and
this tends to be good enough, it does not preserve column information, which in
most cases, leads to inaccurate error messages on any code which is inside, or
after a macro expansion.\sidenote{The macro call and expansion may have the
same length if one is lucky.}

We are caught between two unappealing choices: either present the error message
in terms of the preprocessed code, thus retaining accuracy but sacrificing
relevancy, or present the error message (with the wrong location) in terms of
the original code, retaining relevancy but not accuracy. \kl{CN} currently does
the latter, and given the fact that macros do often represent constants, or
function-like abstractions~(\cref{subesc:takeaway-features}), we would also
like to have them supported in \kl{CN} as well.

There are no easy solutions to this,\sidenote{Even CompCert just calls a
user-specified preprocessor.} but it appears to be a solvable problem. For
example, there maybe ways to coax GCC or Clang to output debug tokens whilst
preprocessing a file, which could be input to a custom lexer which takes the
tokens \emph{and the source location information} output and connects it to the
\kl{Cerberus} front-end. Or, one could write a Clang plugin, or use the Boost
pre-processing library Wave to attempt a similar trick, with more control over
the resulting format. In both cases, it looks like it should be possible to
preserve comments too, but this will not allow running the preprocessor
\emph{inside} \kl{CN} annotations.

It may however, be feasible to implement a custom C preprocessor. What makes
this feasible is the existence of a relatively straightforward pseudo-code
algorithm by Dave Prosser, \emph{which formed the basis of the prose
    specification of macro expansion in the C89/ANSI C
standard}.~\sidecite{prosser1986complete} This algorithm is annotated and
explained~\sidecite{spinellis2008complete} by the author of the CScout
refactoring browser for C,~\sidetextcite{spinellis2010cscout} Diomodis
Spinellis, who was struggling to implement a C preprocessor correctly for about
five years, before finding a reference to an algorithm by Dave Prosser.
Spinellis could not find the algorithm, so he emailed Prosser who
obliged.\sidenote{\url{https://www.spinellis.gr/blog/20060626/}}

Not only is there an algorithm, Spinellis' implementation is available
online\sidenote{\url{https://github.com/dspinellis/cscout/blob/master/src/macro.cpp}}
along with a suite of test
cases\sidenote{\url{https://github.com/dspinellis/cscout/tree/master/src/test/cpp}}
and expected
outputs.\sidenote{\url{https://github.com/dspinellis/cscout/tree/master/src/test/out}}
There is even expertise within the programming language research community on
the algorithm, since it was recently discussed as being closely related to the
strategy for checking termination for expanding type definitions in type
expressions in OCaml.~\sidecite{chataing2024unboxed} This is not to say that it
will be easy, but at this stage such an effort seems more likely to succeed
than the tree-carver was when I embarked on implementing it.

\subsection{Error, do not crash}

\subsection{Log, do not debug}

\subsection{Gate large changes}

\subsection{Proof maintenance}

\subsection{Miscellaneous}

Good commit messages.
Automatic code foramtting.

\section{Invest in error reports}

% Sean Chen - The Anatomy of Error Messages in Rust — RustFest Global 2020 https://youtu.be/oMskswu1SxM?feature=shared
% https://rustc-dev-guide.rust-lang.org/diagnostics.html

\subsection{Counter examples}\label{sec:counter-ex}

\begin{itemize}
    \item Not minimal
    \item Not consistent
    \item Not easy to relate to source
\end{itemize}

\subsection{The unreasonable effectiveness of good error messages}\label{sec:error-msgs}

\begin{itemize}
    \item Translate standards jargon into C programmer friendly words.
\end{itemize}

