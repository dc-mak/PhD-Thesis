Overview of engineering part.

\chapter{Tree-carving: Taming C Projects}\label{chap:tree-carver}

\begin{definition}[\intro{Observational refinement}]\label{def:obsref}
  A term $\Gamma \vdash t \ty A$ \kl{observationally refines} a term
  $\Gamma \vdash u \ty A$, noted $\intro* t \obsRef u$, if for all boolean-valued observation context
  $\mathcal{C} \ty (\Gamma \vdash A) \Rightarrow (\vdash \Bool)$ closing over all
  free variables, if $\mathcal{C}[u] \red \err[\Bool]$ or diverges,
  then either $\mathcal{C}[t] \red \err[\Bool]$ or $\mathcal{C}[t]$ diverges.
\end{definition}

\begin{comment}

We have already seen in \arefpart{metacoq} how the structure of bidirectional typing can
help with proofs on \kl{CIC}/\kl{PCUIC}. But this is far from being the only advantage of
the approach. Indeed, the extra control provided on the conversion rule can be instrumental.
In this part, we go over one situation where this is the case: the extension of \kl{CIC} to
incorporate \kl(typ){gradual} features.

\kl{Gradual typing} arose as an approach to selectively and soundly relax static type
checking by endowing programmers with imprecise static
types \sidecite{Siek2006,Siek2015}. Optimistically well-typed
programs are safeguarded by runtime checks that detect violations of
statically-expressed assumptions. A gradual version of typed lambda
calculus is flexible enough to embed the untyped lambda
calculus \cite{Siek2015}.
This means that gradually-typed languages tend to accommodate at least
two kinds of effects: non-termination and runtime errors.
% The smoothness of the static-to-dynamic checking spectrum afforded by gradual languages
% is usually captured by (static and dynamic) gradual
% guarantees
% which stipulate that typing and reduction are monotone with respect
% to precision~\cite{siekAl:snapl2015}.

Originally formulated in terms of simple types, the extension of \kl{gradual typing}
to a wide variety of typing disciplines has been an extremely active topic of
research, both in theory and in practice. As part of this quest towards more
sophisticated type disciplines, gradual typing was bound to meet with full-blown
dependent types. This encounter saw various premises in a variety of approaches
to integrate (some form of) dynamic checking with (some form of) dependent
types \sidecite{Ou2004,Wadler2009,Knowles2010,Tanter2015,Lehmann2017,Dagand2018}.
Naturally, the highly-expressive setting of dependent types, in which terms and
types are not distinct and computation happens as part of typing, raises a lot
of subtle challenges for gradualization.

Of those challenges, one of the first is the place of computation.
In the gradual setting, in order to optimistically compare types,
\kl{conversion} is replaced by \kl(grad){consistency}, a relation
akin to unification. This relation is naturally
non-transitive, meaning that the usual, \kl(typ){undirected} setting is not suited for
gradualization.%
\sidenote{%
  This is because \ruleref{rule:cic-conv} can be applied any number of times, which is sensible
  only if these successive application amount to just one.
}
Moreover, the semantics of \kl(typ){gradual} languages is usually explained
through an elaboration phase to a second language, responsible for the runtime checks ensuring
safety of evaluation. This elaboration is naturally described in a bidirectional
system, which furthermore provides enough constraints on the typing derivation so that replacing
\kl{conversion} with \kl(grad){consistency} is reasonable. Finally, the identification of
the role of \kl{reduction} for \kl{constrained inference} clarifies
how the latter should be extended to incorporate imprecise types.
In fact, I told the story upside down: it is the pressing need for bidirectional typing
in the context of \kl{gradual typing} that led me to its investigation!

In this part, we go over a collaboration with Kenji Maillard, Éric Tanter and Nicolas
Tabareau to address the challenge of gradualizing a full-blown dependently-typed language:
\kl{CIC} \sidecite{LennonBertrand2022}.
\cref{chap:gradual-dependent} gives an overview of the challenges and trade-offs
involved in \kl(typ){gradual} \kl(typ){dependent} types, culminating with the
\kl{Fire Triangle of Graduality}, which identifies an irreconcilable tension between
the properties one should demand of such a type system. It ends with a broad picture
of our proposed solution to those difficulties,
the \reintro{Gradual Calculus of Inductive Constructions} (\kl{GCIC}).
\Cref{chap:bidir-gradual-elab} describes precisely this \kl{GCIC},
via a relation representing type-based, bidirectional elaboration,
which represents my main technical contribution to \textcite{LennonBertrand2022}.
Finally, \cref{chap:beyond-gcic} gives an overview of the rest of our work in the area:
models used to establish properties of the target language of the elaboration procedure,
and the thorny question of indexed inductive types and consistent reasoning about gradual
programs.
Due to their absence of direct relation to bidirectional typing and my lower
involvement in their technical development, this chapter does not go into full details,
but they are of course present in the publications –
either \sidetextcite{LennonBertrand2022}, or \sidetextcite{Maillard2022}.

\end{comment}
