\chapter{Names for Type Systems}\label{chap:names}

\subsection*{MLTT and CIC}

% Dependent type theory is a vast field, and as in any other field there are numerous
% variations in the objects considered, both due to advances in their understanding,
% and to diverging purposes and techniques. In the end, when
% choosing a particular type system to investigate, there are many more parameters to
% fix than names available for them, so that the same name is bound to be used for
% different systems.

In the field of dependently types, I think we can safely delineate two main schools,
with different histories and cultures. The first goes back to Martin-Löf –
in particular \sidetextcite{MartinLoef1972} –, and is strongly linked to the \kl{Agda}
proof assistant. The second is related to the proof assistant \kl{Coq}, in the filiation of
Coquand and Huet – since \sidetextcite{Coquand1988}.
The umbrella name “MLTT”, for Martin-Löf Type Theory is the one usually used for systems
in the first school, while ones in the second tend to use “CIC” – Calculus of Inductive
Constructions –, or variants thereof.

This separation is of course not a strict one,
and researchers from both schools interact, exchange
theoretical and implementation ideas, and move forward together. But still, this cultural
difference is not anecdotal, as seemingly small differences between the approaches on both
sides lead to wildly different behaviours between the systems, so that some techniques
that are very successful on one side can prove unusable on the other.

I tried to probe the community of proof assistants%
\sidenote{Using a \href{https://proofassistants.stackexchange.com/questions/267/what-are-the-differences-between-mltt-and-cic}{question} on the proof assistant
Stack Exchange.}
as to what they consider the more important differences between the two schools,
which led to quite different answers,
although this is very approximate: \kl{Agda} has a general scheme for inductive types
(including cubical ones in the cubical library) while many articles on \kl{CIC} only
consider a few example inductive types – as was the case in parts of this thesis –, etc.
So this should be read as “this tradition is more prone to taking that approach”.
The results are summarized in \cref{fig:mltt-cic}.

\begin{figure*}[h]
  \begin{tabular}{cccccc}
    \rule{0pt}{4ex} & Universes & Inductive Types & Pattern-matching & Philosophy & Conversion \\
    \midrule
    MLTT \rule{0pt}{4ex} & Predicative hierarchy & $0$, $1$, $W$ and $Id$ & Top-level clauses & Constructivism & Typed \\
    CIC \rule{0pt}{4ex} & Impredicative $\Prop$ & General scheme & First-class terms & None/Formalism & Untyped
  \end{tabular}
  \caption{General characteristics of MLTT and CIC}\label{fig:mltt-cic}
\end{figure*}

\subsection*{Why “CIC”?}

The one feature which came out maybe as the more prominent in the distinction between MLTT and
CIC is the presence of an impredicative sort of propositions, which immensely augments the
logical power of the theory, and makes it much harder to prove normalization.
Despite the exclusion of propositions by default, I still chose to use the name \kl{CIC}
in this thesis, for multiple reasons.

First, regarding all other columns in the table, the system fits more
in the bottom line than the top one. In particular, the general spirit of studying
conversion using tools from rewriting theory which appears as a repeated pattern throughout
the thesis is incompatible – or, at the very least, must be heavily amended –
with a typed conversion.
Moreover, apart from \arefpart{gradual}, the absence of
treatment of $\Prop$ on the paper presentation was done mostly due to simplification concerns
than to theoretical limitations, as the formalization of \kl{PCUIC} as a whole illustrates.
This also applies to \cref{chap:bidir-gradual-elab}, even though the models presented
in \cref{chap:beyond-gcic} do not scale to $\Prop$, meaning that the target
of \cref{chap:bidir-gradual-elab} would then be on a precarious foundation.

But more importantly, in the bidirectional approach,
there is again a clear cultural difference between
\kl{Agda}/MLTT and \kl{Coq}/CIC\@. The former have used the bidirectional ideas for a long
time in order to allow for a lightweight syntax using Curry-style abstractions,
at the cost of losing completeness of typing on non-normal forms.%
\sidenote{This is a deliberate trade-off, at least in the case of \kl{Agda}~\cite[p.~19]{Norell2007}.}%
\margincite{Norell2007}
The latter insist on keeping enough annotations in the kernel syntax by using Church-style
abstractions to let every term infer,
and use a mechanism of implicit arguments during elaboration to lighten the weight of for users.
This means that the completeness theorem as stated in \cref{thm:compl-ccomega}
does \emph{not} hold in any of the standard presentations of MLTT,
while it does to CIC’s, as this thesis shows.

% \chapter{\kl{MetaCoq} Contributors}
% \label{chap:meta-contrib}

% Where I list the contributors and contributions to \kl{MetaCoq}.
