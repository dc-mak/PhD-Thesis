\chapter{Bidirectional \kl(tit){PCUIC}}
\label{chap:bidir-pcuic}

\margintoc

As we have seen in \cref{sec:tech-pcuic}, there is much more to the real \kl{Coq} than
\kl{CCω}.
The ideas exposed in the previous chapter nevertheless
scale very well to these extensions.
There are two areas, though, where some care needs to be taken.
The first is cumulativity, which in particular forces us to reconsider the statement
of the completeness and uniqueness properties, see \cref{sec:bidir-pcuic-cumulativity}.
But the main one is the introduction of inductive types. In particular, there is a subtle
interplay with cumulativity in the treatment of pattern-matching.
Working on the formalized proof of completeness in \kl{MetaCoq} led to the discovery of
an incompleteness bug in the kernel of \kl{Coq} linked to this.
In \cref{sec:bidir-pcuic-inductives} we show how the bidirectional setting adapts to
inductive types, and try and give an intuition of the origin of the completeness issue.

We do not give precise proofs in this chapter,
instead relying on the formalization in \kl{MetaCoq} described in \arefpart{metacoq}.

  % The first area of difference are the universes. While on paper those are simply integer, to handle typical ambiguity and polymorphic (co)-inductive types, PCUIC uses algebraic universes, containing level variables, algebraic $\vee$ and $+1$ operators, and a special level for the sort Prop. Moreover, those universes are cumulative, that is they behave as if smaller universes were included in larger ones. The precise handling of the algebraic universes is abstracted away in MetaCoq, and quite similar in the directed and undirected systems, so it did not prove too difficult to handle. Cumulativity, however, introduces some not-so-small differences with the previous presentation, so we spend some time on it in \cref{sec:pcuic-cumul}.

  % The second is the addition of new base type and term constructors. We describe the treatment of inductive types in \cref{sec:pcuic-indu}. Co-inductive types and records behave very similarly to inductive types at the level of typing, so we do not dwell on them. The difference lies mainly at the level of reduction/conversion, but as our type system treats those as black boxes the differences have a negligible impact.
  
  
\section{Cumulativity}
\label{sec:bidir-pcuic-cumulativity}

\begin{marginfigure}
  \begin{mathpar}
    \inferdef{CheckCum}
    {\inferty{\Gamma}{t}{T} \\ T \cum T'}{\checkty{\Gamma}{t}{T'}}
    \label{rule:bd-check-cum}
  \end{mathpar}    
\end{marginfigure}
The introduction of the more liberal cumulativity rules in the undirected system
of course calls for an update to the computation rules.
The change to \ruleref{rule:bd-check} is direct: simply replace conversion with cumulativity,
as done in \ruleref{rule:bd-check-cum} opposite.
As for the constrained inference rules, they do not even need any modification.
Intuitively, this is because there is no reason to degrade a type to a larger one,
unless it is forced by a given target type in the checking judgment.

The statement of completeness also needs to account for cumulativity,
and becomes the following one.

\begin{theorem}[Completeness, with cumulativity]
  \label{thm:comp-cumul}
  If $\Gamma \vdash t \ty T$, then $\inferty{\Gamma}{t}{T'}$ is derivable
  for some $T'$ such that $T' \cum T$.
\end{theorem}

This also means that in the setting of \kl{PCUIC},
\kl{uniqueness of types} up to \kl{conversion} is not true any more.
For instance, we both have $\Gamma \vdash \uni[0] \ty \uni[1]$ and $\Gamma \vdash \uni[0] \ty \uni[2]$, but $\uni[1]$ and $\uni[2]$ are not convertible. In that context, however,
the type $\uni[1]$ still has a special property: it is minimal among all types, what
we call a \kl{principal type}.

\begin{definition}[Principal type]
  The type $T$ is a \intro{principal type} for term $t$ – in a context $\Gamma$ –
  if $\Gamma \vdash t \ty T$ and for any $T'$ such that $\Gamma \vdash t \ty T'$,
  we have $T \cum T'$.
\end{definition}

The existence of such a principal type is the same as \kl{uniqueness of types} 
up to cumulativity. Moreover, even in the cumulative setting, \cref{thm:unique-inf}%
\sidenote{Uniqueness of inferred types up to joinability.}
stays true. Intuitively, this is because it only relies on properties of reduction, but not of
conversion. Thus, following the same proof as that of \cref{thm:unique-undir},%
\sidenote{Uniqueness of types for undirected typing.}
we obtain that inferred types are principal.

\begin{theorem}[Inferred types are principal]
  \label{thm:princ-types}
  If $\Gamma$ is well-formed and $\inferty{\Gamma}{t}{T}$,
  then $T$ is a \kl{principal type} for $t$ in $\Gamma$.
\end{theorem}
  
\begin{proof}
  If $\Gamma \vdash t \ty T'$, then by completeness there exists some $T''$ such that
  $\inferty{\Gamma}{t}{T''}$, and moreover $T'' \cum T'$.
  But by \cref{thm:unique-inf}, $T \conv T'' \cum T'$ and thus $T \cum T'$, and $T$ is thus indeed a principal type for $t$ in $\Gamma$.
\end{proof}

The existence of \kl{principal types} is not so easy to prove directly, as it more or less
amounts to showing soundness and completeness of the bidirectional system at once.
Nevertheless, it is useful, because it in particular means that any well-typed term $t$
has an unambiguous smallest universe, which can be obtained as the \kl{principal
type} of its \kl{principal type}. This means that there is a good separation between irrelevant 
propositions – those terms whose smallest universe is $\Prop$ – and relevant terms
– those whose smallest universe is some $\uni[i]$ –, and that this stays true even in
presence of cumulativity, and even if $\Prop \cum \uni[i]$. If this were not the case,
the erasure of propositional content – which is one of the important use cases of $\Prop$ –
would not make sense.

\section{Inductive Types}
\label{sec:bidir-pcuic-inductives}

\subsection{An example: the pair type}[The pair type]

\begin{figure}
    
  \begin{mathpar}
    \inferdef{PairTy}{
      \pinferty{\uni}{\Gamma}{A}{\uni[i]} \\ 
      \pinferty{\uni}{\Gamma, x : A}{B}{\uni[j]}}
      {\inferty{\Gamma}{\Sb x : A .\ B}{\uni[\umax{i}{j}]}}
      \label{rule:pair-type-bd} \and
    \inferdef{Pair}{
      \pinferty{\uni}{\Gamma}{A}{\uni[i]} \\
      \pinferty{\uni}{\Gamma, x : A}{B}{\uni[j]} \\
      \checkty{\Gamma}{t}{A} \\
      \checkty{\Gamma}{u}{\subs{B}{x}{t}}}
      {\inferty{\Gamma}{\pair[A][x.B]{t}{u}}{\Sb x : A .\ B}}
      \label{rule:pair-bd} \and
    \inferdef{PairInd}{
      \pinferty{\Sb}{\Gamma}{s}{\Sb x : A .\ B} \\
      \pinferty{\uni}{\Gamma, z : \Sb x : A .\ B}{P}{\uni} \\
      \checkty{\Gamma, y_1 : A, y_2 : \subs{B}{x}{y_1}}{b}
        {\subs{P}{z}{\pair[A][x.B]{y_1}{y_2}}}}
      {\inferty{\Gamma}{\ind{\Sb}{s}{z.P}{y_1.y_2.b}}{\subs{P}{z}{s}}}
      \label{rule:pair-ind-bd} \and
        
    \inferdef{PairInf}{
      \inferty{\Gamma}{t}{T} \\ T \red \Sb x : A .\ B}
      {\pinferty{\Sb}{\Gamma}{t}{\Sb x : A .\ B}}
      \label{rule:sig-inf} 
  \end{mathpar}

  \caption{Bidirectional pair type}
  \label{fig:bidir-pair}
\end{figure}

To set ideas straight, let us look at how we can adapt the dependent pair type of
\cref{fig:sig} to the bidirectional setting: see \cref{fig:bidir-pair}.
To obtain these rules, first notice that all undirected typing rules for the
pair type (\cref{fig:sig})
must become inference rules if we want the resulting system to be complete.
The question therefore is once again to choose modes for the premises.
Rules \nameref{rule:pair-type-bd} and \nameref{rule:sig-inf} are
very similar to the rule for Π-types, there is not much surprise there.

\ruleref{rule:pair-bd} shows why we insisted in the undirected system
on recording the types $A$ and $B$ in the pair. Indeed, they are needed to
know which type to infer for the pair. Without the annotation, one could infer a
type $A$ for $t$ and a type $B'$ for $u$, but there are potentially many incomparable types $B$ that would be correct for the whole pair, depending on which instances of $t$ in $B'$ are abstracted to $x$. We only know that $B'$ is $\subs{B}{x}{t}$,
but this is not enough to inambiguously determine $B$.
This impossibility to invert a substitution is a general source of need
for annotations, which is not specific to pair types!

Finally, \ruleref{rule:pair-ind-bd} is the most complex.
In presentations of \kl{recursors}, often the predicate appears first, then the branches,
and finally the scrutinee. But this is not possible here, as the parameters of the inductive
type are needed to construct the context in which the predicate is typed.
Instead, those parameters can be inferred from the scrutinee.
Thus, a type for the scrutinee is first obtained using a new constrained inference judgment,
forcing the inferred type to be a Σ-type, but leaving its parameters free.
Next, these parameters can be used to construct the context to type the predicate.
And finally, once the predicate is known to be well-formed,
it can be used to type-check the branch.

This same approach can be readily extended to the other inductive types of
\cref{sec:tech-cic}, with recursion or indices posing no specific problems.
%, see \cref{fig:bidir-indu-other}.

\subsection{Polymorphic inductive types}

The account of general inductive types in \kl{PCUIC} is slightly different from
the one we just gave. The reason for this is that giving a general account of rules
which infer type levels like our \ruleref{rule:pair-type-bd} is not easy.
Indeed, the parameters of an inductive type can
be of a type much more complex than simply $\uni$, and in that general setting deciding which
type variable can be inferred is a non-trivial problem.
Instead, the polymorphic inductive types as implemented in \kl{Coq} store explicit universe
levels on inductive types and constructors. The  pair type of \cref{fig:bidir-pair},
for instance, would contain universe levels $i,j$, so that both $A$ and $B$
would be checked rather than having their level inferred.
The rule for the type constructor in that context is given opposite.
\begin{marginfigure}
  \begin{mathpar}
  \inferrule{
    \checkty{\Gamma}{A}{\uni[i]} \\ 
    \checkty{\Gamma, x : A}{B}{\uni[j]}}
    {\inferty{\Gamma}{\Sb\ulev{i,j} x : A .\ B}{\uni[\umax{i}{j}]}}
  \end{mathpar}
\end{marginfigure}
This makes the treatment of complex inductive types possible by using checking uniformly –
rather than relying on constrained inference to infer universe levels –
at the cost of possibly needless annotations, as here with Σ-types.
This is mostly invisible for the end user though, as she does very seldom write universe
levels thanks to \kl{typical ambiguity} anyway.

In the same spirit, pattern-matching in \kl{Coq} – and its counterpart in \kl{PCUIC} –
also stores enough information to easily reconstruct the context
in which the predicate and branches are typed. This information consists in universe levels
– for polymorphic inductive types – and parameters of the inductive type.
Thus, the actual typing rule for pattern-matching in the case of Σ-types
is closer to the following one:

\begin{mathpar}
  \inferrule{
    \pinferty{\Sb}{\Gamma}{s}{\Sb\ulev{i,j} x : A .\ B} \\
    i \le i' \\ j \le j' \\ A \cum A' \\ B \cum B' \\
    \pinferty{\uni}{\Gamma, z : \Sb x : A' .\ B'}{P}{\uni} \\
    \checkty{\Gamma, y_1 : A', y_2 : \subs{B'}{x}{y_1}}{b}
      {\subs{P}{z}{\pair[A'][x.B']{y_1}{y_2}}}}
    {\inferty{\Gamma}{\match{\Sb}{i',j'}{A',B'}{s}{z.P}{y_1.y_2.b}}{\subs{P}{z}{s}}}
\end{mathpar}  

Note that the domain and codomain are compared using \kl{cumulativity}. This is crucial
to retain \kl{subject reduction}. Indeed, reduction of the scrutinee might make its inferred
type decrease. For instance, suppose we have a polymorphic inductive $I\ulev{i}$ with a single
constructor $c$ such that $A : \uni[i] \vdash c\ulev{i}(A)$. Now consider
  \[ \left( \l y : I\ulev{1}.\ y \right)\ c\ulev{0}(\Nat)
    \ored c\ulev{0}(\Nat) \]
the redex infers type $I\ulev{1}$, while the reduct infers
$I\ulev{0}$. Thus, if such a term is plugged as scrutinee in a pattern-matching,
the whole term is still typeable after the reduction of
the scrutinee because we allow inequalities rather than equalities between levels.

But here lies a subtle issue: in pen-and-paper accounts of recursors,
the predicate and branches are often
represented respectively as Π-types and λ-abstractions. This is also how previous versions of
\kl{Coq} represented pattern-matching.%
\sidenote{Until version 8.14 to be precise.}
But recall that in \kl{PCUIC}, cumulativity is
equivariant on the domain of Π-types. This led to an implementation that wrongly compared
the universe levels using equality rather than inequality, leading to a completeness
bug that manifested as a failure of subject reduction in situations such as the one above.%
\sidenote{A precise description of the problem in the kernel and an example similar to the
  one above are given in issue \coqIssue{13495}.}
This prompted subsequent work, both on the theory of \kl{PCUIC} and on the implementation, to
remove the use of Π- and λ-abstractions completely from pattern-matching%
\sidenote{This was carried out by Pierre-Marie Pédrot starting with pull-request \coqPR{13563},
following ideas that had been laid down earlier by Hugo Herbelin in the
\href{https://github.com/coq/ceps/blob/master/text/inductive-branch-predicate-representation-and-reduction.md}{\kl{Coq} enhancement
proposal \#34}.},
making both the implementation less ad-hoc, and the theory cleaner.
A detailed summary has been given in \sidetextcite{Sozeau2022}.

Further investigations in that area might still be valuable though, in particular in order
to determine what kind of annotations are actually needed for pattern-matching, both
in theory and in practice. Can we give a presentation of polymorphic inductive types
that is as lightweight as pair types in \cref{fig:bidir-pair}?
The bidirectional presentation is valuable there, because now
it is clear what the specification of an alternative syntax is:
it should remain complete, in the sense of \cref{thm:comp-cumul}.