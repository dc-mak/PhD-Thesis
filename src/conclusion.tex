\bookmarksetup{startatroot}

\chapter{Future Directions}%
\label{chap:future-directions}

\margintoc{}

In this chapter, I will recap some of \kl{CN}'s more pressing limitations, and
sketch out some solutions to it.

\section{Inferring frames}

Recall the \kl{Core} operators for expressing most control-flow:
\coreinline{run()} and \coreinline{save()}, first presented % chktex 36
in~\nameref{sec:core-grammar}.

These very general constructs are used by \kl{Cerberus}' elaboration into
\kl{Core} to support a variety of control flow constructs: loops (continue,
break, iteration), switches (cases, fall-through and defaults), labels, and
\cinline{goto}.

\kl{CN} currently deals with these on an ad-hoc basis, such as by marking
labels with their original structure and use (loop continue, loop break,
return, switch, case, default) labels, and maximally inlining label bodies to
avoid requiring annotations at awkward places such as loop continues and breaks
and switch cases.

There are two main problems with this approach.

\begin{enumerate}
    \item \textbf{Inlining slows down performance and duplicates
        work.}\sidenote{\href{https://github.com/rems-project/cerberus/issues/289}{Cerberus\#289.}}
        The transformation from \kl{Core} with \coreinline{run()}  % chktex 36
        and \coreinline{save()} to \kl{Core} with non-recursive runs % chktex 36
        inlined and recursive runs hoisted to the top level induces some
        impressive code bloat. A nested-loop
        example,\sidenote{\href{https://github.com/rems-project/cerberus/issues/938}{Cerberus\#938.}}
        shows a 7x increase (226 lines of \kl{Core} to 1389 lines); a
        \cinline{switch} with 3 \cinline{case}s and a \cinline{default}
        example, shows a 5x increase.
    \item \textbf{Loop-annotations do not compose as expected, and are thus confusing and
        verbose.}\sidenote{\href{https://github.com/rems-project/cerberus/issues/913}{Cerberus\#931.}}
\end{enumerate}

The reason for this is the \kl{Core} dynamics for the
\coreinline{run()} operator. In particular, note that the current % chktex 36
continuation $C$ is discarded, and the continuation associated with the label
$\cnnt{id}$, $C_\cnnt{id}$ is resumed.

{\small%
\[
\inferrule[{[Run]}]
  { \mathrm{labelmap} ( \cnnt{id} ) = \left(\cncomp{x_i}{i}\right).\ C_{\cnnt{id}} [ E_{\cnnt{id}} ] \\
    \cncomp{e_i \Downarrow \cnnt{value}_i}{i} }
  { \langle h , C\left[ \cnkw{run}\, \cnnt{id} \left( \cncomp{e_i}{i} \right) \right] , \kappa \rangle
    \rightarrow \left< h , C_{\cnnt{id}} \left[ \left[ \cncomp{\cnnt{value}_i / x_i}{i} \right] E_{\cnnt{id}} \right] , \kappa \right> }
\]}

Its associated (\kl{ResCore}) typing rule is as follows. Note that because
control-flow does not return to this point in the program, the return type is
$\cnkw{false} \wedge \cnkw{I}$, even if, for example, a program is entering and
exiting a loop in a well-bracketed way. This has the effect of requiring the
precondition of the label to consume \emph{all} resources from the context
of a \coreinline{run()}. % chktex 36

{\small%
\[
\cndruleStmtXXRun{}
\]}

% \onlyUseRules{\cndefnStmt{}}{
%     \cndruleStmtXXRun{}
% }

In particular, this means that loop invariants must explicitly mention all
resources in a context (even ones which are morally framed out), so that they
are reinstated upon loop exit. The situation compounds for each level of
loop nesting: invariants of all enclosing loops must be manually threaded through
by the user
(see~\href{https://github.com/rems-project/cerberus/issues/913}{Cerberus\#931}).
The problem holds for all \kl{Core} labels, but only manifests itself at loops
since \kl{CN} inlines all labels whose bodies do not recursively \coreinline{run()} % chktex 36
to themselves.

Whilst it may be possible to engineer \kl{CN} for the common cases,
for example by extending \kl{Core} with specific loop or switch constructs,
the root of the problem will remain to handle labels and \cinline{goto}s in C.

Hence a prinicipled solution is desirable, and would be useful to ensure
our handling of the common cases is sound.

A \emph{potential} solution is to adjust the grammar and elaboration of
\kl{Core} to borrow an idea from an inductive representation of SSA, based on
work by~\sidetextcite{ghalayini2024denotational}: represent
\emph{dominance-based scoping} as \emph{lexical scoping}.

\begin{quote}
\emph{%
In particular, a variable $x$ is considered to be in scope at a specific point
$P$ if and only if all execution paths from the program’s entry point to $P$
pass through a definition $D$ for $x$. In this case, we say that the definition
$D$ \emph{strictly dominates} $P$. The relation on basic blocks ``$A$ strictly
dominates $B$'' intersected with ``$A$ is a direct predecessor of $B$'' forms a
tree called the \emph{dominance tree} of the [control-flow graph].}
\end{quote}

More specifically, replace $\cnkw{save}\ \cnnt{id}\ (x\, {:}{=}\, \cnnt{pce})\ \cnkw{in}\ E$
with a $\cnkw{where}$ expression:
\[
    \cnnt{E}\ \cnkw{where}\ \left( \cncomp{\cnnt{id}_i\ (x_i {:} \beta_i).\ E_i}{i} \right).
\]

Labels $\cnnt{id}_i$ are in scope in $E$ and in each $E_i$, allowing for mutual
recursion, but are encapsulated from the enclosing expression. This is unlike
\kl{Core}'s \coreinline{save()} operator, whose label is in scope of % chktex 36
the enclosing \emph{procedure}. Variables defined in $E$ are not in scope
in $E_i$, which are parameterised over (for simplicity) a single variable
$x_i$. Most importantly, the enclosing expression is its \emph{frame}, and
unlike in \kl{Core}, is preserved in the dynamic semantics.

% {\small%
% \[
% \inferrule[{[OpStmtWhere]}]
%   { l_1 \eqdef \cncomp{\cnnt{id}_i\ (x_i {:} \beta_i).\ E_i}{i} }
%   { \langle h , E\ \cnkw{where}\ l, w, \kappa \rangle
%     \rightarrow \left< h , \cnkw{pop}_l (E) , l \Colon w, \kappa \right> }
% \]}

{\small%
\[
\inferrule[{[OpStmtRunW]}]
  { \cnnt{id}\ (x {:} \beta).\ E \in l
    \and \cnnt{pce} \Downarrow \cnnt{value} }
    { \langle h , C_1 \left[ C_2 \left[ \cnkw{run}\, \cnnt{id}\,\cnnt{pce} \right]\ \cnkw{where}\ l \right], \kappa \rangle
    \rightarrow \left< h , C_1 \left[ \left[ \cnnt{value} / x \right] E\ \cnkw{where}\ l \right], \kappa \right> }
\]}

First, a $\cnkw{where}$ expression is a context $\cnnt{C} \, {:}{=} \, \ldots
\mid C\ \cnkw{where}\ l$; $l$ is a short-hand for the family of labels
$\cncomp{\cnnt{id}_i\ (x_i {:} \beta_i).\ E_i}{i}$. When $\cnkw{run}$ning to a
label, we search the context for its defintion $l$ and jumps to its body with
paramenter substituted $[\cnnt{value} / x] E$, keeping the enclosing context
$C_1$, but discarding the intervening context $C_2$.

Note that, \sidetextcite{ghalayini2024denotational} stratify their language
into statements and expressions; their $\cnkw{where}$ construct is a statement,
so the value it computes is never bound. Here, a $\cnkw{pure}(\cnnt{value})$ expression
exits a $\cnkw{where}$ expression straightforwardly.

{\small%
\[
\inferrule[{[OpStmtPureW]}]
  { }
  { \langle h , \cnkw{pure}(\cnnt{value})\ \cnkw{where}\ l, \kappa \rangle
    \rightarrow \left< h , \cnkw{pure}(\cnnt{value}), \kappa \right> }
\]}

As for typing, I conjecture the $\cnkw{where}$ contexts in the dynamics would be
mirrored in an \emph{ordered} resource context $\mathcal{W}$, consisting of
linear resource contexts $\mathcal{R}$, separated by label definitions $l$ as
ordering markers. Typing these definitions requires threading such a
$\mathcal{W}$ through.

{\small%
\[
\inferrule[{[WhereDefns]}]
  { \cncomp{\cnnt{fun}_i \rightsquigarrow \mathcal{C}_i; \mathcal{L}_i; \Phi_i; \mathcal{R}_i}{i}
    \\ \cncomp{\mathcal{C}, \mathcal{C}_i; \mathcal{L}, \mathcal{L}_i; \Phi, \Phi_i; \mathcal{R}_i \Colon \mathcal{W} \vdash E_i \Leftarrow \cnnt{ret}}{i} }
  { \mathcal{C}; \mathcal{L}; \Phi; \mathcal{W} \vdash \cncomp{\cnnt{id}_i {:} \cnnt{fun}_i.\ E_i}{i} \Leftarrow \cnnt{ret} }
\]}

Typing a $\cnkw{where}$ expression itself would pass any resources it has to the
left expression, and adding the definitions as an ordering marker to bring them
in scope. The same is done for checking the definition bodies too, because
labels can refer to each other (mutually) recursively.

{\small%
\[
\inferrule[{[StmtWhere]}]
  { \mathcal{C}; \mathcal{L}; \Phi; \mathcal{R} \Colon l \Colon \mathcal{W} \vdash E \Leftarrow \cnnt{ret}
    \and \mathcal{C}; \mathcal{L}; \Phi; l \Colon \mathcal{W} \vdash l \Leftarrow \cnnt{ret} }
  { \mathcal{C}; \mathcal{L}; \Phi; \mathcal{R} \Colon \mathcal{W} \vdash{} E\ \cnkw{where}\  l \Leftarrow \cnnt{ret} }
\]}

With this, I can sketch out a rough idea on typing for a $\cnkw{run}$ to any
label in scope. First any resources in the intervening depths would need to
disposed, for example, removing the storage for block-local variables. After
that, the $\cnnt{spine}$ would be checked against the function type
$\cnnt{fun}$ (obtained via a lookup of the label $\cnnt{id}$ in the ordered
resource context). Only the resources at the same level as the target
$\cnkw{where}$ expression need to be consumed, everything above it is `framed'
when checking the label body.

{\small%
\[
\inferrule[{[StmtRun]}]
  { \mathcal{W} \eqdef \mathcal{R}_1 \Colon l_1 \Colon \ldots \Colon \mathcal{R}_k \Colon l_k \Colon \ldots \Colon \mathcal{R}_n \Colon l_n
    \and \cnnt{id} {:} \cnnt{fun}.\ E' \in l_k
    \\ \mathcal{C}; \mathcal{L}; \Phi; \mathcal{R}_1 , \ldots, \mathcal{R}_{k-1} \vdash E \Leftarrow \Sigma y{:} \cnkw{unit}.\ \cnkw{I}
    \\ \mathcal{C} ; \mathcal{L} ; \Phi ; \mathcal{R}_k \vdash \cnnt{spine} \Colon \cnnt{fun} >\!> \outpol{\sigma}; \outpol{\cnkw{false} \wedge \cnkw{I}} }
    { \mathcal{C}; \mathcal{L}; \Phi; \mathcal{W} \vdash{} \cnkw{let\ strong\ ()\ =}\ E\ \cnkw{in}\ \cnkw{run}\ \cnnt{id}\,\cnnt{spine} \Leftarrow \cnkw{false} \wedge \cnkw{I} }
\]}

As in~\arefpart{formalisation}, the aboves rules are phrased over a
hypothetical, explicitly annotated \kl{ResCore} with this new construct. Whilst
the setup enables inferring frames for a label, inferring preconditions
$\cnnt{fun}$ remains a distinct problem. As a first approximation, loops and
mutually recursive labels would require annotations from the user, though
annotations on loops would no longer need to thread through framed resources.
Other labels could have their preconditions inferred by saving the typing
context when the first $\cnkw{run}$ is encountered, and ensuring that
subsequent ones match the first, thus preventing inlining and code bloat.

\section{Weak sequencing}

Sequencing strength is one of the most subtle and confusing aspects about C,
and its treatment in \kl{Core}, whilst inventive, is difficult to understand
and reason about.

Sequencing strength is used to specify the (lack of) ordering between memory
actions. Quoting at length from \sidetextcite{memarian2022cerberus}:

\begin{quote}
\emph{%
Each use of an action is either ``positive'' (the default case) or
``negative'' (if the action appears in the syntax as the operand of the
$\cnkw{neg()}$ operator). [..] Intuitively, a \kl{Core} action is negative
when it elaborates what the C standard calls a ``side-effect'' (as opposed to
value computations), that is, memory accesses which are not directly used for
producing the value of a C expression. This is for example, the store
performed by a postfix increment operator.}
\end{quote}

The $\cnkw{let}\ \cnkw{strong}\ \cnnt{pat}\ \cnkw{=}\ \cnnt{E}_1\ \cnkw{in}\
\cnnt{E}_2$ construct sequences all memory actions performed by $\cnnt{E}_1$
before those performed by $\cnnt{E}_2$. However, $\cnkw{let}\ \cnkw{weak}$ only
sequences positive memory actions similarly; in other words, for $\cnkw{let}\
\cnkw{weak}\ \cnnt{pat} = \cnnt{E}_1\ \cnkw{in}\ \cnnt{E}_2$, any
negative memory actions performed by $\cnnt{E}_1$ are unsequenced with respect
to all memory actions performed by $\cnnt{E}_2$. A \emph{race} occurs if
unsequenced memory actions have overlapping footprints, and is deemed \kl{UB}.

The formalisation and implementation of weak sequencing is complex. Here, I
offer a simpler presentation which I conjecture is equivalent and then offer
few remarks on how to adapt the \kl{CN} resource framework to fit it.

A footprint $\cnnt{fp}$ of an action is the range of bytes in memory it touches.
Like actions, footprints can be positive ${\color{red}\cnkw{pos}(\cnnt{fp})}$
or negative ${\color{red}\cnkw{neg}(\cnnt{fp})}$. A set of such footprints is
called an annotation ${\color{red}A}$, and I use the notation
${\color{red}A^{-}}$ to denote all the negative footprints in ${\color{red}A}$
and ${\color{red}A^{+}}$ for all the positive ones.

Negative actions step to a new $\cnkw{delay}(\cnnt{actions}, \cnnt{E})$
construct, where $\cnnt{actions}$ is a set. Note that the heap $h$
does not change (yet).

{\small%
\[
\inferrule[{[ActionNeg]}]
    { h \overset{\cnnt{action}}{\longrightarrow} \langle \_ , \cnkw{Unit}, \cnnt{fp} \rangle }
    { \langle h, \cnkw{neg}(\cnnt{action}), \kappa \rangle \rightarrow
      \langle
          h,
          \cnkw{delay}(\{\, \cnnt{action} \, \},
            \prescript{\color{red}\{\cnkw{neg}(\cnnt{fp})\}}{}{\cnkw{pure}(\cnkw{Unit})}),
          \kappa
      \rangle }
\]}

The dynamics of this construct allowing reducing either one of the actions in
the set, or the expression inside at any time; the latter because it counts as
a context $\cnnt{C} \, {:}{=} \, \ldots \mid \cnkw{delay}(\cnnt{actions}, C)$.
All $\cnkw{delay}$ed actions are negative, and all negative actions evaluate to
$\cnkw{Unit}$ (since they are side effects) and so the result of the action is
ignored. The footprint is unnecessary as per the previous rule.

{\small%
\[
\inferrule[{[DelayAction]}]
    { h \overset{\cnnt{action}}{\longrightarrow} \langle h' , \cnkw{Unit}, \_ \rangle }
    { \langle
          h,
            \cnkw{delay}(\cnnt{actions} \cup \{\, \cnnt{action} \,\}, E),
          \kappa
      \rangle
      \rightarrow
      \langle h', \cnkw{delay}(\cnnt{actions}, E), \kappa \rangle }
\]}

The next ingredient is a $\cnkw{disj}({\color{red}A}, E)$ construct, which is
used to check if ${\color{red}A}$ races with $E$.

{\small%
\[
\inferrule[{[Disj]}]
    { {\color{red}\neg \mathrm{do\_race}(A_1, A_2)} }
    { \langle
          h,
          \cnkw{disj}({\color{red}A_1}, \prescript{\color{red}A_2}{}{\cnkw{pure}(\cnnt{value})}),
          \kappa
      \rangle
      \rightarrow
      \langle
          h,
          \prescript{\color{red}A_1 \cup A_2}{}{\cnkw{pure}(\cnnt{value})},
          \kappa
      \rangle }
\]}

The construct is used for weak sequencing, and is also a context $\cnnt{C} \,
{:}{=} \, \ldots \mid \cnkw{disj}({\color{red}A}, C)$, allowing for the
expression inside to reduce. Splitting the annotations, placing the positive
ones on the outside and the negative ones in a $\cnkw{disj}$ ensures the
negative actions which \emph{were} part of evaluating $\cnkw{pure}(\cnnt{value})$ are
treated as if unsequenced with $E$, i.e.\ required to not race.

{\small%
\[
\inferrule[{[Weak]}]
    {  }
    { \langle
          h,
          \cnkw{let}\ \cnkw{weak}\ \cnnt{pat}\ =\ \prescript{\color{red}A}{}{\cnkw{pure}(\cnnt{value})}\ \cnkw{in}\ E,
          \kappa
      \rangle
      \rightarrow
      \\ \quad
      \langle
          h,
          \prescript{\color{red}A^{+}}{}{\cnkw{disj}({\color{red}A^{-}}, [ \cnnt{value} / \cnnt{pat} ] E)},
          \kappa
      \rangle }
\]}

Nested $\cnkw{delay}$s can be combined by taking the union of the actions.

{\small%
\[
\inferrule[{[HoistDelay]}]
    {  }
    { \langle h, \cnkw{delay}(\cnnt{actions}_1, \cnkw{delay}(\cnnt{actions}_2, E)), \kappa \rangle
      \rightarrow
      \\ \quad
      \langle h, \cnkw{delay}(\cnnt{actions}_1 \cup \cnnt{actions}_2, E), \kappa \rangle }
\]}

Cruicially, $\cnkw{delay}$s can be hoisted out of $\cnkw{let}\ \cnkw{weak}$,
$\cnkw{unseq}$, $\cnkw{disj}$ and annotated $\prescript{\color{red}A}{}{E}$
expressions. The rule for the first is given below, and the rest follow a
similar pattern.

{\small%
\[
\inferrule[{[HoistWeak]}]
    {  }
    { \langle
          h,
          \cnkw{let}\ \cnkw{weak}\ \cnnt{pat}\ =\ \cnkw{delay}(\cnnt{actions}, E_1)\ \cnkw{in}\ E_2,
          \kappa
      \rangle
      \rightarrow
      \\ \quad
      \langle
          h,
          \cnkw{delay}(\cnnt{actions}, \cnkw{let}\ \cnkw{weak}\ \cnnt{pat}\ =\ E_1\ \cnkw{in}\ E_2),
          \kappa
      \rangle }
\]}

Notably absent, $\cnkw{let}\ \cnkw{strong}$ or $\cnkw{bound}$ expressions
\emph{do not} permit hoisting $\cnkw{delay}$s out of them. This forces any
delayed actions to be evaluated and any races to checked before proceeding.

Though unusual, the constructs and rules are simple. One straightforward
approach to typing could be to split resource types into positive and negative.
Negative resources are not usable, and are converted into positive ones after a
$\cnkw{let}\ \cnkw{strong}$ or a $\cnkw{bound}$ expression.

\section{Join-points}

Currently, \kl{CN} bifurcates control-flow during type checking at every
branch, with no way to merge paths once this is done. The main issue with this
is that any resource inference which is not path dependent repeated across
different branches.

Whilst it is possible to define syntax and rules for user-provided join-points,
the better solution may be to not branch the control-flow in the first place.

This requires (a) defining how to merge constraint and resource contexts after
an if-expression and (b) defining how to handle resource lookup/inference with
conditional-resources in the context whilst avoiding performance slowdown.

Assume a judgement of the form $\Phi ; \mathcal{R} \vdash E \dashv \cnnt{term}
\mathrel{\&} \Phi' ; \mathcal{R}'$ where $\Phi$ is the input constraint
context, $\mathcal{R}$ is the input resource context, context, $E$ is the input
\kl{Core} expression, $\cnnt{term}$ is the output symbolic evaluation of the
\kl{Core} expression, $\Phi' ; \mathcal{R}'$ are the output constraint and
resource contexts respectively (to reduce clutter, I am omitting variable
environments).

We can define join-points for conditional expressions as below.

{\small%
\[
\inferrule[{[StmtIf]}]
  { \Phi \wedge \cnnt{pce} ; \mathcal{R} \vdash E_1 \dashv \cnnt{term}_1 \mathbin{\&} \Phi_1 ; \mathcal{R}_1
    \\ \Phi \wedge \neg \cnnt{pce} ; \mathcal{R} \vdash E_2 \dashv \cnnt{term}_2 \mathbin{\&} \Phi_2 ; \mathcal{R}_2
    \\ \cnnt{term}_3 \eqdef \cnkw{if}\ \cnnt{pce}\ \cnkw{then}\ \cnnt{term}_1\ \cnkw{else}\ \cnnt{term}_2
    \\ \Phi' \eqdef (\cnnt{pce} \rightarrow \Phi_1) \wedge ( \neg \cnnt{pce} \rightarrow \Phi_2)
    \\ \mathcal{R}' \eqdef (\mathcal{R}_1 \cap \mathcal{R_2})
       \ast (\cnkw{if}\ \cnnt{pce}\ \cnkw{then}\ (\mathcal{R}_1 - \mathcal{R}_2)\ \cnkw{else}\ (\mathcal{R}_2 - \mathcal{R}_1)) }
  { \Phi ; \mathcal{R}
    \vdash \cnkw{if}\ \cnnt{pce}\ \cnkw{then}\ E_1\ \cnkw{else}\ E_2
    \dashv \cnnt{term}_3 \mathbin{\&} \Phi' ; \mathcal{R}' }
\]}

Calculating the intersection and difference of two resource contexts, should be
done carefully to avoid having it be accidentally quadratic in complexity.
Marking which resources are not used will also help, by framing them out and
thus reducing the number of items considered for intersection and difference.
Calculating the intersection precisely is not necessary for soundness; its
primary benefit is to reduce work later.


The problem then becomes how to use a conditional-resource in the context.
Specifically, how to handle the below situation without resorting to an
exponential blow-up. Inputs are constraints $\Phi$, resources $\mathcal{R}$ and
the requested resource $r$; Outputs are the remaining lookup $\mathcal{R}'$,
and the leftover context $\mathcal{R}''$ after a partially successful lookup.

{\small%
\[
    \Phi \vdash r \mathrel{{\in}{?}} \mathcal{R} \rightsquigarrow \outpol{\mathcal{R}'; \mathcal{R}''}
\]}

The first thing to check would be \emph{nominal} equality. A conditional
resource may come from what a user wrote (in a specification or a predicate
definition) and so can be identified as an anonymous predicate of sorts:
$\cnnt{\#loc}\;(\ldots)$. In this case, it would suffice to check equality on
source location and arguments.

{\small%
\[
\inferrule[{[StmtIf-Nominal]}]
    { }
    { \Phi \vdash \cnnt{\#loc}\;(\cnnt{iargs})
            \mathrel{{\in}?}
            (\mathcal{R}, \cnnt{\#loc}\;(\cnnt{iargs}))
            \rightsquigarrow \cdot ; \mathcal{R} }
\]}

If the nominal equality fails, the next thing to check would be
\emph{structural} equality, if either $\cnnt{pce}_1 \leftrightarrow
\cnnt{pce}_2$ or $\cnnt{pce}_1 \leftrightarrow \neg \cnnt{pce}_2$, then the
check can recurse on both branches.

{\small%
\[
\inferrule[{[StmtIf-TwoSided1]}]
    { \Phi \vdash \cnnt{pce}_1 \leftrightarrow \cnnt{pce}_2
      \\ \Phi \wedge \cnnt{pce}_1 \vdash \mathcal{R}_{21}
      \mathbin{{\in}?} (\mathcal{R} , \cnnt{r.then} {:} \mathcal{R}_{11}) \rightsquigarrow \cdot ; \mathcal{R}
      \\ \Phi \wedge \neg \cnnt{pce}_1 \vdash \mathcal{R}_{22}
      \mathbin{{\in}?} (\mathcal{R} , \cnnt{r.else} {:} \mathcal{R}_{12}) \rightsquigarrow \cdot ; \mathcal{R} }
    { \Phi \vdash (\cnkw{if}\ \cnnt{pce}_2\ \cnkw{then}\ \mathcal{R}_{21}\ \cnkw{else}\ \mathcal{R}_{22})
           \mathbin{{\in}?}
           (\mathcal{R} , r {:} \cnkw{if}\ \cnnt{pce}_1\ \cnkw{then}\ \mathcal{R}_{11}\ \cnkw{else}\ \mathcal{R}_{12})
           \rightsquigarrow \cdot ; \mathcal{R} }
\]}

And similarly for $\cnnt{pce}_1 \leftrightarrow \neg \cnnt{pce}_2$. If a
two-sided match fails, then the next thing to check would be a one-sided
strucural equality.

{\small%
\[
\inferrule[{[StmtIf-OneSided1]}]
    { \Phi \vdash \cnnt{pce}_1 \rightarrow \cnnt{pce}_2
      \\ \Phi \wedge \cnnt{pce}_1 \vdash \mathcal{R}_{21}
      \mathbin{{\in}?} (\mathcal{R} , \cnnt{r.then} {:} \mathcal{R}_{11}) \rightsquigarrow \cdot ; \mathcal{R} }
    { \Phi \vdash (\cnkw{if}\ \cnnt{pce}_2\ \cnkw{then}\ \mathcal{R}_{21}\ \cnkw{else}\ \mathcal{R}_{22})
           \mathbin{{\in}?}
           (\mathcal{R} , r {:} \cnkw{if}\ \cnnt{pce}_1 \cnkw{then}\ \mathcal{R}_{11}\ \cnkw{else}\ \mathcal{R}_{12})
           \\ \quad \rightsquigarrow (\cnkw{if}\ \cnkw{pce}_2\ \cnkw{emp}\ \cnkw{else}\ \mathcal{R}_{22}) ; \mathcal{R} }
\]}

Similarly we can test $\cnnt{pce}_1 \rightarrow \neg \cnnt{pce}_2$, $\neg
\cnnt{pce}_1 \rightarrow \cnnt{pce}_2$ or $\neg \cnnt{pce}_1 \rightarrow \neg
\cnnt{pce}_2$, and proceed accordingly.

If all of these fail, then the inference fails, or requires the user to give
explicit permission to split on the condition of the resource and duplicate the
lookup across two branched resource contexts. Notably, the permission is given
for a \emph{resource} for every lookup, and not a condition.

% This might be necessary if there's a situation a bit like the below:
% ```
%            r1                  r2
%          /    \              /    \
%    r1.then    r1.else   r2.then r2.else
%         |        \        /       |
%         |         req.then        |
%         |-------\          /------|
%                  \        /
%                   req.else
% ```
%
% In this instance, the user would need to give permission to split on both `r1`
% and `r2` separately, if the their conditions lined up suitably.
%
% If the resource is consumed from the context (e.g. by a lemma or a function
% call) then there will be no more splitting.

Hence the user opts-in to the slow case, if it happens to be the right thing to
do. If there's a smarter way to eliminate conditional-resources from the
context, the user can write a lemma/function to do so.

Thus we have a scheme of automating the fast common cases, explicitly opting-in
to the slow general case, \emph{and} a mechanism for the user to intervene with
something else should it be necessary.

\section{Integers and bitvectors}

Sometimes specifications are written and proved over integers, for example,
with the buddy allocator, but need to be called from the bit-vector world, and
sometimes vice-versa.

Integers are unbounded and bit-vectors are bounded. Mapping from bit-vectors to
integers is straightforward: insert $\cnnt{term} \mod 2^k$ as needed, as long
as accepts the resulting expression may not being decidable/interpretable: $x
\times y$ in bit-vectors will be $\mathrm{mul\_uf}(x,y) \mod 2^{64}$ in
integers. We can apply this transformation and get an expression which will
give the same answer.

Going from integers to bit-vectors is difficult because the process is
fundamentally partial, i.e.\ it is only defined for a certain inputs under
certain conditions, however SMT expressions must be total.

Let $\beta$ be the basetypes in the bounded bit-vector world. Let $\iota$ be
the basetypes in the unbounded integer world. Let $\cnkw{num} \mathrel{{:}{=}}
\cnkw{int} \mid \cnkw{bit}$ be a parameter for the following. Let
$\cnnt{term}_{\cnkw{num}}$ be the grammar of SMT terms, parameterised over the
choice of integers or bit-vectors.

{\small%
\begin{align*}
\cnnt{term}_{\cnkw{num}}, t_{\cnkw{num}}, b_{\cnkw{num}} \mathrel{{:}{=}}
\, & x \mid t_{\cnkw{num}} \oplus t'_{\cnkw{num}} \mid
\cnkw{let}\ x\ = t_{\cnkw{num}}\ \cnkw{in}\ t'_{\cnkw{num}} \mid \\
& \cnkw{if} \, (b_{\cnkw{num}}) \,\{\, t_{\cnkw{num}} \,\}\, \cnkw{else} \,\{\, t'_{\cnkw{num}} \,\} \mid
f(t^1_{\cnkw{num}}, \ldots, t^n_{\cnkw{num}})
\end{align*}}

Under this convention, closed terms $t_{\cnkw{int}} {:} \iota$ and
$t_{\cnkw{bit}} {:} \beta$. I use $b$ for formulas of boolean type:
$b_{\cnkw{bit}} {:} \cnkw{bool}$.

Let $\mathrm{Decls}$ be a context with declaration types of pure SMT functions
(in CN, defined with the $\cnkw{function}$ keyword). The judgement
$\mathrm{Decls} ; \Gamma \vdash t_{\cnkw{int}} \Rightarrow \iota$  means that
under declarations $\mathrm{Decls}$ and context $\Gamma$, expression in the
integer world $t_{\cnkw{int}}$ has type $\iota$. $\mathrm{Decls}$ is a map from
function names to function types, $\Gamma$ is a map from parameters and local
variables to types.

{\small%
\[
\inferrule[]
  { \mathrm{Decls} , ( f {:} ( \iota_1 \times \cdots \times \iota_n \rightarrow \iota ) ) ; x_1 {:} \iota_1 , \ldots , x_n {:} \iota_n \vdash t_{\cnkw{int}} \Rightarrow \iota }
  { \mathrm{Decls} ; \cdot \vdash \cnkw{function}\ (\iota)\ f(\iota_1 x_n, \ldots, \iota_n x_n) \, \{ \, t_{\cnkw{int}} \,\} \Rightarrow \iota_1 \times \cdots \times \iota_n \rightarrow \iota }
\]}

For bit-vectors, the setup is similar, except recursive functions, have named
recursive constraints, named $f_b$.

Note that the bottom-rule has \emph{two} function definitions: the first one is
the precondition, assuming which, the second gives the same result as the
original (integer-world) function.

{\small%
\[
\inferrule[]
  { \mathrm{Decls}',
        {\begin{matrix}
            ( f' {:} ( \beta_1 \times \cdots \times \beta_n \rightarrow \beta ) ),
            \\ ( f'_b {:} ( \beta_1 \times \cdots \times \beta_n \rightarrow \cnkw{bool} ) )
        \end{matrix}} ; x_1 {:} \beta_1, \ldots , x_n {:} beta_n \vdash t_{\cnkw{bit}} \Rightarrow \beta }
  { \mathrm{Decls} ; \cdot \vdash
      {\begin{matrix}
      \cnkw{function}\ (\beta)\ f'_b(\beta_1 x_n, \ldots, \beta_n x_n) \, \{ \, b_{\cnkw{bit}} \,\}
      \\ \cnkw{function}\ (\beta)\ f'(\beta_1 x_n, \ldots, \beta_n x_n) \, \{ \, t_{\cnkw{bit}} \,\}
      \end{matrix}}
     \Rightarrow
     {\begin{matrix}
     \beta_1 \times \cdots \times \beta_n \rightarrow \cnkw{bool}
     \\ \beta_1 \times \cdots \times \beta_n \rightarrow \beta
     \end{matrix}} }
\]}

I will use $\rightsquigarrow$ to define a mapping from the integer world to the
bit-vector world. For example, the mapping between types, from $\iota$ to
$\beta$, would be along the lines of the below.

{\small
\begin{align*}
    \cnkw{integer}   &\rightsquigarrow \cnkw{i64} \\
    \cnkw{unit}      &\rightsquigarrow \cnkw{unit} \\
    \cnkw{alloc\_id} &\rightsquigarrow \cnkw{alloc\_id} \\
    \{\, \cnkw{integer}\ \cnnt{addr}, \cnkw{alloc\_id}\ \cnnt{id} \,\}
        &\rightsquigarrow \{\, \cnkw{i64}\ \cnnt{addr}, \cnkw{alloc\_id}\ \cnnt{id} \,\} \\
    \cnkw{map}\ \iota_1\ \iota_2 &\rightsquigarrow \cnkw{map} \beta_1\ \beta_2
                                 \ \textrm{where}\ \iota_1 \rightsquigarrow \beta_1, \iota_2 \rightsquigarrow \beta_2
\end{align*}}

For typing judgments, variables are typed with a lookup in the context.

{\small
\[
\inferrule[]
  { \Gamma ( x ) = \iota \rightsquigarrow \Gamma' ( x ) = \beta }
  { ( \mathrm{Decls} ; \Gamma \vdash x \Rightarrow \iota ) \rightsquigarrow ( \mathrm{Decls}' ; \Gamma' \vdash x \Rightarrow ( \beta , \cnkw{true} ) ) }
\]}

Typing for binary operation is translated by inserting no-overflow
constraints.\sidenote{\url{https://github.com/Z3Prover/z3/discussions/5138\#discussioncomment-544642}}
I will omit the kind annotation on terms to reduce syntactic noise, using $t$
for $t_{\cnkw{num}}$ and $t'$ for $t_{\cnkw{bit}}$. The context should also
make it clear which is intended.

{\small%
\[
\inferrule[]
    { ( \mathrm{Decls} ; \Gamma \vdash t_1 \Rightarrow \cnkw{int} ) \rightsquigarrow ( \mathrm{Decls} ; \Gamma' \vdash t'_1 \Rightarrow ( \cnkw{i64}, b_1 ) )
    \\ ( \mathrm{Decls} ; \Gamma \vdash t_2 \Rightarrow \cnkw{int} ) \rightsquigarrow ( \mathrm{Decls} ; \Gamma' \vdash t'_2 \Rightarrow ( \cnkw{i64}, b_2 ) ) }
    { ( \mathrm{Decls} ; \Gamma \vdash t_1 \oplus t_2 \Rightarrow \cnkw{int} ) \rightsquigarrow
    \\ \quad ( \mathrm{Decls}' ; \Gamma' \vdash t'_1 \oplus t'_2 \Rightarrow ( \cnkw{i64} , b_1 \wedge b_2 \wedge \mathrm{no\_ovfl}(\oplus, t'_1, t'_2) ) ) }
\]}

Typing for let-expressions is translated as follows. The precondition for the
whole expresison consists of the precondition for the bound expression, and for
the same for the body, which may refer to the bound variable.

{\small%
\[
\inferrule[]
    { ( \mathrm{Decls} ; \Gamma \vdash t_1 \Rightarrow \iota_1 ) \rightsquigarrow ( \mathrm{Decls}' ; \Gamma' \vdash t'_1 \Rightarrow ( \beta_1, b_1 ) )
    \\ ( \mathrm{Decls} ; \Gamma , x {:} \iota_1 \vdash t_2 \Rightarrow \iota_2 ) \rightsquigarrow ( \mathrm{Decls} ; \Gamma' , x {:} \beta_1 \vdash t'_2 \Rightarrow ( \beta_2, b_2 ) ) }
    { ( \mathrm{Decls} ; \Gamma \vdash \cnkw{let}\ x = t_1\ \cnkw{in}\ t_2 \Rightarrow \iota_2 ) \rightsquigarrow
    \\ \quad ( \mathrm{Decls}' ; \Gamma' \vdash \cnkw{let}\ x = t'_1\ \cnkw{in}\ t'_2 \Rightarrow ( \beta_2 , b_1 \wedge (\cnkw{let}\ x = t'_1\ \cnkw{in}\ b_2) ) ) }
\]}

Typing for if-expressions is translated as follows. The evaluation of the
branch condtion $t'_1$ is constrained by $b_1$, and the evaluation of each
branch is constrained similarly. Note that the precondition refers to $t'_1$
too.

{\small%
\[
\inferrule[]
    { ( \mathrm{Decls} ; \Gamma \vdash t_1 \Rightarrow \cnkw{bool} ) \rightsquigarrow ( \mathrm{Decls}' ; \Gamma' \vdash t'_1 \Rightarrow ( \cnkw{bool}, b_1 ) )
    \\ ( \mathrm{Decls} ; \Gamma \vdash t_2 \Rightarrow \iota ) \rightsquigarrow ( \mathrm{Decls}' ; \Gamma' \vdash t'_2 \Rightarrow ( \beta, b_2 ) )
    \\ ( \mathrm{Decls} ; \Gamma \vdash t_3 \Rightarrow \iota ) \rightsquigarrow ( \mathrm{Decls}' ; \Gamma' \vdash t'_3 \Rightarrow ( \beta, b_3 ) ) }
    { ( \mathrm{Decls} ; \Gamma \vdash \cnkw{if}\, (t_1)\, \{\, t_2 \,\}\, \cnkw{else} \,\{\, t_3 \,\}\, \Rightarrow \iota ) \rightsquigarrow
    \\ \quad ( \mathrm{Decls}' ; \Gamma' \vdash \cnkw{if}\, (t'_1)\, \{\, t'_2 \,\}\, \cnkw{else} \,\{\, t'_3 \,\}\, \Rightarrow ( \beta , b_1 \wedge (\cnkw{if}\ t'_1\ \cnkw{then}\ b_2\ \cnkw{else}\ b_3) ) ) }
\]}

Lastly, function names, which may be recursive, are handled similarly to variables, and
their arguments by recursion.

{\small%
\[
\inferrule[]
    { \mathrm{Decls}( f ) = \iota_1 \times \cdots \times \iota_n \rightarrow \iota
    \\ \mathrm{Decls}'( f' ) = \beta_1 \times \cdots \times \beta_n \rightarrow \beta
    \\ \mathrm{Decls}'( f'_b ) = \beta_1 \times \cdots \times \beta_n \rightarrow \cnkw{bool}
    \\ ( \mathrm{Decls} ; \Gamma \vdash t_1 \Rightarrow \iota_1 ) \rightsquigarrow ( \mathrm{Decls}' ; \Gamma' \vdash t'_1 \Rightarrow ( \beta_1 , b_1) )
    \\ \cdots ( \mathrm{Decls} ; \Gamma \vdash t_n \Rightarrow \iota_n ) \rightsquigarrow ( \mathrm{Decls}' ; \Gamma' \vdash t'_n \Rightarrow ( \beta_n , b_n) ) }
    { ( \mathrm{Decls} ; \Gamma \vdash f ( t_1, \ldots, t_n) \Rightarrow \iota ) \rightsquigarrow
        \\ ( \mathrm{Decls}' ; \Gamma' \vdash f' (t'_1, \ldots, t'_n )
        \Rightarrow ( \beta , b_1 \wedge \ldots \wedge b_n \wedge f'_b (t'_1 , \ldots , t'_n) ) ) }
\]}

\section{Higher order predicates}

Currently, \kl{CN} has `predicates' which can access and affect both the
constraint context and the resource context, and `functions' which are pure and
can do neither.

The two not only have incompatible grammars, but also different auto-unfolding
behaviour, leading to confusion for the user.

Grammar for predicates is restricted to either straight-line code, or
one-top-level if-then-else, but has good auto-unfolding behaviour. Grammar for
functions is not just restricted, but has no auto-unfolding behaviour,
requiring manual `unfold' commands.

Though they have very different semantics, unifying the grammar and unfolding
behaviour is desirable, so long as there is some way to ensue pure expressions
do not refer to resourceful ones.

Additionally, the lack of a half-way ground where we can affect the constraint
context and access the SMT solver, but not the resourceful one is confusing.
This would allow the user to write morally partial functions and specifications
more naturally, and have \kl{CN} do the heavy lifting of inferring appropriate
preconditions, similar in spirit to the aforementioned translation from
integers to bit-vectors,

One way to solve this is by introducing the distinction at the basetype-level.
Pure basetypes are $\beta$, resourceful ones could be $\cnkw{res}\,\beta$, and
the SMT-related ones could be $\cnkw{smt}\,\beta$. Note that this can be merely
an internal representation to make it easier to specify type inference and
checking. If one is apprehensive about systems programmers seeing parameterised
types (despite the pervasiveness of C++ and popularity of Rust), it does not
need to affect the surface syntax that the user sees.

Terms of type $\cnkw{res}\,\beta$ can be interpreted as map into separation
logic propositions, $\beta \rightarrow \mathsf{Prop}^\mathsf{SL}$, as mentioned
in~\cref{fig:monad-sl}. Terms of $\cnkw{smt}\,\beta$ are instead interpreted as
a \emph{pairs} $\beta \times \mathsf{Prop}^\mathsf{SMT}$. Consider the
aforementioned term grammar extended with $\cnkw{assert}(b)$. A full treatment
would involve a typed translation similar to that for the integer-to-bit-vector
translation, particularly to get the type $\beta$ for translating
$\cnkw{assert}(b)$ to $\cnkw{default}\,\beta$, but a sketch is given below.

{\small%
\begin{align*}
    \llbracket \cnkw{assert}(b) \rrbracket &= ( \cnkw{default}\,\beta, b ) \\
    \llbracket x \rrbracket &= ( x, \cnkw{true} ) \\
    \llbracket t_1 \oplus t_2 \rrbracket &= ( t'_1 \oplus t'_2 , b_1 \wedge b_2 )
    \ \text{where}\ \llbracket t_i \rrbracket = ( t'_i , b_i ),\ i \in \{\, 1, 2 \,\} \\
    \llbracket \cnkw{if}\, (t_1) \,\{\, t_2 \,\}\, \cnkw{else} \,\{\, t_3 \,\} \rrbracket &=
        (\cnkw{if}\, (t'_1) \,\{\, t'_2 \,\}\, \cnkw{else} \,\{\, t'_3 \,\}, \\
        & \qquad b_1 \wedge (\cnkw{if}\, (t'_1) \,\{\, b_2 \,\}\, \cnkw{else} \,\{\, b_3 \,\})) \\
        & \quad \text{where}\ \llbracket t_i \rrbracket = ( t'_i , b_i ),\ i \in \{\, 1, 2, 3 \,\} \\
    \llbracket \cnkw{let}\ x = t_1\ \cnkw{in}\ t_2 \rrbracket &=
    (\cnkw{let}\ x = t'_1\ \cnkw{in}\ t'_2, b_1 \wedge (\cnkw{let}\ x = t'_1\ \cnkw{in}\ b_2)) \\
        & \quad \text{where}\ \llbracket t_i \rrbracket = ( t'_i , b_i ),\ i \in \{\, 1, 2 \,\} \\
    \llbracket f (t_1, \ldots, t_n) \rrbracket &=
    (f (t'_1, \ldots, t'_n), b_1 \wedge \ldots b_n \wedge f_b (t'_1, \ldots, t'_n)) \\
        & \quad \text{where}\ \llbracket t_i \rrbracket = ( t'_i , b_i ),\ i \in \{\, 1, \ldots, n \,\} \\
\end{align*}}

\section{Predicate definition checks}

\kl{CN}'s resource inference can diverge on certain inputs. Below is a small
example to illustrate the problem, and later, a solution.

\cfile[lastline=21,highlightlines={4-21}]{code/diverge.c} % chktex 8

First the code declares the simplest possible recursive structure:
\cinline{struct node} contains one \cinline{struct node*} called
\cinline{next}.

The first predicate asserts ownership of a points-to at \cninline{front}, binds
the pointee to \cninline{F}, and then calls \cninline{List_Next_Seg} with
\cninline{front}, \cninline{F.next}, and \cninline{back}. In turn, that
predicate checks whether \cninline{front} and \cninline{back} are equal,
returning if so, and recursing on \cninline{next} and \cninline{back} on the
original predicate if not. If not for \kl{CN}'s syntax limitations
(\nameref{sec:restriction-branching}), the two definitions could be combined
into one. Though the predicate is unusual, it is derived from an example
written by a user, and is a priori a sensible
definition (for non-empty list segments).\sidenote{\href{https://github.com/rems-project/cerberus/issues/451}{Cerberus\#451}}

\cfile[firstline=23,highlightlines={25-27}]{code/diverge.c} % chktex 8

The definitions are problematic because they can be used in a context which
causes \kl{CN} to diverge. Using \cinline{List_Seg(front, back)} % chktex 36
with \cinline{RW(back)} causes the auto-unfolding mechanism to loop. % chktex 36
This is because the \cinline{List_Seg} predicate included ownership of
\cinline{front}, and so by the time \kl{CN} symbolically evaluates the
condition in \cinline{List_Next_Seg}, it deduces
\cinline{!ptr_eq(front, back)} and recurses. This introduces new % chktex 36
ownership for \cinline{front->next}, the condition is guaranteed to always be
false in a context with ownership of \cinline{back}.

The root of the problem is the combination of automatically deriving logical
facts based on the resource context, and the auto-unfolding scheme for
recursive predicates. Intuitively, the problem is that the constraints on, and
behaviour of pointers changes after they have been owned. A potential remedy
for this is a monotonic dataflow analysis.

Informally, for a block of (mutually) recursive predicates, all parameters
start off marked as safe. The predicates are evaluated symbolically, and any
variables used in the argument of \cninline{RW}/\cninline{W} predicate are then
marked as unsafe. Those unsafe variables cannot then be used inside the
condition of an \cninline{if}-expression which is guarding a (mutually)
recursive call.

In the above example, at line 4, \cninline{front} is safe, but after line 5
\cninline{front} is unsafe. The call on line 6 then propogates that information
to the parameter on lines 11. Finally, line 15 has the condition which guards
the recursive call on line 18. Because \cninline{front} is unsafe, its use in
on line 15 is rejected.

\chapter{Conclusion}%
\label{chap:conclusion}

\margintoc{}

In this thesis, I argued that building a verification tool for C, suitable
for handling low-level systems programming idioms, is two parts engineering,
and one part theory.

In~\arefpart{formalisation}, I discussed the origin of \kl{CN} as a tool
to verify the \kl{pKVM} hypervisor, and \kl{CN}'s design goal of lowering the
cost of verification and being usable by `systems programmers who know
Haskell'. I showed how these engineering constraints informed the choice of
\kl{CN}'s many theoretical foundations. I showed that those these foundations
are well-established and individually well understood, its application to a
large calculus \kl{Core} with unique features presents new challenges and
solutions, such as closely linking resource contexts to heaps by representing
them as separation logic predicates, and using explicit proof terms to encode
transformations on both. I concluded by feeding back insights from the theory
into actual engineering questions we face in developing and extending \kl{CN}.

In~\arefpart{mem-model}, I described a complex, formal, memory object
model for C which accurately captures most of the behaviour exhibited by
existing optimsing compilers. I showed that we can simplify some of the
complexity by exploiting the fact that in verification, small and
easy-to-explain changes to the code are justified if they make the typing rules
easier to implement and explain. These engineering constraints defined a
flexible perimeter for the design space, which I decomposed, guided by the
theory of the model's definitions, features, purposes, and representations in
the \kl{CN} type system. I showed that there are multiple reasonable design
points for typing rules to capture, and that doing so soundly can come down to
very subtle technical details. After summarising the point I chose for
\kl{CN-VIP}, I showed that factoring out the definition and usage of the heap
allows for significantly more modular updates and structure to the
overall proof of soundness. I then explained how I implemented these rules in
\kl{CN} and their effect on the annotation and performance overhead of \kl{CN}.
At the end, I discussed how I fed back insights from the theory to improve performance,
and outlined the development of new features such as support for lemmas and
better foundations for \kl{CN}.

Lastly, in~\arefpart{engineering}, I showed that a verification tool for
real-world C, needs some way of taming the complexity of large repositories and
non-standard or unsupported aspects of C. I showed that whilst \kl{CN}'s first
sub-goal of proving \kl{pKVM} seems to be feasible, there are several concrete
ways in which needs to become more usable. I showed that the second sub-goal of
on-going proof maintenance still needs substantial work. I then compared
\kl{CN} to other proof tools by discussing my experience on verifying a small
but informative example in all of them, which indicated where it stands with
respect to the academic state-of-the-art. I then discussed industry feedback on
\kl{CN}, which indicated where \kl{CN} stands in relation to user
expectations. Lastly, I synthesised the large gap between them by outlining
concrete expectations and how to achieve them for \kl{CN} and similar tools.

There are of course, longer-term lines of research to pursue based on \kl{CN}.
I see at least four broad directions for this. I already covered in detail the
direction of more trustworthiness in \nameref{sec:better-foundations}; I will
discuss the remaining aspects here.

\section{More expressiveness}

% Increasing expressiveness of \kl{CN} means adding support for more C features,
% and in tow, a richer separation logic.
\begin{itemize}
    \item \textbf{User-defined variadic functions.} I believe the types of
        \kl{CN} are rich enough to support this, but supporting the
        \kl{Cerberus} constructs which enable this will require more work.
    \item \textbf{Function pointers.} Supporting this in the general
        case would require a higher-order separation logic, which would
        be considerable departure from the current design and require
        thinking deeply about reasonable schemes and expectations for
        automation.
    \item \textbf{Standard library support.} Full support for the above
        two features should allow this straightforwardly, though
        there may be considerable engineering involved.
    \item \textbf{Locks and relaxed-memory concurrency.} A first-order solution
        would be to provide locks as a primitive, and that should suffice in
        many cases, but at the very least would require fractional permissions
        to make usable. Relaxed memory models will require an even richer
        separation logic; as before, the impact on automation is unknown.
    \item \textbf{Inline assembly.} As seen in the early allocator in
        \kl{pKVM}, inline assembly is used in real-world C, and supporting
        this soundly would be a rather large integration with frameworks
        such as Islaris~\sidecite{sammler2022islaris}.
    \item \textbf{Binary verification.} Mete Polat did some exploratory
        work~\sidecite{polat2023automated} on translating \kl{CN}
        specifications and proofs to binary using the
        Islaris framework, and turning this into full-fledged support
        for direct binary verification (as opposed to verifying an
        industrial-grade optimising compiler) would be a large undertaking.
\end{itemize}

\section{More performance}

\begin{itemize}
    \item \textbf{Perform a `mem2reg' pass on \kl{Core}.} Representing stack
        variables whose address is not taken as resources is unnecessary.
        Transforming memory actions on such locations into immutable \kl{Core}
        variables and expressions, akin to the \kl{LLVM} memory-to-register
        pass (mem2reg), would reduce the amount of resource inference,
        bounds-checking, disjointness reasonsing required of the \kl{SMT}
        solver.
    \item \textbf{Use ownership typing.} Whilst low-level systems C code relies
        on aliasing pointers, most code and most pointers are not aliasing. Hence,
        a more restrictive discipline handled quickly for those common
        situaitons, for example using Rust-style borrow-checking as a type
        system,\sidenote{\url{https://smallcultfollowing.com/babysteps/blog/2024/03/04/borrow-checking-without-lifetimes}}
        could allow users to benefit from fast checking by default, and opt-in to a slower,
        more general scheme as needed. For this to work, the ownership
        discipline would need to be expressible in terms of \kl{CN}'s resource
        logic.
    \item \textbf{Optimising solvers for typical queries.} It seems
        likely that the types of queries \kl{CN} will typically send a
        solver are quite amenable to automation, especially related
        to pointers and the \kl{CN-VIP} memory model. If these can be
        made substantially faster, then this would have a large
        benefit on \kl{CN} performance.
    \item \textbf{Carefully incrementalisng verification.} For large projects,
        \kl{CN} should cache proofs it already knows and only check the parts
        which have changed.
\end{itemize}

\section{More user-friendliness}

\begin{itemize}
    \item \textbf{Iterating on syntax with user studies.} Syntax matters
        a lot to most people outside of programming language research,
        and the syntax of \kl{CN} right now is elegant but verbose. It
        needs to be made simultaneously more concise and intuitive, whilst
        retaining a clear denotation. C programmers may prefer an imperative
        specification language, and this may be more feasible than one might
        expect~\sidecite{memarian2024ghost}, accommodating which would be a new
        challenge for specification research.
    \item \textbf{Experimenting with graphical representations of resources.}
        It is difficult to understand failed resource lookups in \kl{CN}.
        On top of this, although currently \kl{CN}'s ownership model is simple,
        features such as fractional permissions will add further complexity to
        this. It would be useful to experiment with graphical explanations of
        resource manipulations, either representations of a symbolic heap, or
        resource change timelines à la
        RustViz\sidenote{\url{https://fplab.github.io/rustviz-tutorial/}}~\sidecite{almeida2022rustviz}.
        Of course a key difference between C and Rust is not tying aliasing
        to syntax, and so (potential) aliases would need to be explained
        appropriately.
    \item \textbf{Pedagogy and incremental benefits for professionals.}
        The target audience of \kl{CN} is people who are more invested in
        correctness and learning than the average programmer, but will still
        have limited time in which to become productive and show benefits
        to stakeholders. Even if the preceding two ideas are executed perfectly,
        verification is different enough from programming and test-based tooling
        to require structured pedagogy. On top of this, it will always be
        competing with the other two for attention. A sensible pedagogy,
        combined with incremental benefits (such as runtime checking of
        incomplete assertions~\sidecite{banerjee2025fulminate}), will
        be required for the substantial uptake.
    \item \textbf{Suggesting specifications, proofs and repairs.} Aside from
        being difficult, verification is often tedious. There is rich history
        of using techniques from artificial intelligence to improve the
        automation of both SAT solvers and interactive proof assistants, and
        there is much excitement about the potential for using machine
        learning, particularly large language models, for improving on the
        state-of-the art~\sidecite{first2023ml4tp}.
\end{itemize}
