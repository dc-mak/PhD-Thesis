\PassOptionsToPackage{outputdir=_build}{minted}
\PassOptionsToPackage{dvipsnames,usenames}{color}
% Load the kaobook class
\documentclass[
  french,english,
  fontsize=10pt, % Base font size
  twoside=true, % Use different layouts for even and odd pages (in particular, if twoside=true, the margin column will be always on the outside)
  %open=any, % If twoside=true, uncomment this to force new chapters to start on any page, not only on right (odd) pages
  secnumdepth=2, % How deep to number headings. Defaults to 1 (sections)
  numbers=enddot,
]{kaobook/kaobook}

% Choose the language
\usepackage{babel} % Load characters and hyphenation
\usepackage[english=british]{csquotes}	% English quotes

% Load packages for testing
\usepackage{blindtext}
\usepackage{comment}
%\usepackage{showframe} % Uncomment to show boxes around the text area, margin, header and footer
%\usepackage{showlabels} % Uncomment to output the content of \label commands to the document where they are used

% Load the bibliography package
\usepackage[style=alphabetic,maxbibnames=99,useprefix=true]{kaobook/kaobiblio}
\DefineBibliographyExtras{french}{\restorecommand\mkbibnamefamily} %To have consistent citations between French and English text
\addbibresource{biblio.bib} % Bibliography file

%Load my own style
\usepackage{styles/layout}

% Load mathematical packages for theorems and related environments
\usepackage[boxed]{kaobook/kaotheorems}

% Load the package for hyperreferences
\usepackage{kaobook/kaorefs}

% Macros are after the knowledge package
%TC:ignore
\input{knowledge_declarations.tex}
%TC:endignore
\usepackage{tikz}
\usetikzlibrary{arrows,automata,positioning}
% \usepackage{styles/macros}
\usepackage{styles/fontcd}
\usepackage{mathtools}
\usepackage{mathpartir}
\usepackage{stmaryrd}
\knowledgenewrobustcmd{\astRef}{\mathrel{\cmdkl{\ast}}}
\usepackage{styles/alectryon-minted}

\includeonly{header,intro.gen,formal-intro,formal-kernel.gen,formal-statics.gen,formal-discussion.gen,mem-model-intro,mem-model-cn.gen,eng-intro,eng-usable,conclusion,defns_list,soundness_proof.gen}
%\includeonly{formal-kernel.gen}

\graphicspath{{./figures/}} % Paths where images are looked for

%TC:ignore
\input{cn_included.gen}    % auto-gen'd rules and grammars
\usepackage{ott/ottlayout} % customising layout
\input{cn_override.gen}    % intercepting premise, rule and defn blocks to ottlayout ones
\input{cn_drulenames.gen}  % auto-gen'd macros for rule names in small caps

\ottstyledefaults{%
numberpremises=yes,%
premisenamelayout=topright,%
rulelayout=nobreaks}

%%%%%%%%%%%%%%%%%%%
% Grammar helpers %
%%%%%%%%%%%%%%%%%%%

\ExplSyntaxOn%
\cs_new_eq:NN\StrIfInTF\str_if_in:nnTF
\cs_new_protected:Npn \MaybeSwallowCnProdNewline {
  \peek_remove_spaces:n
  \peek_catcode:NTF \cnprodnewline { \use_none:n } { }
}
\ExplSyntaxOff%

\NewDocumentCommand{\grammartabularWrapComment}{mm}{%
\begin{tabular}{@{}l@{}l@{\ \ }c@{\ \ }ll@{}lp{#1}}#2\end{tabular}}

\ExplSyntaxOn%
\NewDocumentEnvironment{mathparcomma}{ +b } % <-- capture body
 {
  \seq_set_split:Nnn \l_tmpa_seq { , } { #1 }
  \begin{mathpar}
  \seq_map_indexed_inline:Nn \l_tmpa_seq
   {
     \tl_set:Nn \l_tmpb_tl { ##2 }
     \tl_replace_all:Nnn \l_tmpb_tl {\ } { }
     %\tl_show:N \l_tmpb_tl
     \tl_if_eq:NnF \l_tmpb_tl {\_}
       {
         \ensuremath{ \l_tmpb_tl }%
         \int_compare:nNnTF { ##1 } < { \seq_count:N \l_tmpa_seq }
           { ,\and }
           { }
       }
   }
  \end{mathpar}
 }
\ExplSyntaxOff%

%%%%%%%%%%%%%%%%%%%%%%%%
% Grammar presentation %
%%%%%%%%%%%%%%%%%%%%%%%%

\renewcommand{\cnprodline}[6]{%
\StrIfInTF{#3}{M}{\MaybeSwallowCnProdNewline}{%     % hide meta prods
  \StrIfInTF{#5}{P}{\MaybeSwallowCnProdNewline}{%   % hide proof-related prods
    & & $#1$ & $#2$ & $#3$ & $#5$ & #6}}}           % no bind specs (#4)

\renewcommand{\cnbindspecprodline}[6]{} % hide extra bind specs

\renewcommand{\cnrulehead}[3]{$#1$ & & ${\Colon}{=}$ & \multicolumn{4}{l}{#3} }

\NewDocumentCommand{\cngrammarcompressed}{mm}{
    \begingroup%
    \renewcommand{\cngrammartabular}[1]{\begin{align*}##1\end{align*}}
    \renewcommand{\cnrulehead}[3]{##1  &\mathrel{{\Colon}{=}}}
    \renewcommand{\cnfirstprodline}[6]{\begingroup\begin{minipage}[t]{#1}\raggedright\ $##2$}
    \renewcommand{\cnprodline}[6]{\StrIfInTF{##3}{M}{}{\StrIfInTF{##5}{P}{}{\ $##1$~$##2$}}}{}
    \renewcommand{\cnprodnewline}{}
    \renewcommand{\cninterrule}{\end{minipage}\endgroup\\}
    \renewcommand{\cnafterlastrule}{\end{minipage}\endgroup\\}
    \cngrammartabular{#2}
    \endgroup%
}

%%%%%%%%%%%%%%%%%%%%%%%
% Rule premise filter %
%%%%%%%%%%%%%%%%%%%%%%%

\ExplSyntaxOn%
\NewDocumentCommand{\onlyUseNthCnPremise}{m +m}
 {
   \group_begin:

   \seq_set_split:Nnn \l_indices { , } { #1 }
   % \seq_show:N \l_indices

   % Backup original \cndrule
   \cs_set_eq:NN \orig_cndrule: \cndrule

   % Locally override \cndrule
   \RenewDocumentCommand{\cndrule}{O{}mmm}
     {
       % Split on \cnpremise
       \seq_set_split:Nnn \l_premises { \cnpremise } { ##2 }
       % First one is empty
       \seq_pop_left:NN \l_premises \l_tmpa_tl%
       % \seq_show:N \l_premises

       \seq_clear_new:N \l_filtered
       \seq_map_indexed_inline:Nn \l_premises
         {
           \seq_if_in:NnTF \l_indices { ####1 }
             {
               \seq_put_right:Nn \l_filtered { \cnpremise { ####2 } }
               % \iow_term:x { Adding~premise~####1:~\tl_to_str:n {####2} }
             }
             {
               % \iow_term:x { Skipping~premise~####1:~\tl_to_str:n {####2} }
             }
         }
       % \seq_show:N \l_filtered

       \orig_cndrule: [##1]{\seq_use:Nn \l_filtered {}}{##3}{##4}
     }

   #2%
   \group_end:
 }
\ExplSyntaxOff%

%%%%%%%%%%%%%%%%%%%%%
% Defn block filter %
%%%%%%%%%%%%%%%%%%%%%

\ExplSyntaxOn%
% argument #1 calls \cnusedrule{\cmdX{}} with different \cmdX{}, and we want it
% to only have an effect if \cmdX{} is in the list #2.
\NewDocumentCommand{\onlyUseRules}{mm}
{
  \group_begin:%
  % split #2 on comma (clist = comma list) and put each element in a sequence
  \seq_clear:N \l_tmpa_seq
  \clist_map_inline:nn {#2}
  {
    \seq_gput_right:Nn \l_tmpa_seq {##1}
  }%

  %\typeout{Sequence}%
  %\seq_show:N \l_tmpa_seq {}%

  % save the original command
  \NewCommandCopy{\origcnusedrule}{\cnusedrule}%

  % intercept \cnusedrule to check if its argument is in the sequence,
  % and only call the original if so (otherwise do nothing)
  \renewcommand{\cnusedrule}[1]{%
    \seq_if_in:NnT \l_tmpa_seq {##1}
      {%
        %\typeout{Found!}
        \origcnusedrule{##1}%
      }%
  }
  #1
  \group_end:%
}
\ExplSyntaxOff%


%%%%%%%%%%%%%%%%%%%%%%%%%%%
% Defn block presentation %
%%%%%%%%%%%%%%%%%%%%%%%%%%%

% Put judgement comment on a new line, half-BLS below the judgement form (in a box)
\renewenvironment{cndefnblock}[3][]{\framebox{\mbox{#2}} \\[0.5\baselineskip] #3 \leavevmode\\[0pt]}{\leavevmode\\[0pt]}
\renewcommand{\uninitCross}{\mathord{\text{\ballotX✗}}}
\renewcommand{\oftype}{ {:} }

\usepackage{ott/pf2}
\beforePfSpace{15pt, 10pt, 10pt, 10pt, 5pt, 2pt}
\afterPfSpace{15pt, 10pt, 10pt, 10pt, 5pt, 2pt}
\interStepSpace{15pt, 10pt, 10pt, 10pt, 5pt, 2pt}
\pflongindent%
%TC:endignore

%\makeindex[columns=3, title=Alphabetical Index, intoc] % Make LaTeX produce the files required to compile the index

\begin{document}

\include{header}

%----------------------------------------------------------------------------------------
%	MAIN BODY
%----------------------------------------------------------------------------------------

\mainmatter% Denotes the start of the main document content, resets page numbering and uses arabic numbers

\selectlanguage{english}

\chapter{Introduction}\label{chap:intro}

\emph{“Our goal as computer scientist today, is to design the legacy systems of tomorrow.”}
\vspace{-1.5em}
\begin{flushright}
  \sidecite[][Timothy G.\ Griffin]{griffin2017legacy}
\end{flushright}

\margintoc%

\section{Context}

On the 1\nth{9} of July, 2024, 8.5 million Windows computers inside banks, airlines, TV
broadcasters, supermarkets and many other businesses suffered the infamous `Blue Screen of
Death' after (ironically) a faulty security update from cybersecurity
provider CrowdStrike was released.\sidecite[4\baselineskip]{verge2024crowdstrike}.

Even though it took only 78 minutes between when the issue was identified and
rectified, the ensuing chaos and disruption lasted hours if not days, as getting
the fix on affected machines and restarting complex systems was very difficult.
Although not as important as the human cost, the total financial cost in terms
of downtime for businesses was estimated by one insurer to be between 5 and 9
billion USD.\sidecite{fitch2024crowdstrike}.

To explain what went wrong, I need to outline some key concepts. An
\intro[OS]{operating systems} (OS), such as iOS, macOS, Windows or Ubuntu, is
the ``software you don't care about which runs the software you do care
about'', knowns as applications (`apps') or programs, such as a timer, a web
browser or a spreadsheet editor. It provides services to software, in a way
that abstracts from the details of the particular kind of hardware, for
example, the ability to respond to click from a mouse, or a tap from a touch
screen, or to play a sound via headphones or speakers. At the heart of the
operating system is the \intro[OS]{kernel} (for example Linux or Darwin for
Apple systems); it is the lowest level of abstraction within the operating
system. It uses the hardware via \emph{drivers} to provide key services like
deciding what gets to run (scheduling),and how much of the hardware it gets to
use (resource management), and protecting itself and other code from each other
(isolation).

As such, kernels are critical bits of a sometimes precarious tower of
abstractions, since they need to be both \emph{fast} \textemdash{} since any
latency here compounds throughout the whole computer \textemdash{} and
\emph{secure} \textemdash{} since any vulnerability here could crash or leak
data throughout the whole system. They are \emph{by necessity}, low-level
software, and need to be written in programming languages which expose lots of
control to the programmer, known as \intro{systems programming languages}.
However, more human control means more chance for human error. Historically,
this did not pose a multi-million computer, multi-billion dollar risk: both
hardware and software were much rarer, simpler and more trusted.

Cyber-security providers like CrowdStrike are written in \kl{systems
programming languages} and run at the kernel level\sidenote{A technically
questionable design choice, likely motivated by other factors.} to scan and
protect computers at a fundamental level, but this great power comes with great
responsibility, since, as we witnessed, the risks of an error is magnified
greatly at this level.

Whilst most of the commentary, including the root cause
analysis,\sidecite{rca2024crowdstrike} emphasised the need for better testing
and better deployment strategies, both of which are eminently sensible,
\emph{what} to test, and more importantly \emph{what's missing} in the tests
are only sufficiently clear with hindsight.

\emph{%
``The new IPC Template Type defined 21 input parameter fields, but the
integration code \ldots supplied only 20 input values to match against. This
\ldots evaded multiple layers of build validation and testing, as it was not
discovered during the sensor release testing process, the Template Type (using
a test Template Instance) stress testing or the first several successful
deployments of IPC Template Instances in the field.''
}

Aside from testing, Microsoft is also reported to be discussing locking down
access to the kernel with security vendors, and generally designing it to be
more robust to rogue drivers and updates.

To me, the situation seems akin to building a wall with Swiss cheese, and when
something gets through, saying `we should have had more layers' or `layers
designed this way'. Surely one might wonder if we can consider a less porous
material?

So far, the answer has been no: methods for improving the reliability of such
software are \emph{costly}, mainly in terms of expertise, but also in term of
time and effort, with little scope for critical \emph{ongoing} maintenance. And
whilst the exact contributions of this thesis would not have prevented
CrowdStrike (even hinting at that would be temeritous), they are extremely
relevant to the general domain.

This thesis argues that the answer is now a \emph{qualified yes}. A `less
porous material' is possible with what we know now, and not too complex
conceptually (though novel in its application). The main challenges come due to
\emph{scale}. Whilst the cost of the technology is still too high for mass
deployment, the trend is downward and should continue that way with sustained
effort.


\section{Thesis statement}

In this thesis, I will argue that building a verification tool for C, suitable
for handling low-level systems programming idioms, is two parts engineering,
and one part theory. Little of the theory are novel or complex, but its
application at scale present new challenges and insights. The proportions do
not correspond to three different topics, which fit neatly in either bucket.
Rather, each conceptual part of this thesis has varying mix of those
categories, which I will explain later in \nameref{sec:contributions}.

To put those parts into context, I first need to explain the role and quirks of
the venerable C programming language, and the relevant developments in
verification theory.

\section{The C programming language}\label{sec:c-lang}

More than fifty years after its introduction, and despite competition from
C++~\sidecite{isocpp1998} and Rust~\sidecite{rust}, C remains in common use. Part
of this is simply legacy: a lot of old and useful software is written in C.
However, most of this is the success C had in meeting its initial design goals:

\begin{itemize}
    \item \textbf{Portability}. C was designed to have a relatively small set
        of defined behaviours, leaving several key choices as
        \intro[UB]{undefined behaviour}s (UBs). This allows C programs to
        exploit the advantages of several different kinds of hardware to run
        quickly.
    \item \textbf{Simplicity}. C was designed to be concise and simple to
        compile (in one pass if need be), so that it was relatively
        straightforward to write C compilers to support new hardware.
    \item \textbf{Proximity to hardware}. C was designed to be a `portable
        assembly', close to hardware, so that the programmer had precise
        control over the resources used (especially when CPUs were much slower
        and memory far more constrained).
\end{itemize}

As time went on, hardware became faster, by becoming more complex and so C's
proximity to it waned. However, it continued to be used in performacne critical
code such as the Linux \kl{kernel}. Portability, assisted by a large set of
\kl{UB}, became avenues for optimisations. Under pressure for faster code,
simplicity of compilation gave way to complex alias analysis and pointer
provenance reasoning to optimise code.

\begin{marginfigure}
    \centering
    \cfile{code/pointer_from_integer_1pg.c}
    \caption{Example pointer\_from\_integer\_1pg.c.}\label{fig:ptr-from-int-ub}
\end{marginfigure}%

\cref{fig:ptr-from-int-ub} shows a slightly contrived example (courtesy
of~\sidetextcite{memarian2019exploring}), which nevertheless illustrates the
alias assumptions at play here. It assumes that \cinline{ADDRESS_PFI_1PG} is
the guess the address of a variable local to function \cinline{f}, which is
cast to a pointer in \cinline{main}, and then passed in as an argument to
\cinline{f}. The question this raises is: even if the guess matches the runtime
address of the local variable, should the compiler be allowed to assume that
pointers passed in as arguments cannot alias local variables? From a purely
concrete point of view of pointers (they are simply numbers), the answer is no,
yet this would disable constant propagation in the printed value in this
example.

\begin{marginfigure}
    \centering
    \cfile{code/pointer_from_integer_1ie.c}
    \caption{Example pointer\_from\_integer\_1ie.c.}\label{fig:ptr-from-int}
\end{marginfigure}%

However, the other extreme, of assuming pointers are purely symbolic and always
non-aliasing is definitely incorrect in C, because of C's ability to
\emph{compute} with pointers (increment, calculate offsets, cast them to and
from integers). An example similar to the previous one (from the same source)
is shown in \cref{fig:ptr-from-int}, with the difference that instead of the
address being cast to a pointer in the calling function \emph{before} the local
variable is in scope one, it is cast to a pointer \emph{after} and its address
has been variable is \intro{exposed}: cast to an integer (but otherwise
unused). Though we can clearly see that there is no dataflow beteween
\cinline{k} and \cinline{i}, in general, the compiler cannot rule it out, so it
conservatively disables constant propagation to the call to \cinline{printf}
later.

Stakeholders have attempted to resovle such questions by the formation of and
continual updates to the \kl{ISO} standard of C,\cite{isoC1990} however, the
resolutions can be complex and subtle. This is because of the different
preferences for performance, control and language simplicity, amongst
application programmers, systems programmers and compiler writers. To add to
the confusion, as a prose English document, the standard has some irreducible
ambiguities, and does not necessarily reflect \intro{de facto} C, as is it
used in practice, so even memorising the standard would not be enough to
safeguard against unexpected language quirks.

Given this state of affairs, the \intro{Cerberus}
project~\sidecite{memarian2022cerberus} aims to provide a formally defined,
executable and \emph{empirically validated} semantics of C, both ISO and \kl{de
facto}. We shall explain Cerberus' particular advantages over other tools in
\cref{sec:cerberus-core}. For now, it suffices to say that Cerberus works by
\emph{\kl{compositional}ly} elaborating C into a first-order functional
language (known as `\intro{Core}') with a few purpose-built constructs. It
makes explicit many implicit quirks of C such as integer promotion, \kl{UB} and
loose evaluation order.

The \intro{compositional} nature of in particular allows a user to see how the
elaborated \kl{Core} relates to the C a user wrote. I will use a function which
appends linked list of integers to another as a running example.

\begin{marginfigure}
    \centering
    \cfile[breaklines]{code/append_plain.c}
    \caption{Linked integer list append in C.}\label{fig:append-c}
\end{marginfigure}%

In this (admittedly unidiomatic) example, \cinline{NULL} pointers represent
empty lists, and so the function returns \cinline{ys} if \cinline{xs} is empty,
otherwise it recurses on \cinline{xs->tail} to get the new tail
\cinline{new_tail} and sets \cinline{xs->tail} to point to that,  returning the
result as \cinline{xs}.

This is elaborated into \kl{Core} (\cref{fig:append-core}). All the
subtleties and sources of \kl{UB} in C, including signed integer over/underflow,
use-after-free, leaking, and double-free memory management errors,
\kl[OOB]{out-of-bounds} indexing in arrays, and dereferencing a \cinline{NULL}
pointer. These form the contract between the compiler and the programmer, and
violations can result in difficult to debug, and costly mistakes. From the
perspective of the compiler, they are \emph{assumptions about the program and
its execution}, and so ideally be \emph{proven} absent; they are not things one
can check by running the program (though there are tools which instrument code
in various ways can find such errors, CN included). To aid programmers in
proving such \kl{UB} absent, we must understand how we can prove things about
imperative, memory manipulating programs.

\section{Verification with Separation Logic}\label{sec:sep-logic-intro}

The key ideas around proving properties about imperative programs originate all
the way back towards the end of the 1960s, with~\citeauthor{floyd1993assigning}
and~\citeauthor{hoare1969axiomatic}. The basic setup is a triple of
$\{P\} \;C \; \{Q\}$, where $P$ is a \intro{precondition}, a predicate describing the
intial state of the program; $C$ is the program which executes; and $Q$ is
\intro{postcondition}, a predicate describing the final state of the program.
Combined with a set of inference rules to construct proofs from smaller parts
of $C$, this gave programmers a way to do pen-and-paper proofs about the
behaviour of imperative programs.

This approach works well enough, up until the programming language introduces
support for potentially aliasing pointers, at which it becomes
unfeasible.\sidenote{This and the following two paragraphs rely heavily on the
explanation of \textcite{pichon2017hlogmodc}.} In short, the issue is that
whilst assertions in the specification language might \emph{syntactically}
refer to different locations, \emph{semantically} those locations may alias,
thus breaking the rule of constancy (\cref{fig:rule-of-constancy}). This rule
is critical for \intro{modular} verification for programs because we can use it
to glue togethter two separately verified programs, and compose them (e.g.\
sequentially) so long as they refer to separate program variables.

\begin{marginfigure}
  \begin{mathpar}
      \inferrule{\vdash{} \{P\} \; C \; \{Q\}  \\ \mod{(C)} \cup{} \mathit{FV} (R)}
                {\vdash{} \{ P \wedge{} R \} \; C \; \{ Q \wedge{} R \}}
  \end{mathpar}
  \caption{The rule of constancy, where $\mathrm{FV}$ refers to the free
      variables of an assertion and $\mod{}$ is a syntactic
      over-approximation to the set of program variables a program might
      modify. It states that \kl{precondition}s which do not refer to mutated
      program variables remain true that program terminates.}\label{fig:rule-of-constancy}
\end{marginfigure}

To prevent this, we would have to use the following rule, which requires the
pre- and postconditions to mention $E_3$ and $E_4$ even though they are not
mentioned syntactically in $ [ E_1 ] \mathbin{{:}{=}} E_2$, so that the
non-aliasing condition $E_1 \noteq E_3$ can be stated, and the morally disjoint
fact $E_3 \hookrightarrow{} E_4$ is preserved into the postcondition.%
\[
    \inferrule{}{\vdash{}
        \{ \exists{} v.\ E_1 \hookrightarrow{} v \wedge{} E_1 \noteq E_3 \wedge{} E_3 \hookrightarrow{} E_4 \}
        \; [ E_1 ] \mathbin{{:}{=}} E_2 \;
        \{ E_1 \hookrightarrow{} E_2 \wedge{} E_3 \hookrightarrow{} E_4 \} }
\]

This scales poorly: composing one program with $n$ variables and up to $O(n^2)$
no-aliasing conditions, with another program of $m$ variables and up to
$O(m^2)$ leads to new assertions with up to ${O(n + m)}^2$ conditions. The
problem is bad enough that the design of the verification-oriented Euclid
programming language put in place several restrictions to prevent aliasing in
the language, unlike its main influence, Pascal, which permitted
it.\sidecite{popek1977notes}

The key breakthrough came with the arrival of separation logic by John C\@.
Reynolds and Peter O'Hearn.\sidecite{reynolds2002separation} Specifically, the
introduction of the \intro{separating conjunction}, $\astRef$, allows us to
state a version of the rule of constancy which is sound, known as a the
\intro{frame rule} (\cref{fig:frame-rule}). Not only did this enable the
practical pen-and-paper verification of programs with pointers, aliasing, and
dynamic memory management, it was very soon extended to aid in reasoning about
several types of concurrency and is now mechanised in a very general way in
proof assistants~\cite{jung2018iris, appel2011verified}, enabling complex
proofs about large and subtle systems.

\begin{marginfigure}
  \begin{mathpar}
      \inferrule{\vdash{} \{P\} \; C \; \{Q\}  \\ \mod{(C)} \cap{} \mathit{FV} (R)}
                {\vdash{} \{ P \ast{} R \} \; C \; \{ Q \ast{} R \}}
  \end{mathpar}
  \caption{The frame rule. We still need to be careful about non-intereference
      about program variables on the stack, so we retain $\mod{(C)} \cup{}
      \mathrm{FV}(R) = \emptyset{}$, but locations on the heap are ensured
      disjoint by the definition of $\astRef$. The name comes from the
      \emph{frame problem} in artificial intelligence, where using first-order
      logic to represent the world requires many axioms simply to state that
      things do not change arbitrarily.}\label{fig:frame-rule}
\end{marginfigure}

To conclude this section, I will continue the example of appending to a list,
but this time in separation logic, in a simple imperative language. It says
that expression $\mathsf{i}$ is either $\mathsf{NULL}$ and thus represents an
empty list in memory, or there exists an integer $v$ and list $l'$ and location
$\mathsf{j}$ such that $l = v {:}{:} l'$, $\mathsf{i}$ points to $v$, its
adjacent cell $\mathsf{i}+1$ points to $\mathsf{j}$ and the list predicate
holds recursively for values $\mathsf{j}$ and $l'$. In this way, it relates the
contents of linked heaps cells, laid out in a particular format, to a
mathematical list \intro{ghost} value, which exists only in the specification
of the program but not in the runtime.

\begin{marginfigure}
    \centering
    \begin{align*}
        \mathrm{list} &(\mathsf{p}, l) \mathrel{{=}^\mathrm{def}} \\
                      &\mathsf{emp} \astRef{} (\mathsf{p} = \mathsf{NULL} \wedge{} l = []) \\
                      &\vee{} \exists{} \; {head}, \; {tl}, \mathsf{p\_tail}.\\
                      &\qquad (\mathsf{p} \mapsto{} {head}) \\
                      &\qquad \astRef{} (\mathsf{p} + 1 \mapsto{} \mathsf{p\_tail}) \\
                      &\qquad \astRef{} \mathrm{list} (\mathsf{p\_tail}, {tl}) \\
                      &\qquad \astRef{} l = {head} {:}{:} {tl} \\
    \end{align*}
    \caption{Definition of a recursive list predicate in a simple separation
        logic.}\label{fig:list-pred}
\end{marginfigure}

With the predicate in \cref{fig:list-pred}, we can write a proof sketch of a
version of the list \mintinline{text}{append} program from \cref{fig:append-c},
with intermediate assertions inserted (\cref{fig:append-annot}). Because the
definition \mintinline{text}{append} is recursive, we annotate it with a pre-
and postcondition, and prove that the implementation matches it (assuming it
holds at structurally smaller values). The precondition states we start with
two disjoint heaplets representing two \kl{ghost} lists,
$\mathrm{list}(\mathsf{xs}, l_1)$ and $\mathrm{list}(\mathsf{ys}, l_2)$, and
the postcondition says that the value returned by this function
($\mathsf{ret}$) represents the concatenation ($@$) of the two logical lists
from the input.

Under the true-branch of the \mintinline{text}{if}, we have that $\mathsf{xs} =
\mathsf{NULL}$ and so it represents the empty heap, meaning the return value
and associated \kl{ghost} list is simply $\mathsf{ys}$ and $l_2$ respectively.
Under the false-branch, because $\mathsf{xs} \noteq{} \mathsf{NULL}$, we may
assume we can unroll the definition of $\mathrm{list}$, before calling
\mintinline{text}{append} recursively on the smaller
$\mathrm{list}(\mathsf{xs}', l_1')$, which allows us to conclude
$\mathrm{list}(\mathsf{new\_tail}, l_1' @ l2)$. Using the frame rule, the
$\mathsf{xs} \mapsto v \astRef (\mathsf{xs} + 1) \mapsto \mathsf{xs}'$ remains
unchanged and so we can fold these components into $\mathrm{list}(\mathsf{xs},
l_1 @ l_2)$.

\begin{marginfigure}
    \inputminted[breaklines,mathescape,fontsize=\small]{py}{code/append_annot.py}
    \caption{A separation logic proof sketch of a linked integer list
        append.}\label{fig:append-annot}
\end{marginfigure}

Whilst very useful, and certainly a huge leap forward based on what was
available before, this example also highlights the limitations of the
traditional approach. It works over an idealised imperative language, suitable
for simple proofs of algorithms and their implementations, and the link between
this and the C implementation in \cref{fig:append-c} is based on trust. This
issue persists with contemporary, mechanised frameworks for separation
logic,\sidecite{appelSF5,sammler2021refinedc,jacobs2011verifast}
which use trust-based model of C, rather than an \emph{empirically validated}
one.

\section{CN:\ C, No bugs!}\label{sec:cn-intro}

\intro{CN},\sidenote{`\kl{CN}' does not stand for anything; its name is a
historical accident, though the backronym in the section title was suggested by
Elizabeth Austell.} in its \intro{proof mode}\sidenote{There are other modes to
\kl{CN}, most notably instrumentation and test generation, which I will mostly
ignore.} is a verification tool whose aspirational goal is to lower the cost of
C verification from a Rocq~\sidecite{CDT2024} programmer who knows separation
logic to a systems programmer who knows
Haskell~\sidecite{haskell}.\sidenote{This pithy wording is courtesy of Neel
Krishnaswami.} It is designed to be used pre-existing C programs (so they are
not written or structured in a way to be verified beforehand). Before I explain
how \emph{works}, I will first explain how it is \emph{used}.

When given a C file, \kl{CN} ensures that (a) the input code is free of
undefined behaviour and (b) correctly implements any specification written in
\intro{annotations} in comments with an @ symbol \cinline{/*@..@*/}
(so that they are first and foremost, for almost all C tools, a regular C
comment). Importantly, for compositionality, performance (paralellisability),
and ease of annotation,\sidenote{Annotations follow the structure/units of the
program.} these are checked on a per function basis (\cref{sec:bidir-subtyping}).

This means that even in the absence of annotations, code is being checked for
undefined behaviour. Incrementing an unsigned integer is perfectly acceptable
(\cref{fig:un-incr}) but incrementing a signed integer triggers an error message
(\cref{fig:incr-broken}).

\begin{marginfigure}
    \centering
    \cfile[breaklines]{code/unsigned_increment.c}
    \caption{Unsigned integer increment in CN.}\label{fig:un-incr}
\end{marginfigure}%

\begin{marginfigure}
    \centering
    \cfile[breaklines,breakafter=\_]{code/increment_broken.c}
    \caption{Failing signed integer increment in CN.}\label{fig:incr-broken}
\end{marginfigure}%

The wording of the error message is inherited from \kl{Cerberus}, and shows its
origins as a formal, executable specification for \kl{ISO} and \kl{de facto}
\kl{C}. In particular, it points to the relevant section of the standard which
has been violated, and uses jargon (``exceptional condition'') to indicate that
there are values of \cinline{x} for which executing this function would result
in \kl{UB}. Because of the \intro{per function} checking, the violation is not
guaranteed, but \emph{possible}. Specifically, it is considered an error
because there exist values, which if used to call this function, would result
in \kl{UB}. Phrased differently, it is the combination of the absence
\kl{annotation}s constraining the input, \emph{with respect to} what the body
of the function does with those inputs, which is erroneous.

Admittedly, as we shall discuss in \cref{sec:error-msgs}, there is much room
for improvement to take the language of the \kl{standards committee} and
translate it into something suitable for mere mortals. Yet the source location
and the \intro{state file} it points to is helpful. If \kl{proof mode} fails
because \kl{CN} was not able to prove a constraint, it produces a
\kl{counter-example} with values assigned (\cref{fig:incr-broken-counter-ex})
internal representations of program variables (see \cref{sec:counter-ex}).

\begin{marginfigure}
    \centering
    \includegraphics[width=\textwidth]{figures/increment_broken_state.png}
    \caption{Counter example for increment\_broken.c.}\label{fig:incr-broken-counter-ex}
\end{marginfigure}

We can avoid this error by constraining the values of the input with a
precondition annotation, as in \cref{fig:incr}. Here we see the keyword
\cinline{requires} is used to introduce a pre-condition on the input.
\cinline{MAXi32()} is an in-built function which represents the maximum value a
signed 32-bit integer can represent. By constraining the input so that it is
strictly less than the maximum value, the function is now guaranteed to have
no \kl{UB} for all its inputs, no matter what the context. This is because
whilst pre-conditions are \emph{assumed} inside the function, they are
\emph{required} when calling it.

\begin{marginfigure}
    \centering
    \cfile[breaklines]{code/increment.c}
    \caption{Successful signed integer increment in CN.}\label{fig:incr}
\end{marginfigure}

\begin{marginfigure}
    \ContinuedFloat*
    \centering
    \cfile[breaklines,lastline=8]{code/call_increment.c}
    \caption{Calling a signed integer increment in CN.}\label{fig:call-incr}
\end{marginfigure}

\begin{marginfigure}
    \ContinuedFloat{}
    \centering
    \cfile[breaklines,firstline=10]{code/call_increment.c}
    \caption{Calling a signed integer increment in CN.}\label{fig:call-incr-fail}
\end{marginfigure}

This is demonstrated in \cref{fig:call-incr} and \cref{fig:call-incr-fail}. In
the first, we see that from the constraint \cinline{y <=
100i32},\sidenote{Integer literals are currently written with a type
annotation, similar to Rust.} \kl{CN} deduces that \cinline[breaklines]{y <=
MAXi32()}\sidenote{As I shall discuss in \cref{sec:bidir-subtyping,
chap:kernel-typing}, this is a \kl{subtyping} relation, and integrating it smoothly
relies on \kl{bidirectional} type-checking.} and permits the call to
\cinline{increment}. Conversely, for the second, \cinline{INT_MAX} does not
meet that constraint, and so \kl{CN} raises an error.

\begin{marginfigure}
    \centering
    \cfile[breaklines,firstline=4]{code/decrement_broken.c}
    \caption{Failing to decrement the result of a signed integer increment in
        CN.}\label{fig:decr-broken}
\end{marginfigure}

However, if we try to decrement the result of the successful call, \kl{CN}
raises an error (\cref{fig:decr-broken}). This is because that \kl{CN} has no
indication on the constraints of the return value (other than those deduced
from its C type, namely that it fits within a signed 32-bit integer). To fix
this, we need to provide a postcondition for the function which
expresses additional constraints on the returned value (\cref{fig:decr}).

\begin{marginfigure}
    \centering
    \cfile[breaklines]{code/decrement.c}
    \caption{Successfully decrementing the result of a signed integer increment
        in CN.}\label{fig:decr}
\end{marginfigure}

To specify pointer manipulating programs, I need to introduce some new syntax.
Where in separation logic, we may say $p \mapsto v$ for arbitrary expressions
$p$ and $v$, in \kl{CN}, we restrict it so that $v$ is always a variable, akin
to $\exists{} v.\ p \mapsto v \wedge v = e$ for some expression $e$. For both
usability and technical reasons explained later (\cref{sec:monadic-syntax}), we
write this as \cninline{take v = Owned(p);}. % chktex 36

\begin{marginfigure}
    \centering
    \cfile[breaklines]{code/owned_increment.c}
    \caption{Incrementing a signed integer via a pointer in CN.}\label{fig:owned-incr}
\end{marginfigure}

This generalises to work with arrays, with syntax of the form
\cninline[breaklines,breakafter=\{=()]{take arr = each (u64 i; .. ) { Owned(p) };}, % chktex 26 chktex 37 chktex 36
called \intro{quantified} or \intro{iterated} resources, for the pre- and
postconditions of the function. \kl{CN} can also handle (in a limited, careful
fashion to preserve decidability) \intro{quantified constraints}, such as the
one in the postcondition of \cref{fig:owned-array}, to express constraints
about the elements of an array. Within the \cninline{Owned}, we express the
location with \cninline{array_shift(p,i)}, which is the specification language % chktex 36
equivalent of pointer arithmetic in \cinline{&p[i]}. Within the body of the
function, we use a \intro{CN statement}, \cninline{extract Owned<int>, 0u64;},
which acts a proof hint to \kl{CN} to tell which index of the \kl{iterated}
resource we wish to read or write from.

\begin{figure*}[tp]
    \centering
    \begin{minipage}{1.5\textwidth}
        \cfile[breaklines]{code/owned_array.c}
    \end{minipage}
    \caption{Summing up a two-element array of unsigned integers in
        CN.}\label{fig:owned-array}
\end{figure*}

\kl{Iterated} resources are powerful because they supports random access
(unlike a recursive predicate over an array, which would fix the order of
traversal).\cref{fig:init-arr-rev} shows how \kl{CN} can transform an array of
uninitialised values (\cninline{Block}s) from the precondition, into an array
of initialised values (\cninline{Owned}s) in the postcondition, by looping over
it \emph{in reverse}. All loops in \kl{CN} must be annotated with \intro{loop
invariants}, which are mostly as expected with exception of the verbose and
confusing \cninline|{_} unchanged|
syntax,\sidenote{\url{https://github.com/rems-project/cerberus/issues/443}}
which tells CN that the loop does not mutate the function arguments.

Within the loop, writing to a \cninline{Block} (or an \cninline{Owned}) transforms
it into a new \cninline{Owned}. We need two \cninline{extract}s to tell CN we
wish to first take a \cninline{Block} out of \cinline{Uninit} and then move
it into the \cninline{Init}.

\begin{figure*}[tp]
    \centering
    \begin{minipage}{1.5\textwidth}
        \cfile{code/init_array_reverse.c}
    \end{minipage}
    \caption{Initialising an array of characters in reverse in
        CN.}\label{fig:init-arr-rev}
\end{figure*}

Lastly, in the same way that separation logic predicates use $\mapsto$ as a
building block to express more complex relations between heaps and \kl{ghost}
values, \kl{CN} allows users to write \intro{resource predicate} to express
more complex relations between heaps and \kl{ghost} values.

Continuing our running example of appending two linked lists of
integers, \cref{fig:append-cn} shows how one might annotate and prove such an
example in \kl{CN}. Firstly, it declares the datatype for \cninline{i32} lists,
since these are not part of \kl{CN}'s base logic. Notably, arguments to
constructors are named, not positional. After that, it declares a recursive
\cninline{[rec]} mathematical function \cninline{function} which defines what
it means to append two lists, serving the same function that $@$ did in
\cref{fig:append-annot}. The final \kl{CN} declaration is for the equivalent of
the recursive list predicate in \cref{fig:list-pred}. There are many
differences:
\begin{itemize}
    \item The \kl{CN} \cninline{predicate} looks like a function definition,
        with a signature that takes a \cninline{pointer p}, and a
        return type \cninline{(datatype seq)}, as opposed to
        $\mathrm{list}(\mathsf{p}, l)$ which has two arguments.
    \item It uses \cninline{return} statements to constrain the ghost list,
        rather than $l = \mathsf{NULL}$ or $l = {head}{:}{:}{tl}$.
    \item It uses the \cninline{take} syntax, to give names to the pointee of
        \cninline{p} (\cninline{H}) and the tail of the list \cninline{tl},
        rather than existentials $\exists{} \; {head}, \; {tl}, \mathsf{p}$.
    \item The pointee of \cninline{p} is a record \cninline{H}, with fields
        \cninline{head} and \cninline{tail} corresponding to the C struct
        fields, rather than using pointer arithmetic $\mathsf{p} + 1$.
\end{itemize}

All of these differences will be explained in more detail and justified in
\cref{sec:monadic-syntax}. I will note in passing that the syntax can also be
seen as tracing out the footprint of the heaplet containing the list, and
constructing the ghost value it represents.

For now, I will move on to the annotations on the implementation. The
precondition can be read as saying ``\emph{\cinline{IntList_append} requires
\cninline{L1} to be the ghost list constructed using \cninline{IntList}
with \cinline{xs}, and similarly for \cninline{L2} and \cinline{ys}}''. The
postcondition can be read as saying ``\emph{\cinline{IntList_append} ensures that
\cninline{L3} is a ghost list constructed using \cninline{IntList} with the
return value \cninline{return}, and \cninline{L3} is equal to \cninline{L2}
appended to \cninline{L1}}''.

Within the body of the function, in both branches of the \cinline{if}, there is
a \kl{CN statement}, this time to instruct CN to manually unfold the definition
of the recursive function at that point to aid proving the postcondition.
Conveniently, albeit
inconsistently,\sidenote{\url{https://github.com/rems-project/cerberus/issues/483}}
\kl{CN} auto-unfolds branching predicate definitions, including recursive ones,
when it is able to prove or disprove the branch conditions of that predicate,
based on the assertions in its context at a particular program point.

\begin{figure*}[tp]
    \begin{minipage}{.85\textwidth}
        \cfile[breaklines,firstline=5,lastline=24]{code/append_annot.c}
    \end{minipage}%
    \begin{minipage}{.65\textwidth}
        \cfile[breaklines,firstline=26,lastline=41]{code/append_annot.c}
    \end{minipage}
    \caption{Appending a linked list of integers in CN.}\label{fig:append-cn}
\end{figure*}%

\section{Contributions of this thesis}\label{sec:contributions}

\kl{CN} has been developed by many people; my particular contributions relate
to the following subsections, which provide an overview of the remaining parts
of this thesis.

\subsection{Formalisation of \kl{CN}}

Whilst Christopher Pulte and Thomas Sewell are the primary implementers of
\kl{CN}'s \kl{proof mode}; my contributions to this part of the project have
been to clarify and formalise \kl{CN}'s theory. The end result of this work was
not only more confidence in the theoretical merits of the approach of \kl{CN},
but also clarifying its principles, and generating insights for feeding back to
the implementation.

My contributions in this part include:
\begin{itemize}
    \item A grammar of resource terms for book-keeping resource manipulations.
    \item \intro{ResCore}: A let-normal form of \kl{Cerberus}' \kl{Core} with
        reified resources.
    \item A novel, modular heap definition in a dynamic semantics for \kl{ResCore}.
    \item \intro{Kernel CN}: A formal definition of a bidirectional,
        \kl{separation logic}, \intro{refinement type} system for \kl{ResCore}.
    \item A proof of soundness of \kl{Kernel CN} with respect to the
        \kl{ResCore} dynamic semantics.
    \item A formalisation of two inference algorithms used by \kl{CN}, one for inferring
        quantifiers and one for inferring indices for \kl{iterated} predicates.
    \item A formalisation of the link between the surface syntax of \kl{CN} and
        the formal presentation in this thesis.
    \item An experience report on (a) iterating on the design of a very large type
        system using Ott~\sidecite{sewell2010ott} and (b) feeding back insights
        from the theory to the implementation.
\end{itemize}

\subsection{Design, formalisation and implementation of \kl{CN-VIP}}

In its initial version, to simplify matters, \kl{CN} was built on top of a
simple concrete \kl{memory object model}, where pointers were treated as
interchangeable with integers. However, as I briefly alluded to in
\cref{sec:c-lang}, this is not accurate with respect to how the standard and
modern compilers treat pointers. To remedy this, building on the work of
\sidetextcite{memarian2019exploring} and \sidetextcite{lepigre2022vip}, I
designed, formalised and implemented, \intro{CN-VIP}, a memory model based on
(and sound with respect to) prior work. Because of this, \kl{CN} will now
automatically check and enforce a large number of subtle rules regarding the
creation, modification and destruction of pointers and associated allocations,
at the cost of \kl{CN} becoming harder and slower to use.


My contributions in this part include:
\begin{itemize}
    \item An exploration of the design space and trade-offs when designing a
        type system for real-world memory object models.
    \item The formalisation of \kl{CN-VIP}, focusing on its modular integration into
        \kl{Kernel CN} and \kl{ResCore}.
    \item A proof of soundness of \kl{CN-VIP}, with respect to the VIP dynamic semantics.
    \item A walk-through of the implementation of \kl{CN-VIP}, along with the
        approach required to integrate it without causing pain to myself, other
        developers or users.
    \item A discussion on the problems with supporting byte-level provenance
        and round-trip casts in \kl{CN}, potential resolutions and trade-offs.
    \item A roadmap on the integration of mechanised, sound resourceful lemmas
        into \kl{CN} inspired by the design of \kl{CN-VIP}.
\end{itemize}

\subsection{Will the real world C, please stand up?}

Though it is developed inside academia, \kl{CN} is fundamentally a piece of
software, and its job is not to advance cool theory for its own sake, but to
bring a new level of assurance to \emph{existing, real-world} \kl{systems} C
code. This requires substantial engineering effort, and benefits immensely from
modern software development best practices. Though a lot of it seems obvious in
retrospect, it was not initially clear \emph{if} and \emph{when} many of
engineering challenges would become a bottleneck. I worked on three major
problems, was successful in two of them, and collaborated closely with partners
in industry looking to use \kl{CN}. As such, \kl{CN} is better poised to meet
the challenges facing any verification tool aiming to be useful outside of
academia.

My contributions in this part include:
\begin{itemize}
    \item \intro[tree-carver]{A tree carver for C}. This is Clang based tool
        which, given a C file in a large source tree, and a root file or
        functions, carves out that file and its transitive dependencies,
        including macros, for processing by other tools.
    \item \emph{An experience report}, on updating an existing proof of a \intro{buddy
        allocator} (used in the \intro{pKVM} hypervisor) to work with (a)
        bit-vectors and (b) an early version of CN-VIP\@. In particular, due to
        poor source location information, poor proof migration path and
        performance degradation, this port failed.
    \item A summary of several developer- and user-experience issues we faced,
        with reference to the their technical origin and if they exist,
        solutions.
\end{itemize}


\pagelayout{wide} % No margins
\addpart{Formalisation}%
\label{part:formalisation}
\pagelayout{margin} % Restore margins

In \cref{sec:c-lang}, I discussed how competing forces of inherited
portability requirements, proximity to hardware, and the desire for more
aggressive optimisations led to complex and subtle technical resolution by
stakeholders in the \kl{ISO} standard of C. I also mentioned that its nature as
a prose document, with natural language ambiguities and omissions, as well as
divergence from C as used \kl{de facto}, mean that its semantics are unreasonable
for a human to adhere to, and challenging to build into tools directly,
without making some sort of simplifying assumptions.

Existing program logic frameworks for C such as Verifiable C~\sidecite{appelSF5}
and RefinedC~\sidecite{sammler2021refinedc} take the approach of building a
logic directly above an operational semantics for a language which is
recognisably C, minus some desugaring to consolidate similar constructs. They
attempt to retain as many C features (control flow, variable scoping, aliasing,
loose evaluation order, pointer manipulation rules) as possible, but make
simplifying assumptions where it would be impractical otherwise.

Given that \kl{CN}'s headline goal (\cref{sec:cn-intro}) is to work with
pre-existing C programs, which rely on many if not all of those impractical
features, adopting the conventional approach would quickly use up most of its
complexity budget and make the other goal (of reducing the expertise required
to do verification) unfeasible.

Instead, \kl{CN} builds directly upon the
\kl{Cerberus}~\sidecite{memarian2022cerberus} executable and empirically
validated semantics for C. Not only does \kl{CN} benefit from the
\emph{accurate} semantics for both \kl{ISO} and \kl{de facto} C, it benefits
most from the \emph{usability} of it. This is because, Cerberus is elaborated
into a relatively small calculus \emph{\kl{Core}}, which translates all of C's
complexity into a first-order functional language with a few special (but easy
to understand and specify) constructs.

Additionally, \kl{CN} is intended to be used more like a \emph{type system} in
an IDE than a program logic inside a proof assistant. Ideally, instead of
seeing intermediate goals in a sophisticated separation logic, and needing to
be well versed with a range of inference rules and automation tactics, a user
sees their C program, scattered with predictable and lightweight annotations in
comments, in an editor which either indicates success, or clear and helpful
error message.

Aside from the fact that the notion and mode of use of a type system is more
familiar to most programmers (an advantage not to be scoffed at), this approach
also allows \kl{CN} to use and advance the extant literature on building
refinement type systems on top of existing languages.

This type system approach also leads to other desiderata and their
corresponding responses. If we want to follow a type system approach, we want
to minimise obvious annotations and justify why the necessary ones are so, we
need to track carefully the flow of information in the type system, using a
\kl{bidirectional} approach. We also need some sort of automation so as to not
burden the programmer with proving things like $1 + 1 = 2$. Similar to
VeriFast~\sidecite{jacobs2011verifast} and Frama-C~\sidecite{kirchner2015frama},
\kl{CN} enlists the support of SMT solvers to mitigate this. When trying to
verify code against expressive specifications, this could lead to
non-termination, so \kl{CN} also restricts the expressiveness of the assertion
language, and the queries it sends to the SMT solver. And given the importance
of managing resources in C, the typing discipline needs to be substructural.

The \kl{CN} assertion language syntax aims to be expressive enough to verify
real world C, but also restricted enough to limit the aforementioned technical
problems, and intuitive enough to a target audience of systems programmers who
happen to know Haskell (or Rust).

With this many constraints and design decisions, it is easy to be doubtful of the
elegance and feasibility of this approach, let alone consider proving such a
type system sound. As I will show in \nameref{chap:kernel-statics}, whilst the setup
might be novel, multi-faceted and large, the definitions are relatively
straightforward, and the proof of soundness can be done syntactically. Both the
definitions and the proof are modular with respect to the heap, so that
changing the memory object model does not require redoing the entire soundness
proof. The formalisation is close enough to the surface syntax of \kl{CN} so
that a correspondence between the two can be stated simply and precisely, and
close enough to the implementation to offer actionable insights.

\chapter{Formalisation Background}%
\label{chap:formal-background}

\margintoc{}

The components of \intro{Kernel CN} all have precedent in prior work; the main
new contribution is the adaptation and confluence of those ideas. This chapter
will set out \kl{CN}'s design goals and origins, recapitulate the disparate
concepts used in CN, and along the way discuss how they satisfy the
aforementioned design goals.

\section{\kl{CN} design goals and constraints}%
\label{sec:cn-goals}

Aiming for \emph{``a verification tool whose aspirational goal is to lower the
cost of C verification from a Rocq programmer who knows separation logic to a
systems programmer who knows Haskell''} (\cref{sec:cn-intro}) helps narrow
down the large design space of verification tools.

The reason for picking this particular goal is in \kl{CN}'s origins as
an attempt to verify the pKVM hypervisor, developed by Google.

To explain pKVM, I need to situate it in its context: the Android
operating system, which runs on billions of devices worldwide, plays a central role
in many lives, including handling an enormous amount of sensitive data. This
means that security is paramount, however because each device runs its own
kernel (up to half the code is not Android's version of Linux), updates are
very challenging and expensive to test and deploy to each device. Aside from
security issues, this also leads to fragmentation of Android, so devices and
apps are not all up-to-date and the long delay (at least 18 months) between
Linux and device releases makes it difficult to upstream features and fixes.%
\sidenote{%
TODO\@: cite these properly.
\begin{itemize}
    \item \url{https://youtu.be/7novnkldMmQ?feature=shared}
    \item \url{https://youtu.be/wY-u6n75iXc?feature=shared} and \url{https://lwn.net/Articles/836693/}
    \item \url{https://source.android.com/docs/core/architecture/kernel/generic-kernel-image}
    \item \url{https://source.android.com/docs/core/virtualization/whyavf}
    \item \url{https://googleprojectzero.blogspot.com/2020/02/mitigations-are-attack-surface-too.html}
    \item \url{https://lpc.events/event/7/contributions/780/}
\end{itemize}
}

Whilst some of this has been mitigated with the introduction of
\intro[GKI]{Generic Kernel Images (GKI)}, which provide a small and stable
kernel ABI \emph{for a particular long-term release} version of Android, there
are still security issues present in this model, because the kernel is too
large (20 million lines of code) to be a reasonable trusted computing base, and
the drivers vendors ship with a device are part of it.

Some manufacturers use hypervisors, which attempt to isolate the kernel from
the rest of the system by running Android and other hardware components in
virtual machines, such `secure' parts of the device storing sensitive data.
Aside from security, hypervisors are also used to partition memory at boot-time
so that devices can use it for things like direct memory access, and run
arbitrary code outside of Android, which is worrying because this code would
run at a more privileged level than Android itself. All of this just
\emph{shifts} the attack surface, and has also resulted in \emph{more}
fragmentation at the hypervisor layer.

Similar to GKI, the proposed solution to standardise the hypervisor used. There
is already a mature hypervisor which is part of the Linux kernel, the
Kernel-based Virtual Machine (KVM)\@. It is set up so that a host kernel can
dynamically allocate virtual machines for guests to run on, and protect the
host from the guests. However, at the start of the project, the API exposed by
the hypervisor to the host kernel offered too much control, and guests were not
protected from the \emph{host}. This is a problem because the guest could be
running code for a secure hardware component (e.g.\ a biometric authenticator),
the host could be a compromised version of Android, so an attacker could still
get access to sensitive information.

To solve this, Google, as part of the Android Virtualisation Framework, is
developing a \intro[pKVM]{protected KVM (pKVM)}, which runs \emph{underneath}
the kernel, and ships \emph{as part} of the kernel image. Not only does this
tight coupling remove issues around ABI compatibility between the hypervisor
and the kernel, since the source is always in the same repository, it also
allows pKVM to commit to only handling implementing a select few functions such
as virtual memory management and remain very small, and rely on the Linux
kernel to manage the rest, such as scheduling, device drivers and power
management.

If successful, this could make the attack surface a lot smaller, but it could
also make it one that is used very widely. It is in this context that Google
sought assistance from the research community to see if verifying the kernel
was feasible, \emph{on an ongoing basis}. A one-and-done verification of pKVM
which takes an army of PhD students and postdocs a few years to verify and is
years out of date by the time is developed is not worth the investment; a
tool which C kernel programmers can understand, use and maintain proofs as they
make changes to very important security critical code is.

So not only does this background explain \kl{CN}'s headline goal, it also
clarifies some of the \emph{constraints} on its design:
\begin{itemize}
    \item Because kernel programmers wrote the code, and are intending to use
        conventional compilers to build and run it,\sidenote{Assuming the
        binary can be verified as well, perhaps with input from \kl{CN}.} we
        cannot rely on (hopefully sound) approximation to the semantics of C
        \textemdash{} we want and need something that matches and can handle
        its real world behaviour as closely as possible.
    \item Because it will be used by kernel programmers, we want a story that is
        is accessible and acceptable to them. These are very smart and
        capable people, who do not have the time or support to get up to
        speed with separation logics and interactive proof assistants, or be
        amenable to change their (or more importantly, their organisations)
        workflows substantially. A ``fancier type system'' which runs as part of
        the \intro[CI]{continuous integration (CI)} pipeline is much more
        likely to be used and adopted in this context.
    \item Similarly, because the annotations will be read by kernel programmers,
        and upstreamed into Linux, we want them to be minimal and relatively
        easy to understand. Not only does this affect the design of the type
        system to manage the flow of information carefully, this encourages
        exploring how best to automate as many obvious things as feasible.
\end{itemize}

In turn, these constraints feed into concrete technical choices which \kl{CN}
makes:
\begin{itemize}
    \item To capture real-word C behaviour, \kl{CN} uses the \kl{Cerberus}
        empirically validated semantics.
    \item To integrate into existing workflows, \kl{CN} appears to users
        as a fancy type system.
    \item To retrofit the type system on top of existing ones, and in
	particular to ensure erasure, \kl{CN}'s type system uses
	\intro{refinement types}.
    \item To automate proofs of constraints, \kl{CN} relies on SMT solvers.
    \item To ensure decidability (termination), and aim for reasonable in
        practice performance, \kl{CN} follows a \emph{liquid} typing
	discipline, restricting the queries it sends to the SMT solver.
    \item To minimise constraint annotations, \kl{CN}'s type system is
        formalised and implemented in a \kl{bidirectional} style.
    \item To check the resource management of C programs, \kl{CN}'s type
        system uses \intro{linear} types, using the grammar of separation
        logic assertions.
    \item To stay within the \kl{liquid} typing discipline \emph{and} avoid
	quantifier instantiation annotations, \kl{CN} uses precise assertions
	with a \kl{mode}d discipline for predicate arguments.
    \item To make such precise assertions with a mode discipline easier to read,
	write and explain, \kl{CN} presents users with \emph{monadic} syntax.
\end{itemize}

\section{Cerberus and Core for a usable and accurate C semantics}%
\label{sec:cerberus-core}

\intro{Cerberus} is an empirically validated and executable semantics for
\kl{ISO} and \kl{de facto} C, specifically C11. A detailed comparison
between it and other C semantics is available in the Related Work chapter
of \sidetextcite{memarian2022cerberus}, but for the purposes of \kl{CN},
it suffices to say it captures real-world C.

Where \kl{Cerberus} really shines with respect to \kl{CN}'s use case in
\emph{how} it captures this executable semantics. In particular, it does so by
\emph{compositional elaboration} into a \emph{relatively simple first-order
functional language}, unlike other semantics, which are defined over some
desugared and consolidated grammar closely resembling C.

This approach drastically simplifies many tricky parts of C. For example,
\cref{fig:perplexing-ub} is accepted by the frontend, complete with strange
scoping and strange control flow which jumps \emph{into} a loop body. This
program is illegal because of the subtle rules regarding block scopes, object
lifetimes and initialisation (as explained in the caption). Whilst contrived,
any C function which happens to have exception handling and retry behaviour,
and be longer than a programmer's screen height, could make the same error.
Cerberus handles C features such as mutable and potentially aliasing variables,
and the myriad other \kl[UB]{undefined} or \kl{unspecified} behaviours, for
example loose evaluation order and coercions respectively, by making them all
\emph{explicit} in the syntax.

\begin{marginfigure}
    \cfile[breaklines]{code/perplexing-ub.c}
    \caption{Example of subtle scoping and control-flow issues leading to UB
        (courtesy~\cite[p70]{memarian2022cerberus}).
        Jumping to \cinline{l2} starts the lifetime of an object, but does not
        initialise it. Jumping to \cinline{l1} then \emph{ends that lifetime}
        because it exits the block (lines 5\textendash{}12). By the time
        execution reaches line 8, \cinline{p} \emph{refers to a dead object}.
        This is why the seemingly redundant re-assignment of \cinline{p = &x}
        on line 7 would prevent UB in this example.}\label{fig:perplexing-ub}
\end{marginfigure}

\section{Core grammar}\label{sec:core-grammar}

Instead of working directly over something similar to C and trying to express
static and dynamic semantics over something so complex, \kl{Cerberus}
elaborates C into \intro{Core}, a not-too-large calculus where each construct
is designed to capture some peculiarity of C. It is split into two fragments, a
pure (\cref{fig:pure-core-grammar}) and effectful
(\cref{fig:effectful-core-grammar}) which embeds the pure one.

The pure fragment is pure in the sense that it allows no memory operations, but does
include the effect of undefined behaviour explicitly with the
\coreinline{undef()} operator. This aspect of the language handles % chktex 36
things like implicit type conversions or bounded arithmetic. As visible from the
grammar, the pure part is very much a first-order functional language with
recursive functions with some constructs specific to C such as pointer
arithmetic on arrays and struct fields, structs, unions and
specified/unspecified/integer values.

\begin{figure*}[tp]
    \ContinuedFloat*
    \raggedright%
    \small%
    \cngrammarcompressed{\textwidth}{%
        \cnpce{}\cninterrule{}
        \cnname{}\cninterrule{}
        \cncpat{}\cninterrule{}
        \cnctor{}\cnafterlastrule{}
    }
    \caption{The pure fragment of Core.}\label{fig:pure-core-grammar}
\end{figure*}

The effectful fragment captures interactions with memory (via a memory
interface), various ordering constraints, and more exotic control flow with a
goto-like operator used in the elaboration of C's iteration and \cinline{goto}
statements. The distinction between the pure and effectful fragments is in
fact unrelated to the distinction between expressions and statements in C,
since both are elaborated into effectful expressions (for example,
\coreinline{PtrEq} which tests for for pointer equality).

I will discuss \coreinline{memop()} in more detail in % chktex 36
\nameref{chap:mem-model-explained}. For now it suffices to say that the
operations are effects, part of the memory interface \kl{Core} uses to abstract
over choices of different handlers, implementations of those effects in a
specific memory object model.

The following constructs are all related to evaluation order:
\coreinline{neg()}, \coreinline{unseq()}, \coreinline{let weak}, % chktex 36
\coreinline{let strong}, \coreinline{bound()}. These were supported % chktex 36
in the implementation at one point,\sidenote{See note~\ref{sn:new-inf}.} but
stopped working due to a change in the resource inference scheme, and not
enough of a priority to re-enable. I did not attempt to formalise their complex
operational behaviour; I considered supporting these constructs using
fractional permissions, but decided to leave that for future work because of
the noise and complexity it would add to the initial version of the
formalisation. The \coreinline{par()} construct is for spawning % chktex 36
threads. The \coreinline{nd()} construct models non-determinism % chktex 36
as required or implied by aspects of the standard.

The \coreinline{ccall()} and \coreinline{pcall()} constructs for % chktex 36
calling elaborate C functions and Core procedures (effectful functions)
respectively. They differ only in how the name of the procedure to be called is
found, with \coreinline{ccall()} using the memory interface to do so. % chktex 36

The (thread-local) Core operational semantics is defined as a small-step
transition from a configuration to either undefined behaviour or a new
configuration. Formally, it looks like this $\langle h, E, \kappa \rangle
\rightarrow (\textsc{Undef} \mid{} \langle h', E', \kappa' \rangle)$, for a
heap $h$, effectful expressions $E$ and stack $\kappa$ (list of evaluation
contexts $C$). These steps rely on evaluating pure expressions (in a big-step
style) to either undefined behaviour or a value $\cnnt{pce} \Downarrow
(\textsc{Undef} \mid{} \cnnt{value})$.

For example, procedure calls pop the enclosing evaluation context $C$ on
to the stack $\kappa$, evaluate the arguments into values, lookup the function
body $E$ and substitute the parameters for the values, and proceed with
executing the body.

{\small%
\[
\inferrule[{[PCall]}]
 { \mathrm{funmap} ( \cnnt{name} ) = \left(\cncomp{x_i {:} \beta_i}{i}\right).\ E \\
   \cncomp{\cnnt{pce}_i \Downarrow \cnnt{value}_i}{i} }
 { \left< h , C\left[ \cnkw{pcall} \, \left( \cnnt{name}, \cncomp{\cnnt{pce}_i}{i} \right) \right] , \kappa \right>
   \rightarrow \left< h , \left[ \cncomp{\cnnt{value}_i / x_i}{i} \right] E , C \Colon \kappa \right> }
\]}

Conversely, returns pop an evaluation context off of the stack, applies it to
the values, and proceeds with the resulting expression.

{\small%
\[
\inferrule[{[Return]}]
  { }
  { \langle \cnnt{h} , \cnkw{pure}(v) , C \Colon \kappa \rangle
    \rightarrow \langle h, C [\cnkw{pure}(v)] , \kappa \rangle }
\]}

The \coreinline{save()} operator is \kl{Core}'s way of introducing named % chktex 36
continuations with default arguments. The label $l$ and arguments $x_1, \ldots,
x_n$ are in scope in $E$; those variables are associated pure expressions
provided by the \coreinline{run()} operator or with the default $e_1, \ldots, % chktex 36
e_n$, if the operator is reached otherwise. This is the \cinline{goto}-like
operator referred to earlier, which is used to elaborate all of C's iteration
and \cinline{goto} statements.

Formally, reaching a label via \coreinline{save()} evaluates its % chktex 36
default expressions into values and then substitutes those values into its
expression.\sidenote{Since this does not manipulate the heap or the stack, it
    is phrased purely as is expression reduction, with a separated, omitted
    rule to lift expression reductions to thread-local reductions.}

{\small%
\[
\inferrule[{[Save]}]
  { \cncomp{\cnnt{e}_i \Downarrow{} \cnnt{value}_i}{i} }
  { \cnkw{save} \, \cnnt{id} \left( \cncomp{x_i \cnkw{:=} e_i}{i} \right) \, \cnkw{in} \, \cnnt{E}
    \rightsquigarrow \left[ \cncomp{ \cnnt{value}_i / x_i }{i} \right] E }
\]}

Reaching the same label via a \coreinline{run()} however, is more % chktex 36
complicated. Like a procedure call, it evaluates its arguments into values and
looks up the label $\cnnt{id}$. Unlike a procedure call, the result of the
lookup is \emph{a context} $C_{\cnnt{id}}$ and an expression $E_{\cnnt{id}}$.
It substitutes the values for the parameters in $E_\cnnt{id}$, \emph{discards
the current context} $C$, and uses $C_\cnnt{id}$ instead. All the while, the
stack $\kappa$ is unchanged.

{\small%
\[
\inferrule[{[Run]}]
  { \mathrm{labelmap} ( \cnnt{id} ) = \left(\cncomp{x_i}{i}\right).\ C_{\cnnt{id}} [ E_{\cnnt{id}} ] \\
    \cncomp{e_i \Downarrow \cnnt{value}_i}{i} }
  { \langle h , C\left[ \cnkw{run}\, \cnnt{id} \left( \cncomp{e_i}{i} \right) \right] , \kappa \rangle
    \rightarrow \left< h , C_{\cnnt{id}} \left[ \left[ \cncomp{\cnnt{value}_i / x_i}{i} \right] E_{\cnnt{id}} \right] , \kappa \right> }
\]}

The important aspect for our purposes is that while control flows into
\coreinline{ccall()}, \coreinline{pcall()} and \coreinline{run()}, it % chktex 36
only returns from the first two and not the last.

\begin{figure*}[tp]
    \ContinuedFloat{}
    \raggedright%
    \small%
    \cngrammarcompressed{\textwidth}{%
        \cncore{}\cninterrule{}
        \cncoreXXctxt{}\cninterrule{}
        \cncmemop{}\cninterrule{}
        \cncaction{}\cnafterlastrule{}
    }
    \caption{The effectful fragment of Core.}\label{fig:effectful-core-grammar}
\end{figure*}

\section{Elaboration example: list append}

We can see an example of Cerberus' elaboration into \kl{Core}, by recalling the
linked integer list append function from \cref{fig:append-c}.

The first thing it does it declare its parameters. In C, these are mutable
local variables.

\cfile[firstline=5, lastline=6]{code/append_plain.c}

In Core, the pointer value passed into the function, and the mutable storage
for it are separated syntactically. The parameters have type
\coreinline{loaded pointer}, where \coreinline{loaded t} is either a
\coreinline{Specified(v:t)} or an \coreinline{Unspecified} value % chktex 36
(an option monad for potentially uninitialised values read from memory). The
result of the function has type \coreinline{eff loaded t} where
\coreinline{eff t} is a value of type \coreinline{t} produced with effects
(also a monad). The \coreinline{create()} and \coreinline{store()} operations are % chktex 36
memory actions, and produce values of type \coreinline{eff pointer} and
\coreinline{eff unit} respectively. Binding constructs \coreinline{let strong}
and \coreinline{let weak} (monadically) bind effectful values, and sequence
their effects.\sidenote{$\cnkw{let}\ \cnkw{strong}\ \cnnt{pat}\ \cnkw{=}\
    \cnnt{E}_1\ \cnkw{in}\ \cnnt{E}_2$ sequences all memory
    actions performed by $\cnnt{E}_1$ before those performed by $\cnnt{E}_2$.
    $\cnkw{let}\ \cnkw{weak}$ only sequences positive memory actions similarly;
    in other words, for $\cnkw{let}\ \cnkw{weak}\ \cnnt{pat}\ \cnkw{=}\
    \cnnt{E}_1\ \cnkw{in}\ \cnnt{E}_2$, any negative memory actions
    ($\cnkw{neg}(\cnnt{action})$) performed by $\cnnt{E}_1$ are
    unsequenced with respect to all memory actions performed by $\cnnt{E}_2$.
    A race only occurs if unsequenced memory actions have overlapping
   footprints, and is deemed \kl{UB}.}
In this example, weak sequencing has no observable effect, and so is
safe to read as \coreinline{let strong}.

\corefile[firstline=3, lastline=10]{code/append_plain.core}


In C, the test to check if \cinline{x} is \cinline{NULL} looks like this.

\cfile[firstline=8, lastline=8]{code/append_plain.c}

First this requires loading the \coreinline{xs} from memory, and then doing a
\coreinline{PtrEq} memory operation to check if it is \coreinline{NULL}.
Because booleans are represented by 0 and 1 in C, the result of this expression
is a \coreinline{loaded integer}.

\corefile[firstline=15, lastline=25]{code/append_plain.core}

That result is bound to \coreinline{a_432} and then converted (omitted) into a
Core boolean, bound to \coreinline{a_431}, which is in turn used in a Core
if-expression. Recall the true-branch of the C code.

\cfile[firstline=9, lastline=9]{code/append_plain.c}

This elaborates into the following true-branch in Core. First, the value
\coreinline{ys} is loaded from memory. Core marks the boundary of a \emph{full
expression} (§6.8p4), such as the optional expression of a return statement,
using the \coreinline{bound()} operator. After that is one % chktex 36
\coreinline{kill()} per live local variable. Lastly there is a % chktex 36
\coreinline{run()} to the return label \coreinline{ret_430}. % chktex 36

\corefile[firstline=48, lastline=56]{code/append_plain.core}

The \coreinline{ret_430} label is defined with a \coreinline{save}, at the end
of the procedure. Like C labels, Core labels can be reached by fall-through
execution as well as being jumped to. In the fall-through case, the arguments
to the label still need to be given a value; these default values are defined
by the right-hand-side of the \coreinline{:=}, here an \coreinline{undef()} % chktex 36
value because it is UB to reach the end of a function with a non-void return
type without returning a value.

\corefile[firstline=143, lastline=144]{code/append_plain.core}

In the C code, the else-branch defines a new local variable \cinline{new_tail},
and assigns it to the result of a recursive call.

\cfile[firstline=11, lastline=11]{code/append_plain.c}

In Core, these two lines are expanded considerably, so I shall omit some
details and focus only on the key parts. First is the allocation of memory for
the local variable.

\corefile[firstline=59, lastline=60]{code/append_plain.core}

Next is getting the pointer for the C function to be called. This expression is
bound to \coreinline{a_447}.

\corefile[firstline=59, lastline=60]{code/append_plain.core}

The first argument is \cinline{xs->tail}; in Core that is represented by a
\coreinline{load()} of \cinline{&xs}, a case-split on the loaded value % chktex 36
(UB if it is \coreinline{Unspecified}), a \coreinline{member_shift()}, % chktex 36
and then another \coreinline{load()}. The whole expression is bound % chktex 36
to \coreinline{a_454}.

\corefile[firstline=71, lastline=81]{code/append_plain.core}

The second argument is \cinline{ys}; in Core that is just a \coreinline{load()} % chktex 36
of \cinline{&ys}. The whole expression is bound to \coreinline{a_461}.

\corefile[firstline=83, lastline=84]{code/append_plain.core}

Finally, the elaboration of the function call itself performs a large amount of
run-time type checking; in the general case this supports calling variadic
functions via function pointers, but for this example, because the function is
(a) known and compile time and (b) not variadic, it simplifies to the below.

\begin{corecode}
ccall("struct int_list* (*) (struct int_list*, struct int_list*)",
      a_447 (* &IntList_append *),
      a_454 (* xs->tail *),
      a_461 (* ys *))
\end{corecode}

The result of this is in turn bound to \coreinline{a_445}, which is then
stored in at \cinline{&new_tail} (in Core, just \coreinline{new_tail}).

\corefile[firstline=105, lastline=107]{code/append_plain.core}

Penultimately, the \cinline{IntList_append} function needs to update the tail
of the first list.

\cfile[firstline=12, lastline=12]{code/append_plain.c}

In Core, \cinline{xs->tail} is elaborated into a \coreinline{load()} % chktex 36
and a member shift, whose result is bound to \coreinline{a_472}.

\corefile[firstline=110, lastline=118]{code/append_plain.core}

The right-hand-side is just a \coreinline{load()} from \cinline{&new_tail} % chktex 36
(which in Core, is just \coreinline{new_tail}), whose result is bound to
\coreinline{a_479}.

\corefile[firstline=122, lastline=123]{code/append_plain.core}

In C, assignments can be used as expressions. In the jargon of the standard,
any loads used to evaluate the operands form a \emph{value computation}, but
the stores do not: they are classed as \emph{side effects}.\sidenote{%
§5.1.2.3\#2: [\ldots] Evaluation of an expression in general includes both
value computations and initiation of side effects.} As
per~\textcite[p61, p66, p99]{memarian2022cerberus}, \emph{``intuitively, a Core action
is negative when it elaborates what the C standard calls a `side
effect'\,''}.
% p61 and p66 have more about value computations and side effects

\corefile[firstline=125, lastline=129]{code/append_plain.core}

Finally, there's the return statement at the end of the function.

\cfile[firstline=13, lastline=13]{code/append_plain.c}

In Core, this is elaborated into the following. First, a
\coreinline{load()}, enclosed in a \coreinline{bound()} operator % chktex 36
to mark the boundary of the \emph{full expression} (§6.8p4) inside the return
statement. After that is one \coreinline{kill()} per live  % chktex 36
local variable. Lastly there is a \coreinline{run()} to the % chktex 36
return label \coreinline{ret_430} mentioned earlier.

\corefile[firstline=132, lastline=140]{code/append_plain.core}

\section{Discussion}

A few things are note-worthy about the Core elaboration:
\begin{itemize}
    \item The translation is \intro{compositional}. Each function, block,
        statement and expression is elaborated in isolation, based only
        on its parts, and follows the structure of the original C program.
    \item Each variable (function argument and local and global) is uniformly
        handled as living on the heap and gets its own storage via the
        \coreinline{create} function. Reads, writes, and de-allocations are
        represented with \coreinline{load}, \coreinline{store},
        \coreinline{kill} respectively.
    \item Loose evaluation order (for example between the expressions of a \cinline{==})
        is represented using \coreinline{unseq} and \coreinline{let weak} constructs.
    \item \kl{UB} is made explicit in the syntax of the program.
\end{itemize}

In particular, \kl{compositional}ity is just as important, if not more, than the
target language being a first-order functional language with effects. Given
that we want users to annotate programs at the C level, if we wish to type
check \kl{Core}, we need to be able to transport those annotations through the
elaboration process too, and place them at the appropriate program points.

If \kl{Cerberus} were to have elaborated into a dataflow graph instead of \kl{Core},
such transporting would be a major undertaking in itself. It may be achievable
for function pre- and postconditions, but would become much more challenging
for loops, \coreinline{goto}, and inter-statement proof
hints to \kl{CN}. With a \kl{compositional} mapping, placing annotations
structurally, and relating annotations mentioning C variables to \kl{Core}
variables becomes feasible.

In principle, the compositional mapping also ensures that errors in \kl{Core}
elaboration can be related back to useful source locations in the original C
program. However, in practice, though \emph{compositional}ity does
\emph{enable} this, it requires a good amount of engineering effort to
accomplish (\cref{sec:error-msgs}). Another challenge is that though
elaboration simplifies greatly the checked language, it also increases the
distance between the checked and the typed language, which is felt acutely when
attempting to relate failures in SMT queries back to what users wrote,
particularly when those SMT are part of an inference procedure, rather
than checking C source assertions (\cref{sec:counter-ex}).

\section{Decidable refinements for retrofitting and counter-examples}

\kl{Core} already has a straightforward \kl{bidirectional} type system as a
sanity check on the output of elaboration. It has not been used for a
rudimentary proof of soundness with respect to the dynamic semantics, since
such a theorem would be just another sanity check (on the dynamic semantics).

\begin{marginfigure}
    \centering
    \begin{mathpar}
        \inferrule{ }{\Gamma{} \vdash{} \cnkw{undef} (\cnnt{UB\_name})\Leftarrow{} \cnnt{T}}
    \end{mathpar}
    \caption{Checking rule for the \coreinline{undef()} pure expression as % chktex 36
        mentioned in \textcite{memarian2022cerberus}.}\label{fig:core-ub-typing}
\end{marginfigure}

This is evidenced by the typing rule for \coreinline{undef()} % chktex 36
(\cref{fig:core-ub-typing}). Since \kl{UB} is by definition, assumed to be
absent in a valid C program, the type system must accept any type as validly
checking against \coreinline{undef()}. % chktex 36

At a minimum, to prove C programs safe, what we would like is a type system
expressive enough to \emph{prove} the absence of \kl{UB}. If the type checking
context is enriched to track the control flow path to reach this point in the
program, for example by tracking conditions (or their negations), then the
typing rule for \coreinline{undef()} would simply require that the context % chktex 36
be inconsistent, i.e.\ able to prove false. To illustrate this point with
a trivial example, consider the nested \mintinline{py}{if} statements in
\cref{fig:dead-code}.

\begin{marginfigure}
    \inputminted[breaklines,mathescape,fontsize=\small]{py}{code/dead_code.py}
    \caption{A contrived example on how tracking control flow assumptions
        within a program could be used to prove the impossibility of
        undesirable behaviour.}\label{fig:dead-code}
\end{marginfigure}

Since we know from the outer \mintinline{py}{if} that \mintinline{py}{x} is
greater than or equal to 5, the check for \mintinline{py}{x} being strictly
less than 4 is always going to evaluate to \mintinline{py}{False} and so go to
the \mintinline{py}{else} branch rather than raise the exception. When tracking
constraints, this unreachability or dead code is represented by an inconsistent
context, from which we can prove $\mathsf{false}$.

In general, to express and track these constraints in types, we need to be able
to talk about computational variables and values \emph{at the type level},
introducing a form of dependency. However, because we do not want to change the
programming language whose type system we are enriching (such as in a
\kl{dependent type} system) we are not adding the ability for computation to
depend on types, so we retain the property of recovering the original program
and dynamic semantics after erasing the expressive types we fit on top of them.

This approach of enriching the expressiveness of types on an existing language
whilst retaining the erasure is known as \intro{refinement types}. There is a
rich history of the origins, naming and development of these
ideas,~\sidecite{jhala2021refinement} and so I will focus only on the strands
relevant to \kl{CN}.

If our goal is to minimise annotations and not require proofs for obvious
statements, then we need some sort of automation to dispatch proof
obligations.\sidenote{Recall the examples shown in
\cref{fig:call-incr,fig:call-incr-fail}, where \kl{CN} deduces
\cinline[breaklines]{y <= MAXi32()} from the constraint \cinline[breaklines]{y
<= 100i32}.} However, once we have rich constraints being tracked by the type
system, we have a first-order logic with arithmetic, proving which is
undecidable in general. There are type systems for real-world
programming languages (such as Hack or F$^*$) which are undecidable, and so
could just never terminate on some code, but have automation which works fine
in practice, so why does this point require any special consideration?

Aside from the pleasing theoretical property of having a decidable type system,
or having to tell users ``just increase the timeout'',  there are a couple of
factors which lean us in this direction, namely to be able to use SMT solvers
in a \emph{fast} and \emph{predictable} manner. Recognising that whilst
decidability can still technically mean exponential, or very often, NP-complete
complexity for problems does not negate the empirical observation that in
practice, decidable theories tend to be quite fast, and their speed (and
solvability) much less susceptible to heuristics being triggered. It is a proxy
for performance, albeit an imperfect one.

It also helps that the size of SMT problems generated by type checking
\emph{tend} to be reasonably small, at least compared to the sorts of problems
which SMT solvers are typically used for, because the amount of code humans
write between annotations is limited by cognitive capacity (and hopefully,
diligent code reviewers).

\kl{CN} ensures decidability by following restrictions on quantifiers as set
out by the liquid types~\sidecite{rondon2008liquid} approach. In practice, this
means ensuring that the constraints sent to the SMT solver are free of
quantifiers. When this approach is insufficiently expressive, \kl{CN} offers a
lemma mechanism to export and prove claims in the \kl{Rocq} interactive proof
assistant.

Aiming for decidability also has the added benefit of extracting a concrete
\kl{counter-example} (see \cref{sec:counter-ex}), which is used to explain
constraint failures in terms of concrete values for \kl{Core} program variables
(a subset of which correspond to C program variables).

\section{Bidirectionality for taming subtyping}\label{sec:bidir-subtyping}

Using a liquid typing discipline automates away the problem of tedious proof
obligations with calls to an SMT solver, but introduces the new problem of \emph{when}
during type checking is the right time to use it? This is because implications
induce a subtyping relation in the type system, and so the question becomes:
when is the correct time to use the subsumption rule, as shown in
\cref{fig:subsumption}?

\begin{marginfigure}
  \begin{mathpar}
      \inferrule{\Gamma{} \vdash{} e \mathrel{{:}} \{\, x \mid{} \phi (x) \,\}  \\ \phi (x) \Rightarrow{} \phi' (x) }
                {\Gamma{} \vdash{} e \mathrel{{:}} \{\, x \mid{} \phi' (x) \,\}}
  \end{mathpar}
  \caption{Subsumption rule using for a system with subset types and logical
      implication $\Rightarrow$ as the subtyping relation.}\label{fig:subsumption}
\end{marginfigure}

Going back to the example in \cref{fig:dead-code}, let us imagine we are in a
type system where the contexts $\Gamma$ are either $\cdot$ or $\Gamma, \{\, x \mid
\phi (x) \,\}$, where $\phi$ represents some constraint on $x$. In particular, we
are not accumulating control flow constraints in the context, only
in the variable's subset type. We would like to check that the function
\mintinline{py}{gt_zero} satisfies its postcondition. $ \{\, \mathsf{ret} \mid
\mathsf{ret} > 0 \,\} $, i.e.\ it returns a value greater than 0 and does not
raise an exception.

Working forwards, we start off with a type for $\{\, \mathsf{x} \mid \top \,\}$.
Inside the function, we are checking both branches because we cannot statically
rule one of them out. In the true case, we now \emph{refine} the type to $\{\,
\mathsf{x} \mid \mathsf{x} \geq 5 \,\}$. Because we are not sure when to apply
the subsumption rule, and we can apply it at any time, we apply it now, and
have $\{\, \mathsf{x} \mid \mathsf{x} > 0 \,\}$, because $\mathsf{x} \geq 5
\Rightarrow \mathsf{x} > 0$. At this point, we cannot statically rule out the
\cinline{else}-branch of the condition, so we start checking the one with the
exception, instead of skipping over it as dead code.

Whilst quite contrived, it illustrates the point that we would like some sort
of principled scheme to know when to use the subsumption rule, and with which
constraints. We could avoid this by requiring users annotate their code at
every step, but though this would be a predictable annotation scheme, it would
be extremely burdensome. Using an SMT solver is meant to reduce the tedium of
proving straightforward facts, not just trade it for the tedium of requiring a
type annotation at each step.

At the same time, while it would be great if we could infer all of them, this
is often not necessary and counter-productive in practice because annotations
on top-level functions (most commonly, types) serve as very useful
documentation, which programmers are already used to writing.

Another reason \kl{CN} is quite willing to accept per function annotations is
that it allows checking a large code base to be parallelised, and obviate the
need to do any complex and potentially slow inter-procedural analysis.
Unfortunately, since loops are effectively function tail calls, and invariants
are also difficult to infer, \kl{CN} requires annotation for them too.

Still, pre- and postconditions on functions, and invariants on loops is a crisp
way to explain to programmers where annotations need to be placed. What we
would like to avoid is annotations on each statement of a C program.

More concretely, in the presence of constraints due to refinement types, this
means we would like \emph{check} types at function calls, function returns,
before and after loops from the top down, and \emph{synthesise} types
everywhere else from the bottom up. This is precisely the sort of problem a
bidirectional type system can solve (see sections 4.4 and 4.6
in~\sidetextcite{dunfield2021bidir}). Moreover, it gives a clear principle of
when to use subsumption rule (call the SMT solver): at the boundary between
synthesis and checking (\cref{fig:simple-bidir}).

\begin{figure}
    \begingroup%
    \newcommand{\synths}[1]{\Rightarrow{} \outpol{#1}}
    \newcommand{\checks}[1]{\Leftarrow{} #1}
    \begin{mathpar}
    \begin{tabular}{l@{}l@{ }l@{ }l}
    \cncom{types}       & $T$      & $\mathrel{{\Colon}{=}}$ & $\cnkw{unit} \mid{} T_1 \rightarrow{} T_2$ \\
    \cncom{expressions} & $e$      & $\mathrel{{\Colon}{=}}$ & $x \mid{} \cnkw{()} \mid{} (e {:} T) \mid{} \lambda{} x.\ e \mid{} e_1\ e_2$ \\
    \cncom{context}     & $\Gamma$ & $\mathrel{{\Colon}{=}}$ & $\cdot \mid{} \Gamma{}, x {:} T$ \\
    \\
    \multicolumn{2}{l}{%
        \multirow{2}{*}{%
        \begingroup%
        \setlength{\fboxsep}{2pt}
        \fbox{%
            \shortstack{$\Gamma{} \vdash{} e \synths{T}$ \\ $\Gamma{} \vdash{} e \checks{T}$ }%
        }%
        \endgroup%
        }%
    }
                         & \multicolumn{2}{l}{\cncom{given $\Gamma$, $e$ synthesises $T$}} \\
    \multicolumn{2}{l}{} & \multicolumn{2}{l}{\cncom{given $\Gamma$, $e$ checks $T$}} \\
    \end{tabular}
    \and
    \inferrule[{[Var]}]
        {(x {:} T) \in{} \Gamma}
        { \Gamma{} \vdash{} x \synths{T}}
    \and
    \inferrule[{[Sub]}]
        { \Gamma{} \vdash{} e \synths{T_1} \and T_1 \cnkw{<:} T_2}
        { \Gamma{} \vdash{} e \checks{T_2}}
    \and
    \inferrule[{[Annot]}]
        { \Gamma{} \vdash{} e \checks{T} }
        { \Gamma{} \vdash{} (e {:} T) \synths{T} }
    \and
    \inferrule[{[$\cnkw{unit}$I]}]
        { }
        { \Gamma{} \vdash{} \cnnt{()} \checks{\cnkw{unit}} }
    \and
    \inferrule[{[$\rightarrow{}$I]}]
        { \Gamma{}, x {:} T_1 \vdash{} e \checks{ T_2 } }
        { \Gamma{} \vdash{} \lambda{} x.\ e \checks{ T_1 \rightarrow{} T_2}}
    \and
    \inferrule[{[$\rightarrow{}$E]}]
        { \Gamma{} \vdash{} e_1 \synths{T_1 \rightarrow{} T_2} \and \Gamma{} \vdash{} e_2 \checks{T_2}}
        { \Gamma{} \vdash{} e_1\ e_2 \synths{T_1 \rightarrow{} T_2}}
    \end{mathpar}
    \endgroup%
    \caption{Simple \intro{bidirectional} type system, adapted
    from~\cite{dunfield2021bidir}.  Note that its splits the traditional
    $\Gamma{} \vdash{} e : T$ judgement, into two mutually recursive judgements,
    one for \intro[synthesis]{synthesising} a type and one for checking a type.
    %
    % Here, introduction forms for $\cnkw{unit}$ and $\rightarrow$
    % (namely $\cnkw{()}$ and $\lambda x.\ e$ respectively) check their types;
    % variables, annotated terms and elimination forms for $\rightarrow$ (namely
    % function application $e_1\ e_2$) synthesise their types.
    I highlight synthesised types in a \colorbox{pink!30}{light pink} to make
    them easier to spot.
    %
    If a term is syntactically synthesising, but is used in a checking position,
    then \textsc{Sub} allows it to be checked by first synthesising a type $T_1$
    for it, and then checking whether $T_1 <: T_2$. Here, the subtyping
    relation $\cnkw{<:}$ is merely structural type equality $=$, but in the
    presence of refinement types, it would be logical
    entailment.}\label{fig:simple-bidir}
\end{figure}

\section{Linearity to manage (non-leaky) resources}

Of course, if the aim is define a type system rich enough to prove the absence
of \kl{UB}, we must acknowledge the fact that there are plenty of ways of
getting \kl{UB} which do not rely on pure fragment evaluation rules of the
\kl{standard} being violated, but based on effectful fragment, such as loose
evaluation order or a myriad of subtle memory violations.

In \nameref{sec:sep-logic-intro}, I sketched out how separation logic can be
used to reason about the behaviour of imperative programs which manipulate a
heap and can have syntactically distinct terms \intro{alias} the same location
in said heap. If we want to embed separation logic propositions into a type
system, then we need to be able to precisely control how those propositions are
handled.

Whereas in a typical (intuitionistic) type system, duplicating an assumption in
the context is perfectly acceptable, in a separation logic setting, this would
mean that one could take $x \mapsto{} y \ast{} y = 5$, duplicate the points-to
to get $x \mapsto{} y \ast{} x \mapsto{} y \ast y = 5 \vdash{} x \mapsto{} 0$,
instead of $x \mapsto{} 5$ as we would like. This is because the semantics of
$\ast{}$ means that the context entails $\mathsf{False}$. Such duplication is
known as \intro{contraction} and we need to forbid this to reason soundly
(whilst keeping other assumptions, such as the type of \kl{Core} program
variables, and constraints, contractible).

\begin{marginfigure}
  \begin{mathpar}
      \inferrule[{[Weakening]}]{
          \Delta{}, A \vdash{} C
      }{
          \Delta{} \vdash{} C}
      \\
      \inferrule[{[Contraction]}]{
          \Delta{}, A, A \vdash{} C
      }{
          \Delta{}, A \vdash{} C}
      \\
      \inferrule[{[Exchange]}]{
          \Delta_1, B, A, \Delta_2 \vdash{} C
      }{
          \Delta_1, A, B, \Delta_2 \vdash{} C}
    \\
  \end{mathpar}
  \caption{Substructural sequent calculus rules.}\label{fig:substructural}
\end{marginfigure}

Similarly, in an intuitionistic setting, we can freely forget things we know,
in a separation logic setting this really depends on whether the user of the
logic cares about memory leaks or not. In the C standard, nothing relies on an
allocation being \emph{dead}, it only relies on it being \emph{live}, or not
caring at all. So whilst the theory and the C standard leave this open as a
design choice, the intended use of \kl{CN} in resource constrained settings
requires us to be explicit about when we deallocate and more importantly
\emph{when we forget to}.\sidenote{Though not yet supported by \kl{CN},
concurrency also requires being explicit about destroying a resource, for
example, to prevent a user from acquiring the same mutex twice.} This
`forgetting' an assumption is known as \intro{weakening}, and given the
intended use case we would like to forbid this too (whilst not requiring type
information and ambient constraints to be forgotten freely).

We would however like the ability to commute assumptions freely, since it is
perfectly acceptable to free in any order, regardless of the order of
allocation, so we keep \intro{exchange}. Sequent calculus rules for contraction,
weakening, and exchange are shown in \cref{fig:substructural}. Jettisoning
weakening and contraction but keeping exchange means that we handle our
(heap-related) separation logic assumptions
\emph{linearly}~\sidecite{girard1987linear} in \kl{CN}.

Note that this is in constrast to popular approaches such as the drop trait in
Rust,\sidenote{\url{https://doc.rust-lang.org/std/ops/trait.Drop.html}} or RAII
(Resource Acquisition Is Initialisation) in
C++,\sidenote{\url{https://isocpp.github.io/CppCoreGuidelines/CppCoreGuidelines\#Re-raii}}.
These allow the programmer to elide explicit resource deallocation, by
automatically inserting such a call when the resources goes out of scope.
Modelling that would require an \kl{affine} rather than linear resources.

Linearity may be necessary, but it can also be cumbersome. If a linear resource
represents the permission to dereference a pointer, a typing rule which
consumes that must also return an identical resource, so that it may be
dereferenced again. If we require the programmer to handle these permissions
explicitly in the annotations, such a discipline would quickly violate the goal
to minimise obvious annotations. This means that in addition to automation for
proving constraints, by using SMT solvers, we need to have some scheme to
\intro[resource inference]{infer resources} (see \cref{sec:elaboration}) to
reduce the burden of annotations.

\begin{marginfigure}
  \begin{mathpar}
      %\inferrule[Declarative Var]{
      %}{
      %    x {:} t \vdash{} x {:} t}
      %\\
      %\inferrule[Algorithmic Var]{
      %}{
      %    \Gamma{} , x {:} t \vdash{}  x {:} t \dashv{} \Gamma'}
      %\\
      %\{\, x {:} t \,\} = \Gamma' - \Gamma
      %\\
      \inferrule[{[Declarative Pair]}]{
          \Delta_1 \vdash{} e_1 {:} T_1
          \and
          \Delta_2 \vdash{} e_2 {:} T_2
      }{
          \Delta_1, \Delta_2 \vdash{} (e_1, e_2) {:} T_1 \times{} T_2}
      \\
      \inferrule[{[Algorithmic Pair]}]{
          \Gamma_1 \vdash{} e_1 {:} T_1 \dashv{} \Gamma_2
          \\
          \Gamma_2 \vdash{} e_2 {:} T_2 \dashv{} \Gamma_3
      }{
          \Gamma_1 \vdash{} (e_1, e_2) {:} T_1 \times{} T_2 \dashv{} \Gamma_3}
      \\
      \Delta_1 = \Gamma_2 - \Gamma_1 \and \Delta_2 = \Gamma_3 - \Gamma_2
  \end{mathpar}
  \caption{Declarative ($\Delta$) versus algorithmic ($\Gamma$) typing for
  linear pairs. A declarative system is easier to describe and
  prove sound, because the recursion is obviously structural, but is not
  directly implementable: for pairs, the correct split of the contexts must be
  guessed before checking each component. In contrast, an algorithmic system
  changes and chains the contexts across premises, which is easy to implement as
  mutating state, but requires a notion of ``context difference'' to relate to
  the declarative version.}\label{fig:decl-alg}
\end{marginfigure}

I should note that, for the purposes of the formalisation of \kl{Kernel CN}, I
do use explicit linear resource terms, since that makes the system easier to
state and prove sound. For clarity and convenience, the linear context in the
system is used in a \intro{declarative} manner (\cref{fig:decl-alg}).
Explicit linear resource terms also makes it easy to specify and explain the
\kl{CN} \kl{resource inference} algorithms as an \emph{elaboration}, which I
will also do.\sidenote{I have not embarked on any proofs about the properties of
these inference algorithms.} Those rules use the linear context in
an algorithmic manner.

\section{Precise assertions for inferring resources and quantifiers without
backtracking}\label{sec:precise-assertion-inferring}

Lurking not too far in the background of the decision to use linear separation
logic and liquid types is a tension between the rich scheme of quantifiers
needed to write specifications expressive enough to prove \kl{UB}-freedom (if
not full functional correctness, where those can be separated) and the desire
to (a) not send quantified formulas to the SMT solver and (b) minimise the
annotations (quantifier instantiations) a user needs to write.

So aside from minimising annotations for proofs of obvious statements (by
carefully using an SMT solver) and minimising annotations for linear resource
terms (by inferring them), we would also, as far as possible, like to
infer how quantifiers are instantiated, based on what the user wrote.\sidenote{
Note that this is not at all related to \emph{inferring specifications}, which
is a large and challenging problem in its own right.} This is a tall order,
because the problem is not decidable in
general~\sidecite{turing1936computable,church1936unsolvable}.

Yet, we can still proceed and make the problem clearer by using a concrete
example. Let us recall the separation logic proof sketch of linked integer list
append from \cref{fig:append-annot,fig:list-pred}, reproduced in
\cref{fig:append-annot-formal,fig:list-pred-formal} for convenience.

\begin{marginfigure}
    \small
    \centering
    \begin{align*}
        \mathrm{list} &(\mathsf{p}, l) \mathrel{{=}^\mathrm{def}} \\
                      &\mathsf{emp} \astRef{} (\mathsf{p} = \mathsf{NULL} \wedge{} l = []) \\
                      &\vee{} \exists{} \; {head}, \; {tl}, \mathsf{p\_tail}.\\
                      &\qquad (\mathsf{p} \mapsto{} {head}) \\
                      &\qquad \astRef{} (\mathsf{p} + 1 \mapsto{} \mathsf{p\_tail}) \\
                      &\qquad \astRef{} \mathrm{list} (\mathsf{p\_tail}, {tl}) \\
                      &\qquad \astRef{} l = {head} {:}{:} {tl} \\
    \end{align*}
    \caption{Definition of a recursive list predicate in a simple separation
        logic.}\label{fig:list-pred-formal}
\end{marginfigure}

\begin{figure}[h]
    \inputminted[breaklines,mathescape,fontsize=\small]{py}{code/append_annot.py}
    \caption{A separation logic proof sketch of a linked integer list
        append.}\label{fig:append-annot-formal}
\end{figure}

We see that in addition to the pointers $\mathsf{xs}$ and $\mathsf{ys}$, which
can be thought of as a quantification over \intro{computational} variables, the
specification for append also quantifies over \emph{logical} or ghost variables
$l_1$ and $l_2$. Given that the ``computational terms'' are just code, the
computational variables need not be inferred, since the user needs to provide
them for the call to the function to be valid.\sidenote{For the rest of this
section, I am only going to talk about quantifiers for function calls, but the
exact same principle applies for dealing with both computational and logical
return values too. I will discuss quantifiers related to array reasoning in
\cref{sec:it-array}.}

However, it would be a shame if the user had to provide explicit instantiations
for $l_1$ and $l_2$ at each call site, including the recursive call within the
implementation of \mintinline{py}{append}. If we want to stick to a liquid
typing discipline, where every quantifier must be matched up to a program
variable, we need a way to bind the ghost values which were existentially
quantified over in the $\mathrm{list}$ predicate of
\cref{fig:list-pred-formal}, to names the user would have to choose.

This seems even more of a tragedy because before the recursive function
call, we already know (a) $\mathrm{list}(\mathsf{xs}', l_1')$ and
$\mathrm{list}(\mathsf{ys}', l_2')$ and (b) the precondition requires choices
of $l_1$ and $l_2$ such that $\mathrm{list}(\mathsf{xs}', l_1)$ and
$\mathrm{list}(\mathsf{ys}', l_1)$ so there is \emph{only one sensible choice}
for those quantifiers.

To recap, when calling a function, we need some way of guessing some
instantiation of quantifiers such that it satisfies the precondition. \emph{If}
we are in a situation where there is only one sensible instantiation given the
required and the available assertions, a simple inference scheme would be to
simply scan the context for predicates which match on constructor (i.e.\
$\mathsf{emp}$ matches with $\mathsf{emp}$, $\_ \mapsto{} \_$ matches $\_
\mapsto{} \_$, $\mathrm{list(\_, \_)}$ matches $\mathrm{list}(\_, \_)$) and
\emph{computational} arguments, and instantiate any quantifiers based on the
remaining values.

So in this instance, when calling \mintinline{py}{append(xs->tail, ys)}, % chktex 36
such an inference scheme would
\begin{enumerate}
    \item Delay instantiating $l_1$ and $l_2$.
    \item Note that the precondition requires $\mathrm{list}(\mathsf{xs}',
        l_1)$ and $\mathrm{list}(\mathsf{ys}', l_2)$.
    \item Check the context for a match with $\mathrm{list}(\mathsf{xs}',
        \_)$.\sidenote{Checking whether two symbolic terms are equal can be
        automated with an SMT solver.}
    \item Find $\mathrm{list}(\mathsf{xs}', l_1')$.
    \item Select $[l_1' / l_1]$ as its instantiation.
    \item Repeat similarly for $\mathsf{ys}$ and $l_2'$.
\end{enumerate}

It is a cute idea, but it glosses over several details, such as
\begin{itemize}
    \item Are such assertions characterisable, and if so, how?
    \item Can such a scheme handle disjunction, $ P \vee{} Q$?
    \item Are assertions where \kl{computational} arguments uniquely determine
        \kl{logical} arguments, expressive enough for realistic code?
    \item What is the correct atomic predicate to look up in a context?
    \item Can all assertions be decomposed into a context of such atomic
        predicates?
    \item Should it unfold recursive predicates, and if so, when?
\end{itemize}

I will tackle the first three questions here, and leave the latter three
to be discussed in~\nameref{sec:heap-types}.

For the first question, the answer is: in practice, \emph{yes}. If one assigns
input and output \kl{modes} to the arguments of separation logic predicates,
and ensures that predicates are \emph{mode-correct} via a syntactic check
(\nameref{sec:monadic-syntax}), then that suffices to guarantee that the system
can \emph{always infer quantifiers}.

However, the theoretical justification for this unclear. Such assertions happen
to be characterisable as \kl{precise}~\sidecite{reynolds2008intro}, but the
connection to modes has not been formally explained.

\subsection{Precise assertions}\label{subsec:precise-assertion}

\begin{definition}[Precise assertions]%
\label{def:precise-assertion}
    \AP{} An assertion $Q$ is \intro{precise} iff, for all stores $s$, and heaps
    $h$, there is at most one $h' \subseteq{} h$ such that $h' \in [\![ Q ]\!] (s)$.

    In other words, if a \kl{precise} assertion holds on a subheap, then it
    does so uniquely.
\end{definition}

Based on this definition, one can derive some examples and properties of
precise assertions as shown in~\cref{fig:precise} (examples of imprecise
assertions are shown in~\cref{fig:imprecise}). As I shall show later, each
production in the grammar for \kl{linear} \kl{resource types}
(\cref{fig:kernel-res}), stays within the precise fragment.

\begin{marginfigure}
\begin{mathpar}
    \mathsf{emp}
    \and p \mapsto v
    \and \exists v.\ p \mapsto v
    \and \mathrm{list}(\mathsf{p}, l)
    \and \exists l.\ \mathrm{list}(\mathsf{p}, l)
    \and P \wedge Q \text{ when $P$ or $Q$ is precise}
    \and P \ast Q \text{ when $P$ and $Q$ are precise}
\end{mathpar}%
\caption{Some examples and properties of precise assertions.}\label{fig:precise}
\end{marginfigure}

\citeauthor{brotherston2016model}~\sidecite{brotherston2016model}~showed that
precise assertions allow for efficient model checking without backtracking;
in~\nameref{sec:elaboration}, I show how the same allow for \kl{CN} to infer
instantiations of quantifiers without backtracking. I do so by leaning on input
and output \kl{modes} which (a) is intuitive, given that in C source a pointer
is an ``input'' and dereferenced value an ``output'' and (b) happens to
coincide.

This brings us nicely to the second question about disjunctions. As visible
from \cref{fig:imprecise}, we see that arbitrary disjunctions pose a problem.
Yet, from \cref{fig:precise}, we see that the list predicate, and even a
version of it which is existentially quantified over its ghost list fits the
definition of precise, despite including a disjunction in its definition
(\cref{fig:list-pred-formal}).

\begin{marginfigure}
\begin{mathpar}
    \mathsf{true}
    \and \mathsf{emp} \vee{} \mathsf{x} \mapsto{} 42
    \and \mathsf{x} \mapsto{} 3 \vee{} \mathsf{y} \mapsto{} 7
    \and \exists \mathsf{p}.\ \mathsf{p} \mapsto{} 1
    \and \exists \mathsf{p}.\ \mathrm{list}(\mathsf{p}, l).
\end{mathpar}
\caption{Examples of im\kl{precise} assertions.}\label{fig:imprecise}
\end{marginfigure}

This is because the two arms of the disjunction make assertions
which are disjoint on \emph{both} the values the computational parameter
$\mathsf{p}$ can take \emph{and} on the shape of the heap; one branch says
$\mathsf{p} = \mathsf{NULL}$ and the heap will be $\mathsf{empty}$, whereas the
other branch guarantees that (implicitly) $\mathsf{p} \neq \mathsf{NULL}$ and
that the heap will have \emph{at least two} heap cells. I will return to the
discussion of disjunctions in precise predicates imminently; for now I want to
note that the locations of those two cells are expressed entirely in terms of
program variables.

At this point, a keen reader may have noticed that while this may be true at
the `top-level' of the predicate, the recursive call does use the existentially
quantified ghost value $\mathsf{p\_tail}$, which is on the right of $\mathsf{p}
+ 1 \mapsto{} \mathsf{p\_tail}$. So why is this definition still \kl{precise}?
It is because in any heap which satisfies the non-empty branch of the list
predicate, $\mathsf{p\_tail}$ is \emph{uniquely} determined by its relation as
the value in the location adjacent to $\mathsf{p}$.\sidenote{A proof of this is
left as an exercise to the reader.} This can be operationalised with a \kl{mode}
discipline for predicate arguments, which I shall discuss in
\nameref{sec:monadic-syntax}.

Finally, the third question is an empirical one, but our experience using
\kl{CN} so far~\sidecite{pulte2023cn,pulte2024tutorial} suggests the answer is
yes, it is sufficient for small to medium tutorial examples, all the way up to
specifying the buddy allocator (\cref{chap:buddy}) used in pKVM\@.

That being said, we have found that being able to abstract over, and supply
quantifiers manually can sometimes be useful for ergonomics, even if not
strictly necessary. An example came from the efforts of Cassia Torczon to
verify a basic string manipulation library in \kl{CN}.%
\sidenote{\href{https://github.com/rems-project/cerberus/issues/540}{see
Cerberus\#540}} The natural representation of null-terminated strings, as
required by many of the library functions, is a recursive predicate. However,
the implementation of such library functions use while-loops and
array-indexing, for which it is convenient (but not neccessary) for proof (and
library clients) to express the string as an array (represented by an iterated
predicate), over a finite range. The end of that range is fixed, but not
present before looping and indexing in the program code. Hence to refer to it
in the specification, we need the ability to abstract over and supply
quantifiers manually.

\section{Monadic syntax to mode-correctness}\label{sec:monadic-syntax}

\intro[mode]{Modes} help us ensure that we can infer quantifiers instantiations
as follows. We start with separation logic predicates, and label their
arguments as either $\mathsf{in}$puts or $\mathsf{out}$puts, as shown in
\cref{fig:mode-pred}. In the scope of each quantifier, for each predicate, we
check the mode of each parameter matches the mode of each argument.
Computational arguments the user provides are considered inputs. If a
quantified variable is used in an input position, it must also be used
elsewhere in an output position.

\begin{marginfigure}
    \centering
    \begin{align*}
        \_ \mapsto{} \_ \:&: \mathrm{Loc}^{\mathsf{in}} \times \mathrm{Value}^{\mathsf{out}} \rightarrow \mathrm{Prop} \\
        \mathrm{list} \:&: \mathrm{Loc}^{\mathsf{in}} \times \mathrm{IntList}^{\mathsf{out}} \rightarrow \mathrm{Prop}
    \end{align*}
    \caption{Modes on separation logic predicates.}\label{fig:mode-pred}
\end{marginfigure}

So for example, the second disjunct of \cref{fig:list-pred-formal}, the
quantified variable $\mathsf{p\_tail}$, occurs in $\mathrm{list}
(\mathsf{p\_tail}, {tl})$ in the input position, but this is fine because it also
occurs in the output position in $\mathsf{p} + 1 \mapsto{} \mathsf{p\_tail}$.
However, in the example from \cref{fig:imprecise}, $\exists \mathsf{xs}. \
\mathrm{list}(\mathsf{xs},l)$ is badly moded because $\mathsf{xs}$ occurs in an
input position, but nowhere else in an output.

Ensuring that quantifiers preserve \kl[precise]{precision} in assertions by
dividing up predicate arguments into input and output modes is
not new~\sidecite{somogyi1996execution,berdine2006smallfoot,nguyen2008runtime,maksimovic2021gillian},
including for the purposes of reducing the burden of
annotation~\sidecite{jacobs2011verifast}.

What is new is the target audience of kernel programmers, rather than
verification specialists. Programmers may not be used to thinking in terms of
ghost state, and discussions with our users suggest that adding quantifiers and
\kl{mode}s on top of that would be step too far.

An elegant solution to both inscrutable mode rules, and the problem of
disjunctions mentioned earlier, is to change the perspective on
\cref{fig:mode-pred}~\sidecite{krishnaswami2022monadic}. A predicate
$\mathrm{list} : \mathrm{Loc} \times \mathrm{IntList} \rightarrow
\mathrm{Prop}$, considered set-theoretically is just $\mathcal{P} (\mathrm{Loc}
\times \mathrm{IntList})$, which is isomorphic to $\mathrm{Loc} \rightarrow
\mathcal{P} (\mathrm{IntList})$. Unlike the initial form, which gave ``equal
weighting'' to input and outputs, this form makes the input role of
$\mathrm{Loc}$ and the output role of $\mathrm{IntList}$ very natural.

\begin{marginfigure}
    \begin{align*}
        &\llbracket \_ \rrbracket \::\: \mathcal{P}(\tau) \rightarrow (\tau \rightarrow \mathsf{Prop}) \\
        &\llbracket \mathsf{return}\ t \rrbracket = \lambda a.\ t = a \wedge{} \mathsf{emp} \\
        &\llbracket \mathsf{Own(t)} \rrbracket = \lambda a.\ t \mapsto{} a \\
        &\llbracket \mathsf{let\ x} = e; e' \rrbracket \\
        &\qquad = \lambda a.\ \exists x.\ \llbracket e \rrbracket x \astRef{} \llbracket e' \rrbracket a \\
        &\llbracket \mathsf{if}\ t\ \mathsf{then}\ e\ \mathsf{else}\ e' \rrbracket \\
        &\qquad = \lambda a.\  \left( t \wedge \llbracket e \rrbracket a \right) \vee \left( \neg\;t \wedge \llbracket e' \rrbracket a \right)
    \end{align*}
    \caption{Monadic syntax for separation logic, along with a translation into the traditional presentations. Pure
        terms are denoted by $t$, and monadic expression are denoted with $e$.}\label{fig:monad-sl}
\end{marginfigure}

Not only does this notational transformation turn \emph{mode correctness} into
\emph{variable scoping} with the $\mathsf{let\ x} = e; e'$ construct, it also
enforces the disjointness of computational values and the shape of the heap
mentioned earlier with the $\mathsf{if}\ t\ \mathsf{then}\ e\ \mathsf{else}\
e'$ construct. Not only can all logical quantifiers now be \emph{inferred}
without backtracking, branching assertions can be checked without backtracking
too: check $e$ if you can can prove $t$ and check $e'$ if you can prove
$\neg\;t$.

\subsection{\kl[iterated]{Iterated} separating conjunctions to handle arrays}\label{sec:it-array}

C programmers use arrays a lot, particularly with computed index (i.e.\
`random') access, rather than following a particular traversal pattern to get
some element in the middle.

Typically (such as in VeriFast) arrays are handled via recursive predicates,
and if the order of traversal in C does not match the order of traversal in the
predicate, or access is required to the middle of the array, then a user needs
to use lemmas of some sort to massage the available assertions into a usable
form.

\kl{CN} implements \intro{iterated separating conjunctions}, inspired by
\sidetextcite{muller2016automatic}, but with restrictions to ensure only
quantifier-free SMT queries. Specifically,
\begin{itemize}
    \item Iterations must take the form
        \cninline[breaklines]|each (<type> i; <guard>) { <pred>( array_shift(p i) ) }|, % chktex 36 chktex 37
        where
        \begin{itemize}
            \item \cninline{i} is the iterating index being quantified over
            \item \cninline{<guard>} is a boolean condition on \cninline{i}
            \item \cninline{p} is a base pointer
            \item \cninline{<pred>} is the name of a named predicate
            \item \cninline{array_shift} is the specification language syntax
                for C's \cinline{p + i}.
        \end{itemize}
    \item Indices which are intended to move into or out of an iterated separating
        conjunction must be declared so explicitly with a \cninline{focus <idx>;}
        statement.
\end{itemize}

This enables a few convenient features:
\begin{itemize}
    \item Iterated separating conjunctions can be tested for equality by
        looking only at the name of the predicate being iterated over, symbolic
        equality of the base pointer, and symbolic equivalence of the guard
        conditions.
    \item Predicates can be moved out of and into iterated separating conjunctions
        automatically.
    \item Iterated separating conjunctions can be split along an arbitrary
        guard automatically in a similar fashion, except `subtraction' is done as follows:
        for guard \cninline{A = 0i32 <= i && i < 10i32} and guard
        \cninline[breaklines]{B = A && mod(i, 2i32) == 0i32}, % chktex 36
        the subtraction is simply \cninline{A && !B}. Note that the split % chktex 26
        does not have to be contiguous.
\end{itemize}

\kl{CN} does not however, support automatically \emph{merging} arrays, because
the constraints required to express the \emph{merged values} are outside the
decidable fragment for the SMT theory of arrays.\sidenote{So \kl{CN} could, but
does not, support merging for uninitialised arrays.} Relatedly, \kl{CN} did
support the decidable array property fragment,~\sidecite{bradley2006whats}, but
it was not expressive enough for verifying pKVM's \kl{buddy allocator}
(\cref{chap:buddy}) and so it was removed. \kl{CN} also supported a scheme for
inferring indices (instead of requiring the movable ones to be declared up
front), but this was removed during a change to the inference
scheme.\sidenote{\href{https://github.com/rems-project/cerberus/commit/7c2c0a364a4373e4eb109f32d01cc9584f51e81f}{Commit
7c2c0a36.}}\label{sn:new-inf}

\section{Alternatives and related work}

\begin{itemize}
    \item Do I want to talk about this here?
\end{itemize}


\chapter{Kernel CN:\ Grammar}%
\label{chap:kernel-grammar}

\kl{CN} and \kl{Kernel} CN have different grammars. This is because \kl{CN} is
intended to be used by C programmers, whereas \kl{Kernel} CN is more for type
theorists, and also to be more convenient to work with as a formalism. The
primary differences are (a) \kl{CN} is implemented over (a version of)
\kl{Core}, whereas the \kl{kernel} is defined over a let-normalised version of
\kl{Core} (b) \kl{CN}'s grammar of types is close to the surface syntax and so
each construct serves many purposes whereas the \kl{kernel}'s grammar of types
is more traditional and each construct only serves one purpose.

In this chapter I will present the relevant parts of \kl{CN}'s syntax of
predicate definitions and assertions, and the \kl{kernel}'s syntax of types and
relevant terms, with a particular focus on explicit resource terms.

\section{CN Syntax}%

\begin{figure*}[tp]
    \centering
    \includegraphics{figures/cn-grammar-1}
    \includegraphics{figures/cn-grammar-2}
    \includegraphics{figures/cn-grammar-3}
    \caption{Grammar of CN.}\label{fig:cn-grammar}
\end{figure*}

A file for \kl{CN} consists of series of top-level declarations of annotated C
functions, (separation logic) predicate definitions, (purely logical) function
definitions, and datatype declarations.\sidenote{Does the kernel formalism
support datatypes?} 

Function definitions for C introduce the identifiers for the arguments into the
scope of the pre- and postconditions, preceded by \cninline{requires} and
\cninline{ensures} respectively. Pre- and postconditions are a list of
`conditions', each followed by a semi-colon.
\begin{itemize}
    \item \cninline{take id = resource} is a \kl{monadic} bind, which bind
        the \emph{output} arguments of the resource to the identifier \cninline{id}.
    \item \cninline{constraint} is a boolean-valued expression.
    \item \cninline{let id = t} is simply an abbreviation for the expression \cninline{t} bound to \cninline{id}.
\end{itemize}

\kl{CN} constraints are either simple terms, or quantified
constraints.\sidenote{These must be manually instantiated by the user.}

\kl{CN} \kl{resource}s are simply a predicate \cninline{p(t1, .., tn)} % chktex 26 chktex 12 chktex 36
or an iterated predicate of the form
\cninline[breaklines]|each (<type> i; <guard>) { <pred>( array_shift(p i) ) }|. % chktex 36 chktex 37
Predicate names \cninline{p} are either \cninline{Owned<ct>}, representing ownership
of an initialised (read and write) location (a points-to $\mapsto{}$) indexed
by a C type, \cninline{Block<ct>} is similar but for an uninitialised (write only) location,
or a user-defined one.

\section{Desugaring CN types into kernel types}\label{sec:desugaring}

\subsection{Heap types}\label{subsec:heap-types}

\subsection{Resource Terms, Quantifier Inference}\label{subsec:resq-inf}

Explain \intro{bidirectional} (for quantifiers), linear resources, constraints.
Explicit witness to having permissions, which are linearly typed (just ingredients).
Explain enough rules \textemdash{} typing, operations, especially the weird heaps.
And of course, type safety statement and its proof.
Linear terms in a refinement type system.

(Dep ML, L3, F star, Jhala, ATS)

Different and unusual compared to Iris style \textemdash{} separation proofs outside the program.

Proof term in one sense, but also factors out operations for resource manipulation.

\subsection{Heaps}


\subsection{Type Safety}

\chapter{Kernel CN:\ Typing rules}%

\chapter{Kernel CN:\ Proof of soundness}%
\label{chap:kernel-soundness}

Weird heaps.

\chapter{Kernel CN:\ Static semantics}%

\margintoc{}

I have covered a large amount of background to the type system so far:
\intro{Core}, liquid types, bidirectional type systems, linear types, precise
separation logic assertions, monadic syntax for the latter and its relation to
kernel syntax for types, let-normalisation and explicit resource terms in
\kl{ResCore}. I use all of these ingredients in defining the type system for
\kl{kernel CN} that I will explain in this chapter. Some of the sections will
based on my contributions to~\sidetextcite{pulte2023cn}%

The \kl{Kernel CN} type system is ordinary \kl{CN}, defined over \kl{ResCore}
instead of \kl{Core}, without any type or resource inference. In particular, It
requires that that all universal quantifiers are explicitly instantiated, that
all existential quantifiers have explicit witnesses, and all resource
operations are embedded into the program itself as linearly typed proof terms.
It does not require proof terms for the logical properties, since by
construction all of the entailments fall into the decidable SMT fragment; many
rules rely on this. The lack of inference make it a simpler language for which
to prove type soundness, whilst still demonstrating all the key ingredients
mentioned above. Since it handles the majority of C, the entire system is very
large, and so I will only discuss the main features.

There are some additional minor differences between the implementation and the
formalisation. As I mentioned \nameref{sec:desugaring}, the formalisation has a
richer grammar of resources: this makes defining predicates to represent tagged
unions more succinct, and allows for opening predicates in more cases. The
formalisation assumes that iterated resources output arguments have type array
of records, whereas the implementation uses records of arrays.\sidenote{This
purely a notational convenience so I could avoid inventing syntax for indexing
over an arbitrary record of arrays.}

Along with the type system, I briefly discuss a formalisation of two different
elaboration algorithms: one for inferring instantiations of logical quantifiers,
one for inferring indices for \kl{iterated} predicates. Because of a change
to the inference scheme used by \kl{CN}, the latter algorithm is no longer
used.\sidenote{\href{https://github.com/rems-project/cerberus/commit/7c2c0a364a4373e4eb109f32d01cc9584f51e81f}{Commit
7c2c0a36.}}\label{sn:new-inf-statics}

\section{Contexts}

The contexts for the static semantics consist of four parts: (1) $\mathcal{C}$
containing the computational variables from the Core program; (2) $\mathcal{L}$
containing purely logical variables mentioned in specifications; (3) $\Phi$,
the constraint context, containing a list of (non-quantified) SMT constraints;
and (4) $\mathcal{R}$ a \emph{linear} context containing the resources
available at that point during type-checking. I assume a constraint context of
only non-quantified constraints because users are required to manually
instantiate quantified constraints to use them.

\section{Pure values and expressions}

\kl{ResCore} programs have both computational and logical (ghost) terms.
Every such term, computational or ghost, has a \kl{base type} $\beta$,
which are things like unit, booleans, (mathematical) integers,\sidenote{After the formalisation was completed,
\kl{CN} switched from using mathematical unbounded integers to bit vectors
(\href{https://github.com/rems-project/cerberus/commit/8fdd4198750446de3b44d00f9e8f185db9610fab}{around
commit 8fdd41987}) to better support common bit-twiddling idioms, used heavily
in the buddy allocator in pKVM, without resorting to lots of lemmas about
uninterpreted functions.} locations, and records and user-defined algebraic datatypes of
other base types. Each C type $\tau$ is mapped to a corresponding base type $\beta_\tau$
\textemdash{} for example, $\beta_{\mathtt{int*}} = \mathsf{loc}$.

Logical terms are variously referred to as ${term}$, ${iguard}$ (for boolean
index guards of iterated predicates), ${ptr}$ (for pointers), ${init}$ (for
initialisation status), ${value}$ (for pointees), ${iarg}$ (for input mode
arguments to predicates),  ${oarg}$ (for output arguments for type record or
array of records), and later, ${alloc}$ (for constraints about the allocation
history).

As seen in \cref{fig:typing-pval-pexpr}, the rules for pure
values\sidenote{Because \kl{Core} factors out the memory object model, it also
factors out the precise representation memory objects such as integers,
pointers, arrays and structs, so I have followed a similar factorising in
the structure of the typing judgements, and left the representation of
these values abstract.} are very simple: they only require a computational
context and synthesise a base type.

Building on the rules for pure values, the rules for pure expressions are not
that much more complicated either, but now it starts to introduce a few more of
the refinement type features gestured at earlier. The type $\Sigma y {:}
\beta.\ \phi(y) \wedge{} I$ is simply the usual refinement type $\{ \, y \in \beta
 \mid\phi(y) \, \}$, translated over to the grammar of types in \kl{Kernel
CN}. Recall that $\Sigma$ is used to bind a computational value in a \emph{return} type,
so these types are simply expressing, symbolically, in constraints, that these
expressions will evaluate to a value. Pure values are simply lifted into the
grammar of SMT constraints with an equality constraint.

\begin{figure*}
    \includegraphics{figures/kernel-pval-typing}
    \includegraphics{figures/kernel-pexpr-typing}
    \caption{\kl{Kernel CN} typing rules for pure values and expressions.}\label{fig:typing-pval-pexpr}
\end{figure*}

\section{Top-level pure value values and expressions}


\section{Resource terms}\label{sec:typing-res-terms}

\section{Pattern matching}

\section{Memory actions and operations}

\section{Spine judgement}

\section{Effectful values and expressions}

\section{Top-level effectful values and expressions}

\section{Elaboration}


\subsection{Heap types}\label{subsec:heap-types}

\subsection{Resource Terms, Quantifier Inference}\label{subsec:resq-inf}

Explain \intro{bidirectional} (for quantifiers), linear resources, constraints.
Explicit witness to having permissions, which are linearly typed (just ingredients).
Explain enough rules \textemdash{} typing, operations, especially the weird heaps.
And of course, type safety statement and its proof.
Linear terms in a refinement type system.

(Dep ML, L3, ATS)

Different and unusual compared to Iris style \textemdash{} separation proofs outside the program.

Proof term in one sense, but also factors out operations for resource manipulation.

\subsection{Heaps}


\subsection{Type Safety}


\chapter{Kernel CN:\ Proof of soundness}%
\label{chap:kernel-soundness}

Weird heaps.

\chapter{Informing implementation discussions}\label{chap:inform-impl}

In the early stages, \kl{CN} was implemented by Christopher Pulte and Thomas
Sewell, based on sketches by Neel Krishnaswami. I started formalising
\kl{Kernel CN} much later, and benefited by the clarity of having an
implementations and implementers which and whom I could refer to in moments of
confusion.

However, this mode of development means that there were \emph{many} design
decisions made in a rather conservative context, because the programming was
always of a system which was being defined along the way, rather than a
well-understood pre-existing one. Extensions to syntax and inference were
always the minimum required for verifying the pKVM buddy allocator, lest
performance and inference suffer greatly, rather than ones based on a strong
formal and holistic consideration of the constructs and interactions at play.

As such, there are several restrictions in the implementation, which with the
benefit of hindsight and formalism, are completely unnecessary, but persist as
technical debt. This chapter list a few of these, and explains how the
formalisation brings much needed clarity to many questions around the
implementation.

\url{https://github.com/rems-project/cerberus/labels/language}
\url{https://github.com/rems-project/cerberus/labels/resource\%20reasoning}

\section{Supporting partially initialised reads of structs/unions}

This is not asked for, and actually seems to add a non-trivial amount of noise
and book-keeping to the formalisation. This suggests that the feature is not
worth implementing in CN unless a strong use-case comes up.

\section{Auto unfolding scheme for logical functions}
\url{https://github.com/rems-project/cerberus/issues/483}

\section{Higher-order resources}
\url{https://github.com/rems-project/cerberus/issues/483}

\section{Restrictions on branching}\label{sec:restriction-branching}
\url{https://github.com/rems-project/cerberus/issues/483}
\url{https://github.com/rems-project/cerberus/issues/266}

\section{Removing the pointer first restriction on predicates}\label{sec:rm-ptr-first}
\url{https://github.com/rems-project/cerberus/issues/303}

\section{Unifying the syntax of functions, predicates and specifications}
\url{https://github.com/rems-project/cerberus/issues/304}


\chapter{An alternative presentation}\label{chap:kernel-alternative}

Perhaps a short chapter about MiniCN\@? This could demonstrate the strong
advantages of defining a type system over a first-order functional language,
rather than trying to do so directly over something C-like.

It would also give some space to the interesting but yet-to-be-baked ideas
from the Fuliminate paper.



\include{formal-discussion.gen}

\pagelayout{wide} % No margins
\addpart{Memory Object Model}%
\label{part:mem-model}
\pagelayout{margin} % Restore margins

Overview of this part.

\chapter{Memory Object Models, explained}\label{chap:mem-model-explained}

Explain the connection to Core \textemdash{} not that strong, thanks to the memory
interface and the invariants of the VIP heaps, separation logic heaps, and
memory actions.

Factorising the formalisation pays off here.

\chapter{Pointers: more than you wanted to know}

Explain pointers in excrutiating detail, and why we need provenance for
optimisations.

Why do we care about provenance, why are pointers not just addresses

Common-subexpression elimination, copy-propogation, etc.

\section{Explaining PNVI-ae-udi and VIP}

Why you need provenance.

\section{Design Space}

Alternatives

\begin{figure}[h]
    \centering
    \includegraphics[width=\textwidth]{../misc/type-system-options.jpg}
\end{figure}


\section{Implementation}

Performance graph

\begin{figure}[h]
    \centering
    \includegraphics[width=\textwidth]{../misc/vip-performance-hit.png}
\end{figure}

\url{https://rems-project.githb.io/cerberus/dev/bench/}


\chapter{CN-VIP}\label{chap:cn-vip}

\margintoc{}

I already mentioned some of the typing rules related to memory actions and
pointer operations in \nameref{sec:kernel-mem-action-ops}, but I can now
recapitulate them with more detail, drawing special attention to parts about
liveness and bounds checks I skimmed past before. For convenience, I have
reproduced \cref{fig:typing-mem-action} in \cref{fig:cnvip-mem-action}.

\section{Memory actions}

\begin{figure*}[tp]
    \includegraphics{figures/kernel-mem-action-typing-1}
    \includegraphics{figures/kernel-mem-action-typing-2}
    \includegraphics{figures/kernel-memop-typing}
    \caption{\kl{Kernel CN} typing rules for memory actions, and a select rule
        for typing a memory operation.}\label{fig:cnvip-mem-action}
\end{figure*}

In \textsc{Expl\_Is\_Action\_Create}, creating an allocation produces new
constraints on its base address and size. These are tracked via constraints on a
logical variable $\mathit{alloc}_\mathrm{var}$, a map from allocation IDs to
pairs of a base address and size. This can simply be interpreted as an implicit
logical argument and return value for each function call.\sidenote{Need to
check the typing rules to ensure enforce this idea consistently.} Creating an
allocation also produces an \cninline{Alloc} token, to track the fact the
allocation is live, as well as the usual ownership/points-to resource.

TODO\@. Having a memory model also allows me to introduce support for dynamic
memory management. Whereas the above rule is only defined for typed objects
such as locals or globals, a region is essentially just an array of memory
bytes. For simplicity, I model every allocation as succeeding, so that the
pointer returned is never \cinline{NULL}. Hence the typing rule is similar,
but instead of ownership of a single object at a given C type, the ownership
is an iterated one over an array of memory bytes.

Conversely, in \textsc{Exp\_Is\_Action\_Kill}, destroying an allocations
requires both ownership of it (remember ownership represent read/write
permissions, but not allocation and freeing permissions), and the
\cninline{Alloc} token, plus proof that the given pointer is the same as the
owned pointer, and has the same base address as indicated by the allocation
token. For static kills (a block variable going out of scope), the rule also
checks that the size of the C type is also the size of the allocation.
TODO\@. Destroying a region is similar, except with an iterated ownership
of memory bytes.

Fortunately, the rules for loads and stores do not change at all. Ownership of
a location is enough to deduce that the allocation is live, and I assume
ownership is guaranteed to be in bounds for any allocation.\sidenote{Ownership
of out-of-bounds resources is equivalent to $\mathsf{false}$.}

\section{Pointer operations}

There are large discrepancies between the rules for pointer offsets presented
in \sidetextcite{lepigre2022vip} and \sidetextcite{memarian2022cerberus}, and
the \kl{Cerberus source code}, which I have detailed in a table in
\cref{fig:offset-confusion}. I am awaiting clarification on the correct way to
proceed.

\begin{figure*}[tpb]
  \begin{tabular}{ccccc}
  \toprule
   & \citeauthor{lepigre2022vip} incl.\ appendix & \citeauthor{memarian2022cerberus} & Cerberus code \\
  \midrule
  Member (P)
    & {\checksymbol✗}
    & case \cinline{NULL}
    & case \cinline{NULL}, 0-offset
  \\
  Member (ISO)
    & bounds, case 0-offset
    & bounds, liveness
    & case \cinline{NULL}, 0-offset
  \\
  Array (P)
    & {\checksymbol✗}
    & \textendash{}
    & \textendash{}
  \\
  Array (ISO)
    & bounds
    & bounds, liveness
    & bounds, liveness
  \\
  \bottomrule
  \end{tabular}
  \caption{Rules for computing pointer offsets (member and array, with
      pure/permissive (P) and ISO variants) in PNVI-ae-udi, across three
      different sources. `{\checksymbol✗}' means the rule is omitted. `case'
      means the rule special cases on that value. `\textendash{}' means
      there are no checks. `bounds' means a bounds check on the resulting
      pointer. `liveness' means a liveness check on the allocation.}\label{fig:offset-confusion}
\end{figure*}

Pointer operations such as taking the difference between two pointers or
relational comparison between two pointers, require both pointers to be in
bounds of the same live allocation. Hence the rule in\sidenote{TODO fix the
rule, which is very wrong in many ways} \cref{fig:cnvip-mem-action} ask
asks the solver to prove (a) the allocation IDs are equal (b) that the
pointer are within bounds of the allocation (c) to check there exists a
live allocation with that ID\@. The evidence of a live allocation can be
either ownership with the same allocation ID, or an \cninline{Alloc} token,
and the rule is agnostic as to which, just that this evidence is returned
in the type so as to not consume/destroy it.

A rule that is present in the code, but missing in other formats for
\kl{PNVI-ae-udi} is that for casting pointers to dead allocations to integers,
which is permitted so long as the address can fit within the target integer
type. Since \kl{VIP} does not track exposure, the live and the dead
pointer cases collapse into the same case. On top of this, as I mentioned in
\cref{subsec:prov-int-bytes}, I chose to support the limited provenance in
integers required via a new C type, to make the any additional complexity and
performance cost as opt-in, and to avoid changing all the base types for
integers of various sizes (bit vectors) to a datatype with two constructors.
Because of these two simplifications, the cast is therefore just an identity,
on the SMT term and its base type.\sidenote{TODO add this rule} The rules for
casting an integer to a pointer is \kl{UB} if it is not in the \cinline{NULL}
or round-trip case, and is an identity in the latter.\sidenote{TODO add
this rule} And the rule for \cinline{copy_alloc_id} performs a bounds check on
the integer using the allocation ID supplied by the pointer, and combines the
two into a new pointer.\cinline{TODO add this rule too}

The new C types need to be handled with care to implement the subtyping
required for a smoother experience.\sidenote{TODO figure out the subtyping
for base types and resources\ldots}

\section{\cinline{memcpy} and \cinline{memcmp}}

The typing rules for \cinline{memcpy} require iterated ownership of two
contiguous arrays of \kl{memory bytes} of length $n$; it returns ownership of
both, with the constraint that the value of the destination (first) is equal to
that of the source (second). The iterations must be contiguous and of the same
length to express the equality constraint on the values correctly.

The typing rules for \cinline{memcmp} require iterated ownership of two
contiguous arrays of \kl{memory bytes} of length $n$; like \cinline{memcpy}, it
returns ownership of both, unlike \cinline{memcpy} its resulting value is not
straightforward to specify, because the concise or obvious specification would
use quantifiers. I refer to a recursively defined logical function which
constrains the result to be (a) unconstrained if it reads any
\coreinline{unspec} values (b) 0 if all bytes (excluding provenances) are equal
and (c) the difference between the first two unequal bytes otherwise. The presence
of \coreinline{unspec} values makes it difficult to give a simpler specification
to the result such as \cninline[breaklines]{src == dest && result == 0i3 || src
!= dest && result != 0i32}, because we want to do not wish to imply % chktex 26
\cninline{unspec == unspec}.\sidenote{The simpler specification could be
achieved with a notion of \emph{comparable bytes}, converting to which would
require ownership of only initialised and non-padding bytes.}

Both of these typing rules require a way to get ownership of memory bytes, for
which, \kl{CN-VIP} adds new annotations.\sidenote{TODO add these typing rules}
In the formal presentation, these are represented by operations on predicates
which consume ownership of an object, and produce ownership of memory bytes, or
vice versa.

\section{Soundness}\label{sec:cn-vip-soundness}

There are few steps involved to updating the formalisation to use a \kl{VIP}
based memory object model from its current concrete one.
\begin{enumerate}
    \item Extend the configuration of the dynamic semantics to be a step
        relation between abstract \emph{states} and expression, rather than
        just \emph{heaps} and expressions.
    \item Extend the heap typing rules to incorporate the newly added
        allocation history.
    \item Update the proof of soundness for resource term reduction and pattern
        matching, with the new rules.
    \item Define an interpretation of heaps in ResCore, to heaps in the CN-VIP
        memory model.
    \item Prove that the ResCore model of heaps is sound with respect to
        \kl{PNVI-ae-udi}, perhaps via an intermediate concrete memory model.
\end{enumerate}

\subsection{Extending the dynamic semantics}

In the typing rules, I modelled the allocation history as a single global
logical variable $\mathit{alloc}_\mathrm{var}$. This means that even morally
closed programs have that variable free in explicit logical and resource terms.
At the same time, because the allocation history is extended during the course
of evaluating a \kl{ResCore} program, it is not a term which can be substituted
once at the start of the program and close. Hence, the allocation history must
be threaded through to any part of dynamic semantics which relies on checking
constraints (in the empty context) using the SMT solver. At the point of
calling, the allocation history is substituted in, with the most up to date
information, to check the constraint as a closed term (\cref{fig:mem-model-dyn-smt}).

\begin{marginfigure}
    \includegraphics{figures/mem-model-dyn-smt}
    \caption{Calls to the SMT solver are now extended to thread through the
        changing allocation history.}\label{fig:mem-model-dyn-smt}
\end{marginfigure}

Only the \coreinline{create} memory action extends the allocation history, and
so it and every grammar node containing it also includes the allocation history
as part of its configuration, rather than threaded through the
side.\sidenote{TODO fix premise 7 of the create dynamic rules.} Note that
because I split the intuitionistic part of the allocation history from the
linear part, it does not get updated to record a dead allocation in the
rule for \coreinline{kill} (\cref{fig:mem-model-dyn-create-kill}).

\begin{figure*}
    \includegraphics{figures/mem-model-dyn-create}
    \includegraphics{figures/mem-model-dyn-kill}
    \caption{The allocation history only tracks a mapping from IDs to a pair of
        base address and size, so when an allocation is killed, existing entries
        are not mutated.}\label{fig:mem-model-dyn-create-kill}
\end{figure*}

With the exception of threading through the allocation history, the rules for
loads and stores are unchanged. The rules for converting ownership of objects
into iterated ownership of memory bytes and vice versa are predicate
operations, much like the ones for manipulating structs and fixed-length
arrays.\sidenote{TODO add them} The rules for \coreinline{memcpy} and
\coreinline{memcmp} are also as expected.\sidenote{TODO add them}

Pointer operations do not extend the allocation history, but do require the
heap to check whether the supplied pointers belong to live allocations. They
are agnostic of whether it is ownership or an \cninline{Alloc} token is
provided as evidence (\cref{fig:mem-model-dyn-ptr-relop}).\sidenote{TODO
add/fix this}

\begin{figure*}
    \includegraphics{figures/mem-model-dyn-ptr-relop}
    \caption{Memory operations involving pointers perform a bounds check using
        the SMT solver and supplied pointers, and a liveness check based on
        evidence from the supplied resource term and the heap.}\label{fig:mem-model-dyn-ptr-relop}
\end{figure*}

Lastly, there are the rules about pointer to integer casts, integer to pointer
casts and the \cinline{copy_alloc_id} primitive. The latter two check that that
resulting pointer is in the bounds of a live allocation; they are also agnostic
as to whether it is ownership or an \cninline{Alloc} token which is provided as
evidence.\sidenote{TODO this too\ldots}

\subsection{State typing}

Because the abstract state now includes an append-only allocation history, the
typing rules for heaps (\cref{sec:heap-types}) needs to be generalised to
include it. The main judgement involved in this is $\mathit{alloc} \Leftarrow
\Phi$, which says that the allocation history $\mathit{alloc}$ is consistent
with constraint context $\Phi$. \cref{fig:alloc-typing} shows that it does so
by checking if each constraint, with the allocation history substituted for the
$\mathit{alloc}_\mathrm{var}$, holds (under the empty context).

\begin{marginfigure}
    \includegraphics{figures/alloc-typing}
    \caption{Definition of a well-constrained allocation history \textemdash{}
        $\mathit{alloc}$ is consistent with each constraint in context
        $\Phi$.}\label{fig:alloc-typing}
\end{marginfigure}

The heap typing rule generalises similarly (\cref{fig:heap2-typing}). It
substitutes the allocation history for $\mathit{alloc}_\mathrm{var}$, and then
types the heap exactly as before.

\begin{marginfigure}
    \includegraphics{figures/heap2-typing}
    \caption{Definition of heap typing in the presence of a allocation history:
        substitute the history into the heap and the type (\kl{normalised}
        resource context) and type as before.}\label{fig:heap2-typing}
\end{marginfigure}

\subsection{Updating the soundness proof}

Recall that I defined resource term reduction and pattern matching in the
dynamic semantics in a big-step style (\cref{sec:heap-types}). Whilst this
intertwines the proofs for progress and type preservation, its advantage of
modularity pays off now. The new constructs such as non-deterministic pointer
equality, \coreinline{allocate_region}, \coreinline{kill_dynamic},
\cinline{memcpy}, \cinline{memcmp}\cinline{copy_alloc_id}, and the conversions
to and from memory bytes, are just additional cases in the proof, the rest are
merely updates. The updates are small because the additional the additional
constraints on the allocation history are easy to link across the static and
dynamic semantics by the definition of allocation history typing. The bounds
and liveness checks are similarly easy to link across the static and dynamic
semantics by the definition of heap typing. The updated theorem statements are
as below: the main differences are the use of the updated abstract state typing
judgements, and the use of the $\mathcal{L}_0 = {\; \mathit{alloc}_\mathrm{var}
\;}$ environment, instead of the empty one, for logical variables. Updated
proofs are in the appendix.

\begin{theorem}[CN-VIP:\ progress and type preservation for resource terms]
For all resource terms ($[[ res\_term ]]$) closed which type check or synthesise
($[[ cdot ; L0 ; N ; nR |- res\_term <= res ]]$), and well-typed states
($[[ alloct ; h <= N ; nR ]]$), there exists a resource value ($[[ res\_val ]]$),
context ($[[ nR' ]]$) and heap ($[[ h' ]]$), such that: the value is well-typed
($[[ cdot ; L0 ; N ; nR' |- res\_val <= res ]]$); the heap is well-typed
($[[ alloct |- h' <= nR' ]]$), and for all frame-heaps ($[[ f ]]$), the resource term
reduces to the resource value without affecting the frame-heap
($[[ alloct | < h + f ; res\_term > ||v < h' + f ; res\_val > ]]$).
\end{theorem}

\begin{theorem}[Progress for the annotated and let-normalised Core]
If a top-level expression ($[[ texpr ]]$) is well-typed
($[[ cdot ; L0 ; N ; nR |- texpr <= ret ]]$) and all computational patterns
in it are exhaustive, then either it is a value ($[[ tval ]]$), or it is
unreachable, or for all well-typed states ($[[ s <= N ; nR ]]$)
then there exists another state ($[[ s' ]]$) and expression ($[[ texpr' ]]$)
which is stepped to ($[[ < s ; texpr > --> < s' ; texpr' > ]]$)
in the operational semantics.
\end{theorem}

\begin{theorem}[Type preservation for the annotated and let-normalised Core]
For all closed and well-typed top-level expressions
($[[ cdot ; L0 ; N ; nR |- texpr <= ret ]]$),
well typed states ($[[ alloct ; h <= N ; nR ]]$),
frame-heaps ($[[ f ]]$),
new states ($[[ alloct' ; h' ]]$),
and new top-level expressions ($[[ texpr' ]]$),
which are connected by a step in the operational semantics
($[[ < alloct ; h + f ; texpr > -->  < alloct' ; heap ; texpr' > ]]$),
if all top-level functions are annotated correctly,
there exists a constraint context ($[[ N' ]]$),
sub-heap ($[[ h' ]]$),
and resource context ($[[ nR' ]]$),
such that the constraint context is extended
($[[ cdot ; L0 ; N ; cdot \sqsubseteq cdot ; L0 ; N' ; cdot ]]$),
the frame is unaffected ($[[ heap ]] = [[ h' + f ]]$),
the sub-state is well-typed ($[[ alloct' ; h' <= N' ; nR' ]]$),
and the top-level expression too
($[[ cdot ; L0 ; N' ; nR' |- texpr' <= ret ]]$).
\end{theorem}

\chapter{Implementation of CN-VIP}

\margintoc{}

In addition to designing, formalising, and proving it sound, I also implemented
CN-VIP\@. This was a substantial project which I worked on for about seven
months, from August to October of 2023, and May, June, Septemeber and October
of 2024.\sidenote{See
    \href{https://github.com/search?q=repo\%3Arems-project\%2Fcerberus+author\%3Adc-mak\&type=commits\&s=committer-date\&o=asc}{my commits}
    to the \kl{Cerberus} repository. In the intervening months, I
    worked on a failed update to the buddy allocator of pKVM to work with
    bitvectors (\cref{chap:buddy}), engineering for accurate source location
    information (\cref{sec:error-msgs}), and MiniCN (\cref{chap:kernel-alternative}).}

What made it more challenging was that it had to be developed and integrated
piecemeal alongside other active \kl{CN} development.

The first step was adding in the various pieces of infrastructure in a non-functional way.
\begin{itemize}
    \item A logical variable for allocation history.
    \item A resource predicate for the \cninline{Alloc} token and using them in
        \coreinline{create} and in \coreinline{kill}.
    \item Additional constructs like \cinline{copy_alloc_id}.
    \item A datatype for the SMT representation of pointers, instead of integers.
    \item A flag to ensure changes to the pointer representation could be toggled on or off.
    \item Deprecating integer to pointer casts in the specification language.
    \item Adding array and member shifting operators.
\end{itemize}

Given all this, the next steps were about implementing support for bounds and
liveness checks, which was relatively straightforward.
\begin{itemize}
    \item Adapt and categorise the PNVI/VIP test suite to CN VIP\@.
    \item Support for non-deterministic pointer equality.
    \item Add a pointer liveness check.
    \item Add bounds checks, sometimes dependent on pointer liveness.
    \item Add basic support for \cinline{memcpy} (no provenance or
        \coreinline{unspec} values).
\end{itemize}

At this stage of development, I switched on CN-VIP by default, but retained the
ability to switch it off behind a flag. Fortunately, this happened just after
benchmarking on the CN tests was added, so we have a measurement of the impact
of the transition, shown in \cref{fig:vip-performance-hit}.

\begin{figure}[h]
    \centering
    \includegraphics[width=\textwidth]{../misc/vip-performance-hit.png}
    \caption{Sharp increase in execution time when enabling VIP, courtesy of
        \url{https://rems-project.github.io/cerberus/dev/bench/}.}\label{fig:vip-performance-hit}
\end{figure}

Aside from performance, at this stage, it also became clear that not supporting
provenance in bytes was unworkable \textemdash{} 19 out of 44 tests required
additional \cinline{copy_alloc_id} annotation to work as intended. Lack of
support for round-trip was also an issue, this affected 2 tests.

\begin{itemize}
    \item Support memory bytes.
    \item Support \cinline{memcpy}, \cinline{malloc} and \cinline{free}.
    \item Support comparable bytes for \cinline{memcmp}.
    \item Support round-trip casts.
\end{itemize}

\section{Definition of allocation history}

An allocation history is a map from allocation IDs (non-empty provenances) to a
record of a base address and a size.

This is reflected in the definition of the \mintinline{ocaml}{Alloc.History}
module, below. Symbols are unique identifiers used to resolved names
immediately after parsing, whereas identifiers are wrappers around strings for
things like names of record fields. I omit the definition of the helper function
\mintinline{ocaml}{make_value} for space. Because I separate the intuitionistic
and linear facts about the allocation history, this is all that is required to
declare it.

\ocamlfile{code/alloc_history.ml}

This is brought into the `empty' typing context, so that it is always in scope
for checking any function, which simply adds the symbol, its type, and some
location information (a built-in variable). Notably, there are no constraints
on it at the beginning.

\ocamlfile{code/empty_context.ml}

\section{Definition of \cninline{Alloc} token}

A definition of a predicate is a record of a location, a symbol for the first
pointer argument, a list of any other input arguments, a type for the output
argument, and an optional list of clauses (the body, potentially guarded by a
series of top-level ifs). The clauses represent the contents of the predicate,
if it can be unfolded (\cninline{Owned} and \cninline{Block} are built-in and
so cannot be unfolded).

\ocamlfile{code/definition_predicate.ml}

Using this, an \cninline{Alloc} token is defined simply as a predicate which
takes only the special pointer argument, no other input arguments, outputs the
record of base address and size, and has no clauses, i.e.\ cannot be unfolded.

\ocamlfile{code/definition_alloc.ml}

This is registered in an environment of definitions and declarations, in the
\ocamlinline{Global} module. Unlike the previous environment, this one does not
change during the course of type checking, but is populated on a per-file
basis. The symbol for \cninline{Alloc} tokens is mapped to the definition
mentioned earlier.

\ocamlfile{code/global_alloc.ml}

The Cerberus front-end supports an extension point, so that the
\cninline{Alloc} token, the \cninline{allocs} logical variables, and other
built-in symbols can be resolved correctly.

\section{Using \cninline{Alloc} in \coreinline{create} and \coreinline{kill}}

These constructs are introduced either by the user, or by a \coreinline{create}
action. Whilst in the formalisation the rules for memory actions are very
clearly synthesising, in the implementation they are more mixed, where terms
are synthesised, but base types are checked, and the contexts are changed along
the way.

The \coreinline{create} action takes as its arguments a pure expression for expressing
the alignment new allocation \ocamlinline{pe}, a C type \ocamlinline{act} and some
source location information \ocamlinline{prefix}. The latter is used only to generate
a helpful name for the logical variable representing the returned pointer.

First the base types are checked to line up, and then the alignment expression.
The continuation for checking the alignment expression names the result as
\ocamlinline{arg}, which is used to create the alignment value
\ocamlinline{align_v}. The return value \ocamlinline{ret} is defined, and a
fresh symbol is created \ocamlinline{ret_s} and added to a (unified) variable
context using \ocamlinline{add_a}. The function \ocamlinline{add_c} add the
constraint that the return value \ocamlinline{ret} is aligned to
\ocamlinline{align_v} to the constraint context. Similarly, \ocamlinline{add_r}
adds the uninitialised ownership resource to the resource context.

\ocamlfile[lastline=18]{code/check_create.ml}

After that, the VIP related code begins. To express bounds constraints on the
new allocation, a constraint that the value keyed by the pointer (actually its
provenance) in the allocation history will equal a record of the return address
and C type size. Note that this constraint is added under a flag
\ocamlinline{use_vip} (the `!' in OCaml is for reading a (mutable) reference,
not for negation). The allocation token is added immediately afterwards. The
typing context logs this action for error reporting, and then passes the
resulting value to an explicit continuation \ocamlinline{k}.

\ocamlfile[firstline=19]{code/check_create.ml}

\section{SMT representation of pointers}

Allocation IDs (non-empty provenances) have their SMT representation
switchable. If VIP is enabled, the representation is just an integer, otherwise
it is the empty tuple.

\ocamlfile[lastline=4]{code/solver_pointer.ml}

Pointers build on this switchable representation, so do not need a switch
themselves. Their SMT representation is a datatype named
\ocamlinline{"pointer"}, which is not polymorphic \ocamlinline{[]}, with two % chktex 18
constructors \ocamlinline{NULL} (which takes no arguments) and
\ocamlinline{AiA} for `allocation ID and address' which takes two arguments,
\ocamlinline{"alloc_id"} of type \ocamlinline{CN_Alloc_Id.t ()} and % chktex 18
\ocamlinline{"addr"} of type bit vector (of a width determined by the memory % chktex 18
interface).

\ocamlfile[firstline=20]{code/solver_pointer.ml}

\section{Array and member shifting}

\section{Adding \cinline{copy_alloc_id}}

\section{Adapting the PNVI/VIP test suite for CN}

\section{Non-deterministic pointer equality}

\section{Checking whether a pointer is live}

\section{Adding bounds checks on pointers}

\section{Basic support for \cinline{memcpy}}

\section{Insufficiency of not tracking provenance in bytes or integers}

\section{Memory bytes and their use in \coreinline{memcpy}, \coreinline{malloc} and \coreinline{free}}

\section{Comparable bytes in \coreinline{memcmp}}

\section{Integer with/out provenance union type for round-trip casts}

\section{Translating resource lemmas}\label{sec:trans-res-lemmas}

Is there a reason, in the discussion about Cassia's PhD, we thought all of
Cerberus needed to be shoved into Iris instead of just a trace of memory model
(and eventually, concurrency) events?

irisification (cf
\url{https://people.mpi-sws.org/~dreyer/papers/iris-ground-up/paper.pdf}).  Either
just the resource algebra or (as DM suggests) also the abstract language of
memory model interface events: which one let one formalise the primitive
resource-context manipulations that CN does (conceivably extractably).

I think a sufficient halfway point would be the language of traces memory
events i.e.\ memory model as the dynamics. Resource algebra would be CN's view
of resources. Resource lemmas are then just statements saying that one resource
represents exactly the same heap as another resource (skips). Changes to the
resource algebra because of memory events (which we could introduce unsoundness
in CN) would be proved sound in Iris.

As a bonus, we could even formalise and prove sounds the inference procedures
CN uses (and with some engineering to handle SMT, even extract from Rocq).

The inference I'm talking about is resource context manipulations: checking if
we can pack or unpack predicates, if an owned is in the context, shifting
indices in and out of iterated predicates, exploding and imploding structs.
These operations don't require core structure.

Even if full extraction of the inference algo is not feasible (it would
involves standard data structures + SMT FFI), having a defined set of resource
manipulation primitives proved sound and extracting those (just standard data
structure manipulations), or even just proving the primitives sound and using a
similar interface would increase confidence.

If ones reads the above in reverse, it even provides a gradual migration path
which doesn't commit us to any next step and allows us to see how far
extraction can take us.

If the inference algs or the primitives are formalised, then we can iterate on
cleverer  inference schemes with a strong safety net

I think it will become more valuable as soon as we start having fancier things
like higher-order resources, locks, fractional permissions. At that point,
checking the steps/moves that any inference algorithm could take would get
closer to essential.

Even if an arbitrary inference algorithm is not stable, the steps available and
the shape of the resources should be more so, and that is worth at least
creating a clean abstraction for (and then pen-paper soundness, and then
mechanised soundness).

Noted and agreed that anything mechanised takes longer than one wants/expects
and that extraction is a pain. But this gives us a concrete use-case,
reasonable sequence of experiments and a clear idea of the benefits and costs.



\pagelayout{wide} % No margins
\addpart{Engineering}%
\label{part:engineering}
\pagelayout{margin} % Restore margins

Overview of engineering part.

\chapter{Tree-carving: Taming C Projects}\label{chap:tree-carver}

\begin{definition}[\intro{Observational refinement}]\label{def:obsref}
  A term $\Gamma \vdash t \ty A$ \kl{observationally refines} a term
  $\Gamma \vdash u \ty A$, noted $\intro* t \obsRef u$, if for all boolean-valued observation context
  $\mathcal{C} \ty (\Gamma \vdash A) \Rightarrow (\vdash \Bool)$ closing over all
  free variables, if $\mathcal{C}[u] \red \err[\Bool]$ or diverges,
  then either $\mathcal{C}[t] \red \err[\Bool]$ or $\mathcal{C}[t]$ diverges.
\end{definition}

\chapter{User experience}

\section{Syntax, syntax, syntax}

\section{How calling conventions affect specifications}

\section{Counter examples}\label{sec:counter-ex}

\begin{itemize}
    \item Not minimal
    \item Not consistent
    \item Not easy to relate to source
\end{itemize}

\section{The unreasonable effectiveness of good error messages}\label{sec:error-msgs}

\begin{itemize}
    \item Translate standards jargon into C programmer friendly words.
\end{itemize}


\chapter{\kl{CN} Comparsion and Feedback}

\margintoc{}

In this chapter, I will discuss the reality and expectations of how usable a
verification tool for C \emph{can} be at this stage. I will start with my brief
and informal comparison between \kl{CN} and similar tools, based on my
experience of verifying \kl{pKVM}'s simpler \emph{early} allocator, in \kl{CN},
\kl{VeriFast}~\sidecite{jacobs2011verifast} and
\kl{Frama-C}.\sidecite{baudin2021dogged}\sidenote{\url{https://github.com/rems-project/CN-pKVM-early-allocator-case-study}}
I will also discuss a \kl{RefinedC} verification of the same, which was
completed by the developers~\sidecite{sammler2021refinedc}. Part of this
comparison was featured in~\sidetextcite{pulte2023cn}, including the annotation
overhead and execution times.

After that, I will discuss some feedback given by some industry partners on
\kl{CN}. As an academic, these seem borderline unreasonable, especially when
compared to the aforementioned state-of-the-art research tools, but as an
ex-industry professional, it is much easier to concede. Given our goal of
targeting kernel hackers, we as academics want industry adoption for these
tools, in which case taking industry feedback seriously is crucial.

\section{Comparison}

The early allocator in \kl{pKVM} during intialisation, before switching over to
the aforementioned buddy allocator. It is a simple allocator which just bumps a
pointer, with no support for reclaiming memory. It features functions to
initialise, get the number of pages allocated, and allocate new zeroed pages.

\subsection{Early allocator in CN}

The proof for the early allocator was not part of CI for \kl{CN}, and so is
considerably out-of-date.\sidenote{TODO Update this, add to CI, check how much
slower it is with bit vectors.} Nevertheless, I will show and discuss a sample
of the code and specifications, because because I expect updated specifications
to look recognisably similar.

The first function is one to zero all the bytes in a page. This is implemented
in assembly in \kl{pKVM}, but for the purposes of comparison, I implemented a
version in C. The precondition specifies that the function expects an array of
\cninline{Bytes}, (not to be confused with memory bytes from
\cref{sec:mem-bytes-use}), which is defined as a wrapper around
\cninline{Owned<char>}. The postcondition specifies the function returns an
array of bytes with value zero, expressed this way to avoid the use of the byte
array equality lemma mentioned in \cref{fig;byte-array-eq-lemma}.

The loop invariant states that ownership moves from the former to the latter at
the loop index. This version of \kl{CN} required manually folding and unfolding
predicates, but inferred indices, and so the body of the loop features
annotations to do the former.

\cfile[fontsize=\footnotesize,breaklines,lastline=18]{code/cn_early_alloc.c}

The second version is to allocate a zeroed page. It accesses two global
variables, \cinline{cur} which marks the current pointer of the allocator, and
\cinline{end} which marks the exclusive end of the allocator. It requires there
is at least a page size difference between the two, and of ownership of the
bytes from \cinline{cur} to \cinline{end}, defined in the \cninline{EarlyAlloc}
predicate. It ensures that it retains ownership of the same predicate, but with
new bounds \textemdash{} \cinline{cur} in the postcondition used to mean the
its value at the \emph{end} of the function, using \cninline|{cur}@start| to
refer to the value at the start of the function. And it ensures that the
returned value is the base address of an array of bytes. The body of the
function simply unfolds and folds the \cninline{EarlyAlloc} predicate,
incrementing the pointer by the page size in between.

\cfile[fontsize=\footnotesize,breaklines,firstline=20]{code/cn_early_alloc.c}

\subsection{Early allocator in VeriFast}

The verification of the early allocator in \kl{VeriFast} is similar to the
\kl{CN} one. The main difference is that since \kl{VeriFast} does not support
iterated separating conjunction, it may require lemmas to manipulate the
inductive predicates involved instead. \kl{VeriFast} supports non-precise
assertions (\cref{subsec:precise-assertion}) as well, but does not support
ordinary disjunction for impure assertions for the same reason \kl{CN} does
not: to avoid backtracking in symbolic
execution.\sidenote{\url{https://verifast.github.io/verifast-docs/faq.html\#how-to-express-a-disjunction-p-or-q-in-an-assertion}}

In this example, the \cninline{characters_zeroed} predicate is recursively
defined; and is marked as precise. This allows \kl{VeriFast} to unfold
(\cninline{open}) and fold (\cninline{close}) the predicate automatically. The
loop invariant is expressed in Tuerk-style,\sidecite{tuerk2010local} but
regular loop-invariants are supported too.

\cfile[fontsize=\footnotesize,breaklines,lastline=13]{code/verifast_early_alloc.c}

If \kl{VeriFast} does not support iterated separating conjunctions, then how
does it support the indexing in this example without lemmas? The answer lies in
a surprising difference between how \kl{VeriFast} handles the semantically
equivalent subscripting \cinline{e1[e2]} and \cinline{*(e1 + e2)}.\sidenote{\url{https://github.com/verifast/verifast/issues/259}} % chktex 36

\begin{quote}
    \emph{Indeed, \kl{VeriFast} symbolically evaluates \cinline{*(start + i)} and % chktex 36
    \cinline{start[i]} differently. \cinline{*(start + i)} is symbolically % chktex 36
    evaluated just like any other dereference of a pointer to int, whereas
    evaluation of \cinline{start[i]} first looks for an
    \cninline{ints(start, ?n, ?vs)} chunk where \cninline{i <= n} and, if it % chktex 26 chktex 36
    finds one, returns \cninline{nth(i, vs)}. This is convenient for ``random % chktex 36
    access'' to ints-encoded arrays. If it does not find such a chunk, it falls
    back to looking for an \cninline{integer(start + i, _)} chunk, but the % chktex 36
    \cninline{start + i} computation is indeed not checked for overflow. This
    seems sound because if the integer chunk exists, it implies that the
    address is within the limits.}
\end{quote}

%This could end up being a useful heuristic.

This particular example was also a good lesson in how phrasing assertions
differently can lead to drastically different performance outcomes. Whilst the
final version uses the \cninline{character} predicate (which is a points-to at
character type), initially I used a more general \cninline{chars} predicate
that generalised \cninline{character} relating a current and end pointer to a
\emph{list} of characters. \cninline{create(count, item)} is a fix-point % chktex 36
function that I defined, which creates a list of only \cninline{item} of length
\cninline{count}.

\cfile[fontsize=\footnotesize,breaklines,lastline=2]{code/verifast_alt_loop.c}

\kl{VeriFast} generally has good support for automation, and the ability to
state and prove lemmas (pure and resourceful) from within the system itself.
For pure lemmas, \cninline{lemma_auto}, marks it as available for use by
automation all the time, whereas \cinline{lemma_auto (expr)} automatically
applies the lemma when a precise predicate \cninline{expr} is in the context.

We need a lemma because the loop exit condition \cninline{i == 4096} means that
\cninline{length(unzeroed) == 0} and so \cninline{unzeroed == nil == create(0,0)}, % chktex 36
and this is too much for automation to deduce. Because of this, it was not clear
to me where to place an annotation to manually instantiate the lemma. At the
same time, I was not adept enough with triggers to figure out how to fire them
exactly when needed. Marking it as an automatic lemma without a trigger worked,
but it slowed down verification considerably: usable in batch mode, a not usable
interactively. There is no fallback to external assistants.

\cfile[fontsize=\footnotesize,breaklines,firstline=3]{code/verifast_alt_loop.c}

Like \kl{CN}, \kl{VeriFast} also requires annotations to read (and write)
global variables. The rest of the specification is very similar to the \kl{CN}
one; the \cninline{earlyAlloc} predicate could be marked as precise, this would
have removed the need for the unfolding and folding statements.

\cfile[fontsize=\footnotesize,breaklines,firstline=15]{code/verifast_early_alloc.c}

\kl{VeriFast} has a similar (but slightly smaller) annotation overhead to
\kl{CN}\@. Though I did not use them, fractional permissions are also
supported. As its names implies, it is indeed very fast, about 10 times faster
than \kl{CN} in this case (50 milliseconds vs 500 milliseconds). It has a
useful graphical user-interface which provides syntax highlighting for
specifications, excellent visibility into the proof state, and highlights the
annotation it cannot prove directly in the source code. Not only that, it
supports replaying the steps of the proof leading up to that state.

There are some aspect of \kl{VeriFast} which I do not understand, and cannot
find any documentation or explanation for, namely its handling of structs.
First off, it provides no in built predicates to do the equivalent of claiming
ownership of a whole struct, so users have to write such predicates themselves,
using auto-generated predicates for each members. With this one can write the
following. The semicolon is to mark the predicate as precise, and to separate
input arguments from output ones. Confusingly, it fails, with an error `no
matching heap chunks s2\_x(..)'. % chktex 36

\cfile[fontsize=\footnotesize,breaklines,lastline=15]{code/verifast_structs.c}

However, changing the predicate definition to explicitly mention the fields in
the \cinline{inner} struct allows it to pass.
\cfile[fontsize=\footnotesize,breaklines,firstline=17,lastline=18]{code/verifast_structs.c}

This strikes me as unusual because any changes to the representation of
\cinline{struct s2} are no longer encapsulated fields of that type. In \kl{CN},
the following is true (modulo padding),
\cninline[breaklines]|s1_inner(p, inner) <=> s2_x(&p->inner, inner.x)|,% chktex 36
\sidenote{`s1\_inner' and `s2\_x' are auto-generated predicates, expressing
points-to facts for struct fields.} but in \kl{VeriFast} this does not seem
to be the case.

Furthermore, the specification itself is weak, because it does not connect the
values in the inner struct to the outer struct of which it is a field. Yet,
adjusting the specification to link the output value \cninline|?val| of the
field to the output value of its member, causes the verification to fail again,
this time saying that it cannot prove \cninline|in.x == 1|.

\cfile[fontsize=\footnotesize,breaklines,firstline=19,lastline=20]{code/verifast_structs.c}

To be clear: I know I am at fault, and specifying things incorrectly. I am
quite confident such an example can be verified in \kl{VeriFast}. My point is
to draw attention to the (a) unexpected properties of nested structs and (b)
the difficulty of understanding any modes/dataflow implicit in the syntax. The
issue with structs may be an intentional design choice to sidestep handling
exploding and imploding structs, which does require extra inference and thus
reduce performance.

Stepping back, \kl{VeriFast} uses an ad-hoc semantics of C the developers
believe to be sound. During this brief experiment, I noted a few missing
features.
\begin{itemize}
    \item Struct literals/initialisers.
    \item Implicit type promotions (e.g.\ from \cinline{long long} to
        \cinline{unsigned long long}).
    \item Function types in struct fields.
    \item Taking the address of local variables.
    \item Expressive/accurate pointer provenance (has basic support).
    \item \kl{UB} or unspecified values for uninitialised reads.
\end{itemize}

\subsection{Early allocator in Frama-C and RefinedC}

Whereas \kl{VeriFast} is most similar to \kl{CN}, \kl{Frama-C} and
\kl{RefinedC} are one step removed. Both use undecidable logics, falling back
to \kl{Rocq} for manual proof. For \kl{Frama-C}, the logic is a Hoare logic,
not separation, whereas in \kl{RefinedC}, the logic is an \kl{Iris} instance.

Frama-C has a smaller annotation overhead, though its lack of support for
separation logic means the end guarantee is weaker than with the other tools.
It works by translating C programs into CIL~\sidecite{necula2002cil}. Its Hoare
logic verifier, WP, uses a custom semantics CIL, which is parametric in the
memory model, allowing the user to select trade-off increased performance for
pointer manipulation expressiveness. Its includes support for multiple
specifications per function, ghost parameters, arguments and code, and runtime
assertion checks, which amongst other things, can be used to error on
uninitialised reads. \kl{Frama-C} was noticeably slower than \kl{CN} for the
early allocator (3.5s).

\kl{RefinedC} is implemented inside \kl{Rocq} atop \kl{Iris}, making its
\kl[TCB]{trusted computing base} (TCB) the smallest out of all the tools
mentioned so far. This trust is undercut by its custom ad hoc semantics for C
based on the \emph{Caesium} kernel over which its typing and automation
framework \emph{Lithium} operates. The rules are intended to be heuristic
rather than decidable, with the main mode of operation inside a \kl{Rocq}
session. Its annotation overhead is similar to \kl{CN}, but the performance is
much slower (16.7s).

\section{Industry feedback}

Lorem ipsum dolor sit amet,\sidenote{Under review from industry, hence
placeholder for formatting and word count.} consectetuer adipiscing elit.
Aenean commodo ligula eget dolor. Aenean massa. Cum sociis natoque penatibus et
magnis dis parturient montes, nascetur ridiculus mus. Donec quam felis,
ultricies nec, pellentesque eu, pretium quis, sem. Nulla consequat massa quis
enim. Donec pede justo, fringilla vel, aliquet nec, vulputate eget, arcu. In
enim justo, rhoncus ut, imperdiet a, venenatis vitae, justo. Nullam dictum
felis eu pede mollis pretium. Integer tincidunt. Cras dapibus. Vivamus
elementum semper nisi. Aenean vulputate eleifend tellus.

Aenean leo ligula, porttitor eu, consequat vitae, eleifend ac, enim. Aliquam
lorem ante, dapibus in, viverra quis, feugiat a, tellus. Phasellus viverra
nulla ut metus varius laoreet. Quisque rutrum. Aenean imperdiet. Etiam
ultricies nisi vel augue. Curabitur ullamcorper ultricies nisi. Nam eget dui.
Etiam rhoncus. Maecenas tempus, tellus eget condimentum rhoncus, sem quam
semper libero, sit amet adipiscing sem neque sed ipsum. Nam quam nunc, blandit
vel, luctus pulvinar, hendrerit id, lorem. Maecenas nec odio et ante tincidunt
tempus.

Donec vitae sapien ut libero venenatis faucibus. Nullam quis ante. Etiam sit
amet orci eget eros faucibus tincidunt. Duis leo. Sed fringilla mauris sit amet
nibh. Donec sodales sagittis magna. Sed consequat, leo eget bibendum sodales,
augue velit cursus nunc, quis gravida magna mi a libero. Fusce vulputate
eleifend sapien. Vestibulum purus quam, scelerisque ut, mollis sed, nonummy id,
metus. Nullam accumsan lorem in dui. Cras ultricies mi eu turpis hendrerit
fringilla.

Vestibulum ante ipsum primis in faucibus orci luctus et ultrices posuere
cubilia Curae; In ac dui quis mi consectetuer lacinia. Nam pretium turpis et
arcu. Duis arcu tortor, suscipit eget, imperdiet nec, imperdiet iaculis, ipsum.
Sed aliquam ultrices mauris. Integer ante arcu, accumsan a, consectetuer eget,
posuere ut, mauris. Praesent adipiscing. Phasellus ullamcorper ipsum rutrum
nunc. Nunc nonummy metus. Vestibulum volutpat pretium libero. Cras id dui.
Aenean ut eros et nisl sagittis vestibulum. Nullam nulla eros, ultricies sit
amet, nonummy id, imperdiet feugiat, pede. Sed lectus.

Donec mollis hendrerit risus. Phasellus nec sem in justo pellentesque
facilisis. Etiam imperdiet imperdiet orci. Nunc nec neque. Phasellus leo dolor,
tempus non, auctor et, hendrerit quis, nisi. Curabitur ligula sapien, tincidunt
non, euismod vitae, posuere imperdiet, leo. Maecenas malesuada. Praesent congue
erat at massa. Sed cursus turpis vitae tortor. Donec posuere vulputate arcu.
Phasellus accumsan cursus velit. Vestibulum ante ipsum primis in faucibus orci
luctus et ultrices posuere cubilia Curae; Sed aliquam, nisi quis porttitor
congue, elit erat euismod orci, ac placerat dolor lectus quis orci.

Phasellus consectetuer vestibulum elit. Aenean tellus metus, bibendum sed,
posuere ac, mattis non, nunc. Vestibulum fringilla pede sit amet augue. In
turpis. Pellentesque posuere. Praesent turpis. Aenean posuere, tortor sed
cursus feugiat, nunc augue blandit nunc, eu sollicitudin urna dolor sagittis
lacus. Donec elit libero, sodales nec, volutpat a, suscipit non, turpis. Nullam
sagittis. Suspendisse pulvinar, augue ac venenatis condimentum, sem libero
volutpat nibh, nec pellentesque velit pede quis nunc.

Vestibulum ante ipsum primis in faucibus orci luctus et ultrices posuere
cubilia Curae; Fusce id purus. Ut varius tincidunt libero. Phasellus dolor.
Maecenas vestibulum mollis diam. Pellentesque ut neque. Pellentesque habitant
morbi tristique senectus et netus et malesuada fames ac turpis egestas. In dui
magna, posuere eget, vestibulum et, tempor auctor, justo. In ac felis quis
tortor malesuada pretium. Pellentesque auctor neque nec urna. Proin sapien
ipsum, porta a, auctor quis, euismod ut, mi. Aenean viverra rhoncus pede.

Pellentesque habitant morbi tristique senectus et netus et malesuada fames ac
turpis egestas. Ut non enim eleifend felis pretium feugiat. Vivamus quis mi.
Phasellus a est. Phasellus magna. In hac habitasse platea dictumst. Curabitur
at lacus ac velit ornare lobortis. Curabitur a felis in nunc fringilla
tristique. Morbi mattis ullamcorper velit. Phasellus gravida semper nisi.
Nullam vel sem. Pellentesque libero tortor, tincidunt et, tincidunt eget,
semper nec, quam. Sed hendrerit. Morbi ac felis. Nunc egestas, augue at
pellentesque laoreet, felis eros vehicula leo, at malesuada velit leo quis
pede.

Donec interdum, metus et hendrerit aliquet, dolor diam sagittis ligula, eget
egestas libero turpis vel mi. Nunc nulla. Fusce risus nisl, viverra et, tempor
et, pretium in, sapien. Donec venenatis vulputate lorem. Morbi nec metus.
Phasellus blandit leo ut odio. Maecenas ullamcorper, dui et placerat feugiat,
eros pede varius nisi, condimentum viverra felis nunc et lorem. Sed magna
purus, fermentum eu, tincidunt eu, varius ut, felis. In auctor lobortis lacus.
Quisque libero metus, condimentum nec, tempor a, commodo mollis, magna.
Vestibulum ullamcorper mauris at ligula. Fusce fermentum.

Nullam cursus lacinia erat. Praesent blandit laoreet nibh. Fusce convallis
metus id felis luctus adipiscing. Pellentesque egestas, neque sit amet
convallis pulvinar, justo nulla eleifend augue, ac auctor orci leo non est.
Quisque id mi. Ut tincidunt tincidunt erat. Etiam feugiat lorem non metus.
Vestibulum dapibus nunc ac augue. Curabitur vestibulum aliquam leo. Praesent
egestas neque eu enim. In hac habitasse platea dictumst. Fusce a quam. Etiam ut
purus mattis mauris sodales aliquam. Curabitur nisi. Quisque malesuada placerat
nisl. Nam ipsum risus, rutrum vitae, vestibulum eu, molestie vel, lacus.

Sed augue ipsum, egestas nec, vestibulum et, malesuada adipiscing, dui.
Vestibulum facilisis, purus nec pulvinar iaculis, ligula mi congue nunc, vitae
euismod ligula urna in dolor. Mauris sollicitudin fermentum libero. Praesent
nonummy mi in odio. Nunc interdum lacus sit amet orci. Vestibulum rutrum, mi
nec elementum vehicula, eros quam gravida nisl, id fringilla neque ante vel mi.
Morbi mollis tellus ac sapien. Phasellus volutpat, metus eget egestas mollis,/
lacus lacus blandit dui, id egestas quam mauris ut lacus. Fusce vel dui. Sed in
libero ut nibh placerat accumsan. Proin faucibus arcu quis ante. In
consectetuer turpis ut velit. Nulla sit amet est. Praesent metus tellus,
elementum eu, semper a, adipiscing nec, purus. Cras risus ipsum, faucibus ut,
ullamcorper id, varius ac, leo. Suspendisse feugiat. Suspendisse enim turpis,
dictum sed, iaculis a, condimentum nec, nisi. Praesent nec nisl a purus blandit
viverra. Praesent ac massa at ligula laoreet iaculis. Nulla neque dolor,
sagittis eget, iaculis quis, molestie non, velit. Mauris turpis nunc, blandit

\section{Summary}

Here, I summarise the many (incompatible, or at the very least, impractical)
directions \kl{CN} could pursue based on the comparison with other tools
and the industry feedback. I leave evaluating and synthesising these points
to the next chapter.
\begin{itemize}
    \item Improve performance by 10x.
    \item Trace type checking steps for replay.
    \item Improve syntax concision and approachability.
    \item Improve error messages.
    \item Infer/suggest specifications.
    \item Reduce the \kl{TCB}.
    \item Support using \kl{CN} `out-of-the-box'.
    \item Focus on industry-favoured use cases, such as mathematics,
        input-validation, memory safety beyond current tools, and concurrency.
\end{itemize}

\chapter{Not-so-great expectations}

In the last chapter, I compared \kl{CN} to some relevant tools in the
space,\sidenote{A notable omission was \kl{VST}, mostly due to lack of time and
expertise for a fair comparison.} and I presented industry feedback on \kl{CN}.
Based on my experience of having worked on programming language tooling both in
industry and in academia, I will synthesise the two and lay out what I believe
are achievable standards of usability for \kl{CN}, with an emphasis on how new
projects can avoid several pitfalls.

\section{Work backwards from examples}

\kl{CN}'s guiding star during its development was the \kl{pKVM buddy
allocator}. Whilst this was understandable, and indeed its unique selling
point, this \emph{also} guided its design to a large degree. Features were only
considered and implemented insofar as they got \kl{CN} closer to verifying the
entire allocator.

Whilst this produced a successful paper, this over-fitted \kl{CN} to the example.
More concretely, it meant that we had accrued significant technical and design
debt, which became more obvious by the time it came to specifying and verifying
more pedestrian examples, such as those now in the \kl{CN
tutorial}.~\sidecite{pulte2024tutorial}

Many of the features mentioned in \cref{chap:inform-impl} might have been
incorporated earlier. An obvious candidate is unifying the syntax for
functions, predicates and specifications: these features were built separately
over time and as a result have ended up similar but
incompatible.\sidenote{\href{https://github.com/rems-project/cerberus/issues/288}{Cerberus
\#288}} For example, one of the many incompatibilities is the separate scopes
for predicates and functions: this results in a confusing error message when
the predicate and a function are mixed up. However, unifying the scopes at
point would require a major refactor to symbol resolution during
desugaring.\sidenote{\href{https://github.com/rems-project/cerberus/issues/288}{Cerberus
\#288}} Had we worked through enough examples and designs concretely, we might
have noticed the emerging similarities and reworked the approach.

With only slight irony, my point is that \emph{test-driven development is
exceptionally useful when implementing a verification tool}. This bore out in
my personal experience in implementing \kl{VIP}: writing examples and
informally but rigorously working through how they ought to be specified and
helped enormously in guiding my intuitions and highlighting problems early on,
and I conjecture a broader range of smaller examples escalating to the
\kl{buddy allocator} might have done similar.

\section{Design with a formalism}

Whilst there are some changes that I think would have been difficult to
pre-empt by more careful design at the start, for example, the change in
inference
scheme,\sidenote{\href{https://github.com/rems-project/cerberus/commit/7c2c0a364a4373e4eb109f32d01cc9584f51e81f}{Commit
7c2c0a36.} This was motivated by reducing the annotation burden for folding
and unfolding predicates, which turned out to be heavier than annotation
saved by inferring indices.}, or the switch to the monadic syntax
(\cref{sec:monadic-syntax}), there are few design decisions I conjecture would
would have benefited from scrutiny prior to implementation.

\subsection{Calling conventions affect syntax}

By default, \kl{Cerberus} elaborates calls to C functions by allocating and
initialising parameters \emph{at the call site} and then \emph{passing
pointers} to the temporary objects to the callee. This leads to very verbose
and awkward specifications, where functions had to (a) claim ownership of their
parameters and (b) promise they did not modify them so that the objects at the
call-site remained unaffected.

This eventually became intolerable enough to realise that what needed to change
was not the specifications but the
elaboration.\sidenote{\href{https://github.com/rems-project/cerberus/commit/772173d6432c86b029fd1bb993b8dc83b80c96c0}{Cerberus
772173d6}} Of course, this required changes to \kl{CN}
too,\sidenote{\href{https://github.com/rems-project/cerberus/commit/186e4e42a75cad2222d441ce608fc1dc84cc7b98}{Cerberus
186e4e42}} to handle the changed elaboration. Ultimately, this resulted in
duplicated effort. Had we spent some time looking at small sample core
programs, and testing out a formalism on it, we might have anticipated the
issue sooner.


\subsection{Error reporting}

Most of the time, \emph{any} programming language tool will be fed an incorrect
or incomplete program. The standard approach to dealing with this is to layer
the program's stages such that only well-formed programs pass from one stage to
the next. Within the context of \kl{CN}, the main stages are:
\begin{itemize}
    \item Parsing
    \item Desugaring
    \item Elaboration of C into \kl{Core}
    % \item Translation of \kl{Core} to $\mu$\kl{Core}
    \item Base type/well-formed specification checks
    \item Per-function constraint and resource inference and checking.
\end{itemize}

Such a strategy is understandable, indeed it makes the system more robust
because each phase assumes the invariants ensured by the previous, and
establishes new ones. It makes the implementer's job easier, but as important
as that is, it is quite likely the cumulative user time will dominate
cumulative implementer time in the long run: code is written once, read many
times, and used orders of magnitude more.

The problem is that users (rightfully) expect to be able to get feedback on
their program at multiple stages of development, and getting one error at a
time slows down the edit-check cycle. Of course, multiple errors can also
overwhelm the user, but it is easier to limit them once they are already
supported, than it is to support them if they do not exist.

In the ideal case, users should be able to execute a whole pipeline of stages
even in the presence of errors from the first; it would be very impressive if
\kl{CN} could do resource inference in the presence of parse errors
\emph{within the function it was checking}, or even just in other parts of the
program.

This is likely unrealistic for \kl{CN}, because of the amount of re-engineering
it would require in the \kl{Cerberus} front-end. A realistic goal for \kl{CN}
is to raise the lower bar of recovery, by designing each step such that:
\emph{it avoids stopping at the first error}. For example, up until relatively
recently, \kl{CN} stopped at the first error in the first failing function,
\emph{even though it was designed to support per-function checking}.

It would be ironic, if, in advocating support for multiple errors, I insisted
that the best way to achieve this was a full rewrite of everything starting
from the parser. And so, below I conjecture a process which I believe can be
applied in an incremental way to existing projects, which allows developers
to reap the benefits of multiple errors in proportion to the amount of
time they are willing to spend on it.
\begin{itemize}
    \item Pick the most important stage.
    \item For every (or each important) sort, a \emph{hole}
        construct.~\sidecite{omar2017hazelnut}
    \item Update the rest of the stage to handle this hole.
    \item If the subsequent stage supports holes, consider joining them up.
\end{itemize}
If, like in \kl{CN} proof mode, the last (semantic analysis) stage backwards is
the most interesting and expensive one, this suggests working backwards through
each stage.

To be clear, I am not talking about \emph{incrementalising} the entire type
checker, though support for holes could make this easier.

What I am saying is that for every \kl{CN} annotation, there is the opportunity
to support multiple errors through multiple stages. Take the example of failed
the bit vector update for the buddy allocator (\cref{sec:buddy-failed-bv}). One
of the major pain points of this update was (tediously) ensuring that
\emph{every datatype, function, predicate, and pre/postcondition} was updated
to switch from integer to bit vector types; without this, \kl{CN} would refuse
to move on to checking any functions. As a result, I was forced to manually
figure out the dependency structure, start at the leaves, and comment out
everything else that was irrelevant. This also made it difficult to
incrementally check my progress into a CI pipeline. Though gating the switch to
integers behind a flag would have helped, I believe multiple error propagation
would have been helpful too, \emph{especially when type checking is slow}.

I conjecture inspiration from the below two papers will prove fruitful
to investigating and guiding the theory and implementation of multiple
errors for \kl{CN}.
\begin{itemize}
    \item \citetitle{zhao2024total}~\sidecite{zhao2024total}.
        This paper defines a bidirectional gradual
        type system for types and expressions (including holes), and a total
        procedure to mark expressions with \emph{all possible} type errors. It
        then adds constraint solving on top of the type holes, so that
        inconsistent type constraints are localised \emph{exclusively to holes
        and marked types and expressions}, by tracking the program origins of
        unknown types. The relevance of this paper is its principled approach
        to all of the above.
    \item \citetitle{spies2024quiver}~\sidecite{spies2024quiver}.
        This paper also uses a two way flow of
        information familiar from bidirectional type checking, but describes it
        using terms more standard in separation logic circles: \emph{abductive
        deductive verification}. It symbolically evaluates a program using a
        context of separation logic assertions, proving everything it can and
        assuming everything it cannot. The specifications (`predicate
        transformers') they infer are analogous to \kl{CN}'s function types.
        The relevance of this paper is its similar domain to \kl{CN}.
\end{itemize}
Though designing such calculi in this way may seem like it increases the amount
of work involved for reporting errors, in actuality, it simply makes
implementation-related error reporting \emph{explicit in the formalism}.

\subsection{Elaborate with care}

Elaboration and large-scale program transformations such as A-normalisation can
affect the accuracy of source locations. On top of this, if handled without
care, generated variables can also end up not associated to any source location
the user wrote, and can be difficult to interpret in a counter-example. \kl{CN}
used to A-normalise, but does not any
more,\sidenote{\href{https://github.com/rems-project/cerberus/commit/21808139bda2ee320756c71eb22dbd57d0986f97}{Commit
21808139}.} because it was difficult to relate such variables to \kl{Core}
expressions. The situation has improved a bit with more helpful auto-generated
names and no A-normalising, but the solution is ad hoc, and the gap between
\kl{Core} and C still remains.

Whilst I have some ideas on how to improve the situation,\sidenote{Low-hanging
fruit is to associate bound variables to the location of the sub-expression
they are binding
(\href{https://github.com/rems-project/cerberus/issues/270}{Cerberus
\#270}).} a systematic treatment of source location information through
elaboration would be helpful to understand what it would mean, and what is
required for maximal source location accuracy.

\section{Software engineering}

\kl{CN} has been developed for around 4 years at the time of writing. It has
had multiple contributors, some of whom have left the project now. As such,
it benefits from good software engineering practices like any other project.
There are many such things, so I will focus on the important ones.

\subsection{Rich regression testing}

For a long time, \kl{CN} did not have any regression testing and so features
which were implemented, but accidentally broke later, went unnoticed.
Another substantial benefit to regression testing is that it enables assists
with fairly large-scale refactors. Regression testing is also very helpful when
multiple people are working on overlapping parts of the project, so that
changes which accidentally break other things are picked up \emph{before} being
merged in to the main branch.

Regression testing should not only capture the ability of the program to
execute with or without errors, but also \emph{the output} in \emph{both} of
those cases. If a test is intended to fail before a \kl{VIP}-related change,
and it continues to fail after the change, it is often good to check that it is
failing \emph{for the same reason} (we would not a division-by-zero test
changing its behaviour after a change to the memory-object model).

To assist with this, I wrote a small but increasingly sophisticated Python
script\sidenote{\href{https://github.com/rems-project/cerberus/pull/703}{Cerberus
\#703}.} to run a given program, in a particular configuration (consisting a
name, a filter on file names, and arguments to the program) and capture and
check the return codes and all the output, against a pre-existing file. If
there is a difference, the program will raise an error, and also provide a diff
the developer can apply \emph{across some or all of the affected files} if the
change is expected.

\subsection{Performance benchmarking}

A slow heavily-automated verification tool is basically an unusable
verification tool, and the lack of good performance measurement makes it
difficult to understand why \kl{CN} is slow and how to fix it.

Prior measurements were ad hoc but revealed that most of the time is spent in
the solver, which we believe to still be the case. Whilst this is useful
information, we need better granularity on the subject, to answer the following.
\begin{itemize}
    \item What is the distribution of times per call to the solver?
    \item What part of \kl{CN} generates the slowest problems?
\end{itemize}

We have some performance benchmarking now, enough to demonstrate that enabling
\kl{VIP} is slow and soundly reducing bounds checks\sidenote{By replacing them
with liveness checks.} reduces execution time (\cref{sec:vip-ref}). However, we
do not have good insight into explaining these observations. The measurements
are very noisy and coarse \textemdash{} the script measures the total execution
time for running each regression test once, when what we need are answers to
the aforementioned questions.

As I mentioned before, performance remains the main reason that it is nigh-on
impossible to update the \kl{pKVM buddy allocator} to verify after \kl{CN}'s
bit vector update (\cref{sec:buddy-failed-bv}). \kl{CN} is still an order of
magnitude slower than we would like or expect it to be, and so fixing this, and
setting up infrastructure prevent backsliding, is a key priority.

\subsection{Get source location right}

Source location information is the fundamental building block of a useful error
message, without which an error is close to useless. Unfortunately, early
versions of \kl{CN} did not give this crucial data its due leading to an
incredibly poor user experience. This is all the more disappointing because
within the OCaml ecosystem, the \kl{Menhir} parser generator makes getting this
sort of thing right a lot easier than one might initially
expect.~\sidecite{pottier2016reachability}

\paragraph{Keep lexer and parser simple.} Initially, support for \kl{CN}
annotations in comments was implemented with reference to lots of global
mutable state.\sidenote{A feature of ocamllex that seems to be
under-appreciated is that each rule is an OCaml function and can be passed
arbitrary parameters, thus avoid the need to use global mutable state.}
This prevented code re-use, complicated the code and introduced subtle
location bugs because of accidentally re-using shared mutable token
buffers. I simplified the lexer and parser to avoid this, and in the
process improved and unified the error-reporting for parsing C and
\kl{CN}.\sidenote{\href{https://github.com/rems-project/cerberus/pull/252}{Cerberus
\#252}.}

\paragraph{Investigate strange source locations.} Strange source locations are
bugs which deserve to be investigated, or at the very least, have their
triggers documented. One such
issue\sidenote{\href{https://github.com/rems-project/cerberus/issues/382}{Cerberus
\#382}.} uncovered a subtle interaction between the `lexer hack' used to parse
C11,~\sidecite{jourdan2017simple} and a token buffer feature offered by
\kl{Menhir} for more helpful error messages. Again, the issue was mutable state
which was accidentally shared across what should be separate, fresh instances
of the lexer. The solution was to stage the creation of the lexer and its
internal state, and make this explicit in its
API.\sidenote{\href{https://github.com/rems-project/cerberus/commit/c970a43b7126560b229fb55c32ce22bfbc5d23f2}{Cerberus
c970a43b}.}

\paragraph{Write parser error messages if feasible.} Initially, a parse error
in \kl{CN} would simply signal `unexpected token' and signal no other
information, except a ballpark source location which was sometimes right. This
was less than helpful for a number of reasons, most notably not telling the
user what token \emph{was} expected. It is possible to \emph{generate} very
inelegant, but informative parse error messages based on the error message
format output by
\kl{Menhir}.\sidenote{\url{https://gallium.inria.fr/~fpottier/menhir/manual.html\#sec\%3Amessages\%3Aformat}}

I shall now explain how I did
this.\sidenote{\href{https://github.com/rems-project/cerberus/commit/127758764cd6efdabaaaddb60eabf40575864117}{Cerberus
12775876}.} The first line shows the sentence used to get to the error state.
This is useful for \kl{Menhir}, but is not considered to be a good source of
information for writing an error message. This is because the information in
the comments shows the production which is being parsed, and at which point in
it the error occurs. More than half of error states (629) have exactly one
production, and in this case, I generate an error message as shown at the
bottom of the excerpt.

\begin{minted}[fontsize=\footnotesize,linenos,breaklines]{py}
cn_statements: ASSERT WHILE
##
## Ends in an error in state: 1585.
##
## cn_statement -> ASSERT . LPAREN assert_expr RPAREN SEMICOLON [ .. ]
##
## The known suffix of the stack is as follows:
## ASSERT
##

parsing "cn_statement": seen "ASSERT", expecting "LPAREN assert_expr RPAREN SEMICOLON"
\end{minted}

Of course, it is better to fill these and the remainder (593) manually, as I
demonstrated for one
case.\sidenote{\href{https://github.com/rems-project/cerberus/issues/245}{Cerberus
\#245}.} Details of how best to do so are in the \kl{Menhir} manual, along with
examples and links to how \kl{CompCert} uses it for its C parser (proof that
the approach can be scaled). For the easy cases I did not auto-generate, I
output an message which says that an error message is missing for the state
(identified by a number), and that it can be added in the stated file. In any
case, I set up \kl{Menhir} to show exactly where and between which tokens
parsing failed, which is often enough for experienced users to fix the syntax.
\kl{Menhir} also provides functionality to check and merge its auto-generated
error messages with existing ones, so that the latter can be preserved despite
changes in the grammar.

\paragraph{Consider a custom pre-processor.} The bane of any academic C tooling
seems to be the preprocessor. Whilst it faithfully preserves line numbers, and
this tends to be good enough, it does not preserve column information, which in
most cases, leads to inaccurate error messages on any code which is inside, or
after a macro expansion.\sidenote{The macro call and expansion may have the
same length if one is lucky.}

We are caught between two unappealing choices: either present the error message
in terms of the preprocessed code, thus retaining accuracy but sacrificing
relevancy, or present the error message (with the wrong location) in terms of
the original code, retaining relevancy but not accuracy. \kl{CN} currently does
the latter, and given the fact that macros do often represent constants, or
function-like abstractions~(\cref{subesc:takeaway-features}), we would also
like to have them supported in \kl{CN} as well.

There are no easy solutions to this,\sidenote{Even CompCert just calls a
user-specified preprocessor.} but it appears to be a solvable problem. For
example, there maybe ways to coax GCC or Clang to output debug tokens whilst
preprocessing a file, which could be input to a custom lexer which takes the
tokens \emph{and the source location information} output and connects it to the
\kl{Cerberus} front-end. Or, one could write a Clang plugin, or use the Boost
pre-processing library Wave to attempt a similar trick, with more control over
the resulting format. In both cases, it looks like it should be possible to
preserve comments too, but this will not allow running the preprocessor
\emph{inside} \kl{CN} annotations.

It may however, be feasible to implement a custom C preprocessor. What makes
this feasible is the existence of a relatively straightforward pseudo-code
algorithm by Dave Prosser, \emph{which formed the basis of the prose
    specification of macro expansion in the C89/ANSI C
standard}.~\sidecite{prosser1986complete} This algorithm is annotated and
explained~\sidecite{spinellis2008complete} by the author of the CScout
refactoring browser for C,~\sidetextcite{spinellis2010cscout} Diomodis
Spinellis, who was struggling to implement a C preprocessor correctly for about
five years, before finding a reference to an algorithm by Dave Prosser.
Spinellis could not find the algorithm, so he emailed Prosser who
obliged.\sidenote{\url{https://www.spinellis.gr/blog/20060626/}}

Not only is there an algorithm, Spinellis' implementation is available
online\sidenote{\url{https://github.com/dspinellis/cscout/blob/master/src/macro.cpp}}
along with a suite of test
cases\sidenote{\url{https://github.com/dspinellis/cscout/tree/master/src/test/cpp}}
and expected
outputs.\sidenote{\url{https://github.com/dspinellis/cscout/tree/master/src/test/out}}
There is even expertise within the programming language research community on
the algorithm, since it was recently discussed as being closely related to the
strategy for checking termination for expanding type definitions in type
expressions in OCaml.~\sidecite{chataing2024unboxed} This is not to say that it
will be easy, but at this stage such an effort seems more likely to succeed
than the tree-carver was when I embarked on implementing it.

\subsection{Error, do not crash}

\subsection{Log, do not debug}

\subsection{Gate large changes}

\subsection{Proof maintenance}

\subsection{Miscellaneous}

Good commit messages.
Automatic code foramtting.

\section{Invest in error reports}

% Sean Chen - The Anatomy of Error Messages in Rust — RustFest Global 2020 https://youtu.be/oMskswu1SxM?feature=shared
% https://rustc-dev-guide.rust-lang.org/diagnostics.html

\subsection{Counter examples}\label{sec:counter-ex}

\begin{itemize}
    \item Not minimal
    \item Not consistent
    \item Not easy to relate to source
\end{itemize}

\subsection{The unreasonable effectiveness of good error messages}\label{sec:error-msgs}

\begin{itemize}
    \item Translate standards jargon into C programmer friendly words.
\end{itemize}



\bookmarksetup{startatroot}

\chapter{Future Directions}%
\label{chap:future-directions}

\margintoc{}

In this chapter, I will recap some of \kl{CN}'s more pressing limitations, and
sketch out some solutions to it.

\section{Inferring frames}

Recall the \kl{Core} operators for expressing most control-flow:
\coreinline{run()} and \coreinline{save()}, first presented % chktex 36
in~\nameref{sec:core-grammar}.

These very general constructs are used by \kl{Cerberus}' elaboration into
\kl{Core} to support a variety of control flow constructs: loops (continue,
break, iteration), switches (cases, fall-through and defaults), labels, and
\cinline{goto}.

\kl{CN} currently deals with these on an ad-hoc basis, such as by marking
labels with their original structure and use (loop continue, loop break,
return, switch, case, default) labels, and maximally inlining label bodies to
avoid requiring annotations at awkward places such as loop continues and breaks
and switch cases.

There are two main problems with this approach.

\begin{enumerate}
    \item \textbf{Inlining slows down performance and duplicates
        work.}\sidenote{\href{https://github.com/rems-project/cerberus/issues/289}{Cerberus\#289.}}
        The transformation from \kl{Core} with \coreinline{run()}  % chktex 36
        and \coreinline{save()}to \kl{Core} with non-recursive runs % chktex 36
        inlined and recursive runs hoisted to the top level induces some
        impressive code bloat. A nested-loop
        example,\sidenote{\href{https://github.com/rems-project/cerberus/issues/938}{Cerberus\#938.}}
        shows a 7x increase (226 lines of \kl{Core} to 1389 lines); a
        \cinline{switch} with 3 \cinline{case}s and a \cinline{default}
        example, shows a 5x increase.
    \item \textbf{Loop-annotations do not compose as expected, and are thus confusing and
        verbose.}\sidenote{\href{https://github.com/rems-project/cerberus/issues/913}{Cerberus\#931.}}
\end{enumerate}

The reason for this is the \kl{Core} dynamics for the
\coreinline{run()} operator. In particular, note that the current % chktex 36
continuation $C$ is discarded, and the continuation associated with the label
$\cnnt{id}$, $C_\cnnt{id}$ is resumed.

{\small%
\[
\inferrule[{[Run]}]
  { \mathrm{labelmap} ( \cnnt{id} ) = \left(\cncomp{x_i}{i}\right).\ C_{\cnnt{id}} [ E_{\cnnt{id}} ] \\
    \cncomp{e_i \Downarrow \cnnt{value}_i}{i} }
  { \langle h , C\left[ \cnkw{run}\, \cnnt{id} \left( \cncomp{e_i}{i} \right) \right] , \kappa \rangle
    \rightarrow \left< h , C_{\cnnt{id}} \left[ \left[ \cncomp{\cnnt{value}_i / x_i}{i} \right] E_{\cnnt{id}} \right] , \kappa \right> }
\]}

Its associated (\kl{ResCore}) typing rule is as follows. Note that because
control-flow does not return to this point in the program, the return type is
$\cnkw{false} \wedge \cnkw{I}$, even if, for example, a program is entering and
exiting a loop in a well-bracketed way. This has the effect of requiring the
precondition of the label to consume \emph{all} resources from the context
of a \coreinline{run()}. % chktex 36

{\small%
\[
\cndruleStmtXXRun{}
\]}

% \onlyUseRules{\cndefnStmt{}}{
%     \cndruleStmtXXRun{}
% }

In particular, this means that loop invariants must explicitly mention all
resources in a context (even ones which are morally framed out), so that they
are reinstated upon loop exit. The situation compounds for each level of
loop nesting: invariants of all enclosing loops must be manually threaded through
by the user
(see~\href{https://github.com/rems-project/cerberus/issues/913}{Cerberus\#931}).
The problem holds for all \kl{Core} labels, but only manifests itself at loops
since \kl{CN} inlines all labels whose bodies do not recursively \coreinline{run()} % chktex 36
to themselves.

Whilst it may be possible to engineer \kl{CN} for the common cases,
for example by extending \kl{Core} with specific loop or switch constructs,
the root of the problem will remain to handle labels and \cinline{goto}s in C.

Hence a prinicipled solution is desirable, and would be useful to ensure
our handling of the common cases is sound.

A \emph{potential} solution is to adjust the grammar and elaboration of
\kl{Core} to borrow an idea from an inductive representation of SSA, based on
work by~\sidetextcite{ghalayini2024denotational}: represent
\emph{dominance-based scoping} as \emph{lexical scoping}.

\begin{quote}
In particular, a variable $x$ is considered to be in scope at a specific point
$P$ if and only if all execution paths from the program’s entry point to $P$
pass through a definition $D$ for $x$. In this case, we say that the definition
$D$ \emph{strictly dominates} $P$. The relation on basic blocks ``$A$ strictly
dominates $B$'' intersected with ``$A$ is a direct predecessor of $B$'' forms a
tree called the \emph{dominance tree} of the [control-flow graph].
\end{quote}

More specifically, replace $\cnkw{save}\ \cnnt{id}\ (x\, {:}{=}\, \cnnt{pce})\ \cnkw{in}\ E$
with a $\cnkw{where}$ expression:
\[
    \cnnt{E}\ \cnkw{where}\ \left( \cncomp{\cnnt{id}_i\ (x_i {:} \beta_i).\ E_i}{i} \right).
\]

Labels $\cnnt{id}_i$ are in scope in $E$ and in each $E_i$, allowing for mutual
recursion, but are encapsulated from the enclosing expression. This is unlike
\kl{Core}'s \coreinline{save()} operator, whose label is in scope of % chktex 36
the enclosing \emph{procedure}. Variables defined in $E$ are not in scope
in $E_i$, which are parameterised over (for simplicity) a single variable
$x_i$. Most importantly, the enclosing expression is its \emph{frame}, and
unlike in \kl{Core}, is preserved in the dynamic semantics.

{\small%
\[
\inferrule[{[OpStmtWhere]}]
  { l_1 \eqdef \cncomp{\cnnt{id}_i\ (x_i {:} \beta_i).\ E_i}{i} }
  { \langle h , E\ \cnkw{where}\ l, w, \kappa \rangle
    \rightarrow \left< h , \cnkw{pop}_l (E) , l \Colon w, \kappa \right> }
\]}

{\small%
\[
\inferrule[{[OpStmtRunW]}]
  { w \eqdef l_1 \Colon \ldots \Colon l_k \Colon w'
    \and \cnnt{id}\ (x {:} \beta).\ E \in l_k
    \and \cnnt{pce} \Downarrow \cnnt{value} }
  { \langle h , \cnkw{run}\, \cnnt{id}\,\cnnt{pce}, w, \kappa \rangle
    \rightarrow \left< h , \left[ \cnnt{value} / x \right] E , l_k \Colon w', \kappa \right> }
\]}

The \kl{Core} thread-local reduction configuration is extended with a
\emph{stack} $w$, where each element is a family of label definitions
($\cncomp{\cnnt{id}_i\ (x_i {:} \beta_i).\ E_i}{i}$). Upon entry, the label
definitions are pushed ($\Colon$) on to the stack. When $\cnkw{run}$ning to a
label, we search the stack ($w \eqdef l_1 \Colon \ldots \Colon{} l_k \Colon
w'$) for it ($\cnnt{id}\ (x {:} \beta).\ E \in l_k$) and jump to its body,
proceeding with only the labels at and below that level on the stack ($l_k
\Colon{} w'$), and $w$ stack is cleared when returning from a procedure.

Note that, \sidetextcite{ghalayini2024denotational} stratify their language
into statements and expressions; their $\cnkw{where}$ construct is a statement,
so the value it computes is never bound. A $\cnkw{pop}_l$ expression
marks the exit from a $\cnkw{where}$ expression when it returns a value.

{\small%
\[
\inferrule[{[OpStmtPop]}]
  { w \eqdef l_1 \Colon \ldots \Colon l_k \Colon w' }
  { \langle h , \cnkw{pop}_{l_k} (\cnkw{pure}(v)), w, \kappa \rangle
    \rightarrow \left< h , \cnkw{pure}(v), w', \kappa \right> }
\]}

As for typing, I conjecture the $\cnkw{where}$-stack in the dynamics would be
mirrored in an \emph{ordered} resource context $\mathcal{W}$, consisting of
linear resource contexts $\mathcal{R}$, separated by label definitions $l$ as
ordering markers. Typing these definitions requires threading such a
$\mathcal{W}$ through.

{\small%
\[
\inferrule[{[WhereDefns]}]
  { \cncomp{\cnnt{fun}_i \rightsquigarrow \mathcal{C}_i; \mathcal{L}_i; \Phi_i; \mathcal{R}_i}{i}
    \\ \cncomp{\mathcal{C}, \mathcal{C}_i; \mathcal{L}, \mathcal{L}_i; \Phi, \Phi_i; \mathcal{R}_i \Colon \mathcal{W} \vdash E_i \Leftarrow \cnnt{ret}}{i} }
  { \mathcal{C}; \mathcal{L}; \Phi; \mathcal{W} \vdash \cncomp{\cnnt{id}_i {:} \cnnt{fun}_i.\ E_i}{i} \Leftarrow \cnnt{ret} }
\]}

Typing a $\cnkw{where}$ expression itself would pass any resources it has to the
left expression, and adding the definitions as an ordering marker to bring them
in scope. The same is done for checking the definition bodies too, because
labels can refer to each other (mutually) recursively.

{\small%
\[
\inferrule[{[StmtWhere]}]
  { \mathcal{C}; \mathcal{L}; \Phi; \mathcal{R} \Colon l \Colon \mathcal{W} \vdash E \Leftarrow \cnnt{ret}
    \and \mathcal{C}; \mathcal{L}; \Phi; l \Colon \mathcal{W} \vdash l \Leftarrow \cnnt{ret} }
  { \mathcal{C}; \mathcal{L}; \Phi; \mathcal{R} \Colon \mathcal{W} \vdash{} E\ \cnkw{where}\  l \Leftarrow \cnnt{ret} }
\]}

With this, I can sketch out a rough idea on typing for a $\cnkw{run}$ to any
label in scope. First any resources in the intervening depths would need to
disposed, for example, removing the storage for block-local variables. After
that, the $\cnnt{spine}$ would be checked against the function type
$\cnnt{fun}$ (obtained via a lookup of the label $\cnnt{id}$ in the ordered
resource context). Only the resources at the same level as the target
$\cnkw{where}$ expression need to be consumed, everything above it is `framed'
when checking the label body.

{\small%
\[
\inferrule[{[StmtRun]}]
  { \mathcal{W} \eqdef \mathcal{R}_1 \Colon l_1 \Colon \ldots \Colon \mathcal{R}_k \Colon l_k \Colon \ldots \Colon \mathcal{R}_n \Colon l_n
    \and \cnnt{id} {:} \cnnt{fun}.\ E' \in l_k
    \\ \mathcal{C}; \mathcal{L}; \Phi; \mathcal{R}_1 , \ldots, \mathcal{R}_{k-1} \vdash E \Leftarrow \Sigma y{:} \cnkw{unit}.\ \cnkw{I}
    \\ \mathcal{C} ; \mathcal{L} ; \Phi ; \mathcal{R}_k \vdash \cnnt{spine} \Colon \cnnt{fun} >\!> \outpol{\sigma}; \outpol{\cnkw{false} \wedge \cnkw{I}} }
    { \mathcal{C}; \mathcal{L}; \Phi; \mathcal{W} \vdash{} \cnkw{let\ strong\ ()\ =}\ E\ \cnkw{in}\ \cnkw{run}\ \cnnt{id}\,\cnnt{spine} \Leftarrow \cnkw{false} \wedge \cnkw{I} }
\]}

As in~\arefpart{formalisation}, the aboves rules are phrased over a
hypothetical, explicitly annotated \kl{ResCore} with this new construct. Whilst
the setup enables inferring frames for a label, inferring preconditions
$\cnnt{fun}$ remains a distinct problem. As a first approximation, loops and
mutually recursive labels would require annotations from the user, though
annotations on loops would no longer need to manually thread through unusued
parts of the resource context. Other labels could have their preconditions
inferred by saving the typing context when the first $\cnkw{run}$ is
encountered, and ensuring that subsequent ones match the first, thus preventing
inlining and code bloat.

\section{Weak sequencing}

Sequencing strength is one of the most subtle and confusing aspects about C,
and its treatment in \kl{Core}, whilst inventive, is difficult to understand
and reason about.

Sequencing strength is used to specify the (lack of) ordering between memory
actions. Quoting at length from \sidetextcite{memarian2022cerberus}:

\begin{quote}
 Each use of an action is either ``positive'' (the default case) or
 ``negative'' (if the action appears in the syntax as the operand of the
 $\cnkw{neg()}$ operator). [..] Intuitively, a \kl{Core} action is negative
 when it elaborates what the C standard calls a ``side-effect'' (as opposed to
 value computations), that is, memory accesses which are not directly used for
 producing the value of a C expression. This is for example, the store
 performed by a postfix increment operator.
\end{quote}

The $\cnkw{let}\ \cnkw{strong}\ \cnnt{pat}\ \cnkw{=}\ \cnnt{E}_1\ \cnkw{in}\
\cnnt{E}_2$ construct sequences all memory actions performed by $\cnnt{E}_1$
before those performed by $\cnnt{E}_2$. However, $\cnkw{let}\ \cnkw{weak}$ only
sequences positive memory actions similarly; in other words, for $\cnkw{let}\
\cnkw{weak}\ \cnnt{pat} = \cnnt{E}_1\ \cnkw{in}\ \cnnt{E}_2$, any
negative memory actions performed by $\cnnt{E}_1$ are unsequenced with respect
to all memory actions performed by $\cnnt{E}_2$. A \emph{race} occurs if
unsequenced memory actions have overlapping footprints, and is deemed \kl{UB}.

The formalisation and implementation of weak sequencing is complex. Here, I
offer a simpler presentation which I conjecture is equivalent and then offer
few remarks on how to adapt the \kl{CN} resource framework to fit it.

A footprint $\cnnt{fp}$ of an action is the range of bytes in memory it touches.
Like actions, footprints can be positive ${\color{red}\cnkw{pos}(\cnnt{fp})}$
or negative ${\color{red}\cnkw{neg}(\cnnt{fp})}$. A set of such footprints is
called an annotation ${\color{red}A}$, and I use the notation
${\color{red}A^{-}}$ to denote all the negative footprints in ${\color{red}A}$
and ${\color{red}A^{+}}$ for all the positive ones.

Negative actions step to a new $\cnkw{delay}(\cnnt{actions}, \cnnt{E})$
construct, where $\cnnt{actions}$ is a set.

{\small%
\[
\inferrule[{[ActionNeg]}]
    { \sigma \overset{\cnnt{action}}{\longrightarrow} \langle \_ , \cnkw{Unit}, \cnnt{fp} \rangle }
    { \langle \sigma, \cnkw{neg}(\cnnt{action}), \kappa \rangle \rightarrow
      \langle
          \sigma,
          \cnkw{delay}(\{\, \cnnt{action} \, \},
            \prescript{\color{red}\{\cnkw{neg}(\cnnt{fp})\}}{}{\cnkw{pure}(\cnkw{Unit})}),
          \kappa
      \rangle }
\]}

The dynamics of this construct allowing reducing either one of the actions in
the set, or the expression inside at any time; the latter because it counts as
a context $\cnnt{C} \, {:}{=} \, \ldots \mid \cnkw{delay}(\cnnt{actions}, C)$.
All $\cnkw{delay}$ed actions are negative, and all negative actions evaluate to
$\cnkw{Unit}$ (since they are side effects) and so the result of the action is
ignored. The footprint is unnecessary as per the previous rule.

{\small%
\[
\inferrule[{[DelayAction]}]
    { \sigma \overset{\cnnt{action}}{\longrightarrow} \langle \sigma' , \cnkw{Unit}, \_ \rangle }
    { \langle
          \sigma,
            \cnkw{delay}(\cnnt{actions} \cup \{\, \cnnt{action} \,\}, E),
          \kappa
      \rangle
      \rightarrow
      \langle \sigma', \cnkw{delay}(\cnnt{actions}, E), \kappa \rangle }
\]}

The next ingredient is a $\cnkw{disj}({\color{red}A}, E)$ construct, which is
used to check if ${\color{red}A}$ races with $E$.

{\small%
\[
\inferrule[{[Disj]}]
    { {\color{red}\neg \mathrm{do\_race}(A_1, A_2)} }
    { \langle
          \sigma,
          \cnkw{disj}({\color{red}A_1}, \prescript{\color{red}A_2}{}{\cnkw{pure}(v)}),
          \kappa
      \rangle
      \rightarrow
      \langle
          \sigma,
          \prescript{\color{red}A_1 \cup A_2}{}{\cnkw{pure}(v)},
          \kappa
      \rangle }
\]}

The construct is used for weak sequencing, and is also a context $\cnnt{C} \,
{:}{=} \, \ldots \mid \cnkw{disj}({\color{red}A}, C)$, allowing for the
expression inside to reduce. Splitting the annotations, placing the positive
ones on the outside and the negative ones in a $\cnkw{disj}$ ensures the
negative actions which \emph{were} part of evaluating $\cnkw{pure}(v)$ are
treated as if unsequenced with $E$, i.e.\ required to not race.

{\small%
\[
\inferrule[{[Weak]}]
    {  }
    { \langle
          \sigma,
          \cnkw{let}\ \cnkw{weak}\ \cnnt{pat}\ =\ \prescript{\color{red}A}{}{\cnkw{pure}(v)}\ \cnkw{in}\ E,
          \kappa
      \rangle
      \rightarrow
      \langle
          \sigma,
          \prescript{\color{red}A^{+}}{}{\cnkw{disj}({\color{red}A^{-}}, [ v / \cnnt{pat} ] E)},
          \kappa
      \rangle }
\]}

Nested $\cnkw{delay}$s can be combined by taking the union of the actions.

{\small%
\[
\inferrule[{[HoistDelay]}]
    {  }
    { \langle \sigma, \cnkw{delay}(\cnnt{actions}_1, \cnkw{delay}(\cnnt{actions}_2, E)), \kappa \rangle
      \rightarrow
      \\ \quad
      \langle \sigma, \cnkw{delay}(\cnnt{actions}_1 \cup \cnnt{actions}_2, E), \kappa \rangle }
\]}

Cruicially, $\cnkw{delay}$s can be hoisted out of $\cnkw{let}\ \cnkw{weak}$,
$\cnkw{unseq}$, $\cnkw{disj}$ and annotated $\prescript{\color{red}A}{}{E}$
expressions. The rule for the first is given below, and the rest follow a
similar pattern.

{\small%
\[
\inferrule[{[HoistWeak]}]
    {  }
    { \langle
          \sigma,
          \cnkw{let}\ \cnkw{weak}\ \cnnt{pat}\ =\ \cnkw{delay}(\cnnt{actions}, E_1)\ \cnkw{in}\ E_2,
          \kappa
      \rangle
      \rightarrow
      \\ \quad
      \langle
          \sigma,
          \cnkw{delay}(\cnnt{actions}, \cnkw{let}\ \cnkw{weak}\ \cnnt{pat}\ =\ E_1\ \cnkw{in}\ E_2),
          \kappa
      \rangle }
\]}

Notably absent, $\cnkw{let}\ \cnkw{strong}$ or $\cnkw{bound}$ expressions
\emph{do not} permit hoisting $\cnkw{delay}$s out of them. This forces any
delayed actions to be evaluated and any races to checked before proceeding.

Though unusual, the constructs and rules are simple. One straightforward
approach to typing could be to split resource types into positive and negative.
Negative resources are not usable, and are converted into positive ones after a
$\cnkw{let}\ \cnkw{strong}$ or a $\cnkw{bound}$ expression.

\section{Join-points}

Currently, \kl{CN} bifurcates control-flow during type checking at every
branch, with no way to merge paths once this is done. The main issue with this
is that any resource inference which is not path dependent repeated across
different branches.

Whilst it is possible to define syntax and rules for user-provided join-points,
the better solution may be to not branch the control-flow in the first place.

This requires (a) defining how to merge constraint and resource contexts after
an if-expression and (b) defining how to handle resource lookup/inference with
conditional-resources in the context whilst avoiding performance slowdown.

Assume a judgement of the form $\Phi ; \mathcal{R} \vdash E \dashv \cnnt{term}
\mathrel{\&} \Phi' ; \mathcal{R}'$ is the input constraint context,
$\mathcal{R}$ is the input resource context, context, $E$ is the input
\kl{Core} expression, $\cnnt{term}$ is the output symbolic evaluation of the
\kl{Core} expression, $\Phi' ; \mathcal{R}'$ are the output constraint and
resource contexts respectively (to reduce clutter, I am omitting variable
environments).

We can define join-points for conditional expressions as below.

{\small%
\[
\inferrule[{[StmtIf]}]
  { \Phi \wedge \cnnt{pce} ; \mathcal{R} \vdash E_1 \dashv \cnnt{term}_1 \mathbin{\&} \Phi_1 ; \mathcal{R}_1
    \\ \Phi \wedge \neg \cnnt{pce} ; \mathcal{R} \vdash E_2 \dashv \cnnt{term}_2 \mathbin{\&} \Phi_2 ; \mathcal{R}_2
    \\ \cnnt{term}_3 \eqdef \cnkw{if}\ \cnnt{pce}\ \cnkw{then}\ \cnnt{term}_1\ \cnkw{else}\ \cnnt{term}_2
    \\ \Phi' \eqdef (\cnnt{pce} \rightarrow \Phi_1) \wedge ( \neg \cnnt{pce} \rightarrow \Phi_2)
    \\ \mathcal{R}' \eqdef (\mathcal{R}_1 \cap \mathcal{R_2})
       \ast (\cnkw{if}\ \cnnt{pce}\ \cnkw{then}\ (\mathcal{R}_1 - \mathcal{R}_2)\ \cnkw{else}\ (\mathcal{R}_2 - \mathcal{R}_1)) }
  { \Phi ; \mathcal{R}
    \vdash \cnkw{if}\ \cnnt{pce}\ \cnkw{then}\ E_1\ \cnkw{else}\ E_2
    \dashv \cnnt{term}_3 \mathbin{\&} \Phi' ; \mathcal{R}' }
\]}

Calculating the intersection and difference of two resource contexts, should be
done carefully to avoid having it be accidentally quadratic in complexity.
Marking which resources are not used will also help, by framing them out and
thus reducing the number of items considered for intersection and difference.
Calculating the intersection precisely is not necessary for soundness; its
primary benefit is to reduce work later.


The problem then becomes how to use a conditional-resource in the context.
Specifically, how to handle the below situation without resorting to an
exponential blow-up. Inputs are constraints $\Phi$, resources $\mathcal{R}$ and
the requested resource $r$; Outputs are the remaining lookup $\mathcal{R}'$,
and the leftover context $\mathcal{R}''$ after a partially successful lookup.

{\small%
\[
    \Phi \vdash r \mathrel{{\in}{?}} \mathcal{R} \rightsquigarrow \outpol{\mathcal{R}'; \mathcal{R}''}
\]}

The first thing to check would be \emph{nominal} equality. A conditional
resource may come from what a user wrote (in a specification or a predicate
definition) and so can be identified as an anonymous predicate of sorts:
$\cnnt{\#loc}\;(\ldots)$. In this case, it would suffice to check equality on
source location and arguments.

{\small%
\[
\inferrule[{[StmtIf-Nominal]}]
    { }
    { \Phi \vdash \cnnt{\#loc}\;(\cnnt{iargs})
            \mathrel{{\in}?}
            (\mathcal{R}, \cnnt{\#loc}\;(\cnnt{iargs}))
            \rightsquigarrow \cdot ; \mathcal{R} }
\]}

If the nominal equality fails, the next thing to check would be
\emph{structural} equality, if either $\cnnt{pce}_1 \leftrightarrow
\cnnt{pce}_2$ or $\cnnt{pce}_1 \leftrightarrow \neg \cnnt{pce}_2$, then the
check can recurse on both branches.

{\small%
\[
\inferrule[{[StmtIf-TwoSided1]}]
    { \Phi \vdash \cnnt{pce}_1 \leftrightarrow \cnnt{pce}_2
      \\ \Phi \wedge \cnnt{pce}_1 \vdash \mathcal{R}_{21}
      \mathbin{{\in}?} (\mathcal{R} , \cnnt{r.then} {:} \mathcal{R}_{11}) \rightsquigarrow \cdot ; \mathcal{R}
      \\ \Phi \wedge \neg \cnnt{pce}_1 \vdash \mathcal{R}_{22}
      \mathbin{{\in}?} (\mathcal{R} , \cnnt{r.else} {:} \mathcal{R}_{12}) \rightsquigarrow \cdot ; \mathcal{R} }
    { \Phi \vdash (\cnkw{if}\ \cnnt{pce}_2\ \cnkw{then}\ \mathcal{R}_{21}\ \cnkw{else}\ \mathcal{R}_{22})
           \mathbin{{\in}?}
           (\mathcal{R} , r {:} \cnkw{if}\ \cnnt{pce}_1\ \cnkw{then}\ \mathcal{R}_{11}\ \cnkw{else}\ \mathcal{R}_{12})
           \rightsquigarrow \cdot ; \mathcal{R} }
\]}

And similarly for $\cnnt{pce}_1 \leftrightarrow \neg \cnnt{pce}_2$. If a
two-sided match fails, then the next thing to check would be a one-sided
strucural equality.

{\small%
\[
\inferrule[{[StmtIf-OneSided1]}]
    { \Phi \vdash \cnnt{pce}_1 \rightarrow \cnnt{pce}_2
      \\ \Phi \wedge \cnnt{pce}_1 \vdash \mathcal{R}_{21}
      \mathbin{{\in}?} (\mathcal{R} , \cnnt{r.then} {:} \mathcal{R}_{11}) \rightsquigarrow \cdot ; \mathcal{R} }
    { \Phi \vdash (\cnkw{if}\ \cnnt{pce}_2\ \cnkw{then}\ \mathcal{R}_{21}\ \cnkw{else}\ \mathcal{R}_{22})
           \mathbin{{\in}?}
           (\mathcal{R} , r {:} \cnkw{if}\ \cnnt{pce}_1 \cnkw{then}\ \mathcal{R}_{11}\ \cnkw{else}\ \mathcal{R}_{12})
           \\ \quad \rightsquigarrow (\cnkw{if}\ \cnkw{pce}_2\ \cnkw{emp}\ \cnkw{else}\ \mathcal{R}_{22}) ; \mathcal{R} }
\]}

Similarly we can test $\cnnt{pce}_1 \rightarrow \neg \cnnt{pce}_2$, $\neg
\cnnt{pce}_1 \rightarrow \cnnt{pce}_2$ or $\neg \cnnt{pce}_1 \rightarrow \neg
\cnnt{pce}_2$, and proceed accordingly.

If all of these fail, then the inference fails, or requires the user to give
explicit permission to split on the condition of the resource and duplicate the
lookup across two branched resource contexts. Notably, the permission is given
for a \emph{resource} for every lookup, and not a condition.

% This might be necessary if there's a situation a bit like the below:
% ```
%            r1                  r2
%          /    \              /    \
%    r1.then    r1.else   r2.then r2.else
%         |        \        /       |
%         |         req.then        |
%         |-------\          /------|
%                  \        /
%                   req.else
% ```
%
% In this instance, the user would need to give permission to split on both `r1`
% and `r2` separately, if the their conditions lined up suitably.
%
% If the resource is consumed from the context (e.g. by a lemma or a function
% call) then there will be no more splitting.

Hence the user opts-in to the slow case, if it happens to be the right thing to
do. If there's a smarter way to eliminate conditional-resources from the
context, the user can write a lemma/function to do so.

Thus we have a scheme of automating the fast common cases, explicitly opting-in
to the slow general case, \emph{and} a mechanism for the user to intervene with
something else should it be necessary.

\section{Integers and bitvectors}

Sometimes specifications are written and proved over integers, for example,
with the buddy allocator, but need to be called from the bit-vector world, and
sometimes vice-versa.

Integers are unbounded and bit-vectors are bounded. Mapping from bit-vectors to
integers is straightforward: insert $\cnnt{term} \mod 2^k$ as needed, as long
as accepts the resulting expression may not being decidable/interpretable: $x
\times y$ in bit-vectors will be $\mathrm{mul\_uf}(x,y) \mod 2^{64}$ in
integers. We can apply this transformation and get an expression which will
give the same answer.

Going from integers to bit-vectors is difficult because the process is
fundamentally partial, i.e.\ it is only defined for a certain inputs under
certain conditions, however SMT expressions must be total.

Let $\beta$ be the basetypes in the bounded bit-vector world. Let $\iota$ be
the basetypes in the unbounded integer world. Let $\cnkw{num} \mathrel{{:}{=}}
\cnkw{int} \mid \cnkw{bit}$ be a parameter for the following. Let
$\cnnt{term}_{\cnkw{num}}$ be the grammar of SMT terms, parameterised over the
choice of integers or bit-vectors.

{\small%
\begin{align*}
\cnnt{term}_{\cnkw{num}}, t_{\cnkw{num}}, b_{\cnkw{num}} \mathrel{{:}{=}}
\, & x \mid t_{\cnkw{num}} \oplus t'_{\cnkw{num}} \mid
\cnkw{let}\ x\ = t_{\cnkw{num}}\ \cnkw{in}\ t'_{\cnkw{num}} \mid \\
& \cnkw{if} \, (b_{\cnkw{num}}) \,\{\, t_{\cnkw{num}} \,\}\, \cnkw{else} \,\{\, t'_{\cnkw{num}} \,\} \mid
f(t^1_{\cnkw{num}}, \ldots, t^n_{\cnkw{num}})
\end{align*}}

Under this convention, closed terms $t_{\cnkw{int}} {:} \iota$ and
$t_{\cnkw{bit}} {:} \beta$. I use $b$ for formulas of boolean type:
$b_{\cnkw{bit}} {:} \cnkw{bool}$.

Let $\mathrm{Decls}$ be a context with declaration types of pure SMT functions
(in CN, defined with the $\cnkw{function}$ keyword). The judgement
$\mathrm{Decls} ; \Gamma \vdash t_{\cnkw{int}} \Rightarrow \iota$  means that
under declarations $\mathrm{Decls}$ and context $\Gamma$, expression in the
integer world $t_{\cnkw{int}}$ has type $\iota$. $\mathrm{Decls}$ is a map from
function names to function types, $\Gamma$ is a map from parameters and local
variables to types.

{\small%
\[
\inferrule[]
  { \mathrm{Decls} , ( f {:} ( \iota_1 \times \cdots \times \iota_n \rightarrow \iota ) ) ; x_1 {:} \iota_1 , \ldots , x_n {:} \iota_n \vdash t_{\cnkw{int}} \Rightarrow \iota }
  { \mathrm{Decls} ; \cdot \vdash \cnkw{function}\ (\iota)\ f(\iota_1 x_n, \ldots, \iota_n x_n) \, \{ \, t_{\cnkw{int}} \,\} \Rightarrow \iota_1 \times \cdots \times \iota_n \rightarrow \iota }
\]}

For bit-vectors, the setup is similar, except recursive functions, have named
recursive constraints, named $f_b$.

Note that the bottom-rule has \emph{two} function definitions: the first one is
the precondition, assuming which, the second gives the same result as the
original (integer-world) function.

{\small%
\[
\inferrule[]
  { \mathrm{Decls}',
        {\begin{matrix}
            ( f' {:} ( \beta_1 \times \cdots \times \beta_n \rightarrow \beta ) ),
            \\ ( f'_b {:} ( \beta_1 \times \cdots \times \beta_n \rightarrow \cnkw{bool} ) )
        \end{matrix}} ; x_1 {:} \beta_1, \ldots , x_n {:} beta_n \vdash t_{\cnkw{bit}} \Rightarrow \beta }
  { \mathrm{Decls} ; \cdot \vdash
      {\begin{matrix}
      \cnkw{function}\ (\beta)\ f'_b(\beta_1 x_n, \ldots, \beta_n x_n) \, \{ \, b_{\cnkw{bit}} \,\}
      \\ \cnkw{function}\ (\beta)\ f'(\beta_1 x_n, \ldots, \beta_n x_n) \, \{ \, t_{\cnkw{bit}} \,\}
      \end{matrix}}
     \Rightarrow
     {\begin{matrix}
     \beta_1 \times \cdots \times \beta_n \rightarrow \cnkw{bool}
     \\ \beta_1 \times \cdots \times \beta_n \rightarrow \beta
     \end{matrix}} }
\]}

I will use $\rightsquigarrow$ to define a mapping from the integer world to the
bit-vector world. For example, the mapping between types, from $\iota$ to
$\beta$, would be along the lines of the below.

{\small
\begin{align*}
    \cnkw{integer}   &\rightsquigarrow \cnkw{i64} \\
    \cnkw{unit}      &\rightsquigarrow \cnkw{unit} \\
    \cnkw{alloc\_id} &\rightsquigarrow \cnkw{alloc\_id} \\
    \{\, \cnkw{integer}\ \cnnt{addr}, \cnkw{alloc\_id}\ \cnnt{id} \,\}
        &\rightsquigarrow \{\, \cnkw{i64}\ \cnnt{addr}, \cnkw{alloc\_id}\ \cnnt{id} \,\} \\
    \cnkw{map}\ \iota_1\ \iota_2 &\rightsquigarrow \cnkw{map} \beta_1\ \beta_2
                                 \ \textrm{where}\ \iota_1 \rightsquigarrow \beta_1, \iota_2 \rightsquigarrow \beta_2
\end{align*}}

For typing judgments, variables are typed with a lookup in the context.

{\small
\[
\inferrule[]
  { \Gamma ( x ) = \iota \rightsquigarrow \Gamma' ( x ) = \beta }
  { ( \mathrm{Decls} ; \Gamma \vdash x \Rightarrow \iota ) \rightsquigarrow ( \mathrm{Decls}' ; \Gamma' \vdash x \Rightarrow ( \beta , \cnkw{true} ) ) }
\]}

Typing for binary operation is translated by inserting no-overflow
constraints.\sidenote{\url{https://github.com/Z3Prover/z3/discussions/5138\#discussioncomment-544642}}
I will omit the kind annotation on terms to reduce syntactic noise, using $t$
for $t_{\cnkw{num}}$ and $t'$ for $t_{\cnkw{bit}}$. The context should also
make it clear which is intended.

{\small%
\[
\inferrule[]
    { ( \mathrm{Decls} ; \Gamma \vdash t_1 \Rightarrow \cnkw{int} ) \rightsquigarrow ( \mathrm{Decls} ; \Gamma' \vdash t'_1 \Rightarrow ( \cnkw{i64}, b_1 ) )
    \\ ( \mathrm{Decls} ; \Gamma \vdash t_2 \Rightarrow \cnkw{int} ) \rightsquigarrow ( \mathrm{Decls} ; \Gamma' \vdash t'_2 \Rightarrow ( \cnkw{i64}, b_2 ) ) }
    { ( \mathrm{Decls} ; \Gamma \vdash t_1 \oplus t_2 \Rightarrow \cnkw{int} ) \rightsquigarrow
    \\ \quad ( \mathrm{Decls}' ; \Gamma' \vdash t'_1 \oplus t'_2 \Rightarrow ( \cnkw{i64} , b_1 \wedge b_2 \wedge \mathrm{no\_ovfl}(\oplus, t'_1, t'_2) ) ) }
\]}

Typing for let-expressions is translated as follows. The precondition for the
whole expresison consists of the precondition for the bound expression, and for
the same for the body, which may refer to the bound variable.

{\small%
\[
\inferrule[]
    { ( \mathrm{Decls} ; \Gamma \vdash t_1 \Rightarrow \iota_1 ) \rightsquigarrow ( \mathrm{Decls}' ; \Gamma' \vdash t'_1 \Rightarrow ( \beta_1, b_1 ) )
    \\ ( \mathrm{Decls} ; \Gamma , x {:} \iota_1 \vdash t_2 \Rightarrow \iota_2 ) \rightsquigarrow ( \mathrm{Decls} ; \Gamma' , x {:} \beta_1 \vdash t'_2 \Rightarrow ( \beta_2, b_2 ) ) }
    { ( \mathrm{Decls} ; \Gamma \vdash \cnkw{let}\ x = t_1\ \cnkw{in}\ t_2 \Rightarrow \iota_2 ) \rightsquigarrow
    \\ \quad ( \mathrm{Decls}' ; \Gamma' \vdash \cnkw{let}\ x = t'_1\ \cnkw{in}\ t'_2 \Rightarrow ( \beta_2 , b_1 \wedge (\cnkw{let}\ x = t'_1\ \cnkw{in}\ b_2) ) ) }
\]}

Typing for if-expressions is translated as follows. The evaluation of the
branch condtion $t'_1$ is constrained by $b_1$, and the evaluation of each
branch is constrained similarly. Note that the precondition refers to $t'_1$
too.

{\small%
\[
\inferrule[]
    { ( \mathrm{Decls} ; \Gamma \vdash t_1 \Rightarrow \cnkw{bool} ) \rightsquigarrow ( \mathrm{Decls}' ; \Gamma' \vdash t'_1 \Rightarrow ( \cnkw{bool}, b_1 ) )
    \\ ( \mathrm{Decls} ; \Gamma \vdash t_2 \Rightarrow \iota ) \rightsquigarrow ( \mathrm{Decls}' ; \Gamma' \vdash t'_2 \Rightarrow ( \beta, b_2 ) )
    \\ ( \mathrm{Decls} ; \Gamma \vdash t_3 \Rightarrow \iota ) \rightsquigarrow ( \mathrm{Decls}' ; \Gamma' \vdash t'_3 \Rightarrow ( \beta, b_3 ) ) }
    { ( \mathrm{Decls} ; \Gamma \vdash \cnkw{if}\, (t_1)\, \{\, t_2 \,\}\, \cnkw{else} \,\{\, t_3 \,\}\, \Rightarrow \iota ) \rightsquigarrow
    \\ \quad ( \mathrm{Decls}' ; \Gamma' \vdash \cnkw{if}\, (t'_1)\, \{\, t'_2 \,\}\, \cnkw{else} \,\{\, t'_3 \,\}\, \Rightarrow ( \beta , b_1 \wedge (\cnkw{if}\ t'_1\ \cnkw{then}\ b_2\ \cnkw{else}\ b_3) ) ) }
\]}

Lastly, function names, which may be recursive, are handled similarly to variables, and
their arguments by recursion.

{\small%
\[
\inferrule[]
    { \mathrm{Decls}( f ) = \iota_1 \times \cdots \times \iota_n \rightarrow \iota
    \\ \mathrm{Decls}'( f' ) = \beta_1 \times \cdots \times \beta_n \rightarrow \beta
    \\ \mathrm{Decls}'( f'_b ) = \beta_1 \times \cdots \times \beta_n \rightarrow \cnkw{bool}
    \\ ( \mathrm{Decls} ; \Gamma \vdash t_1 \Rightarrow \iota_1 ) \rightsquigarrow ( \mathrm{Decls}' ; \Gamma' \vdash t'_1 \Rightarrow ( \beta_1 , b_1) )
    \\ \cdots ( \mathrm{Decls} ; \Gamma \vdash t_n \Rightarrow \iota_n ) \rightsquigarrow ( \mathrm{Decls}' ; \Gamma' \vdash t'_n \Rightarrow ( \beta_n , b_n) ) }
    { ( \mathrm{Decls} ; \Gamma \vdash f ( t_1, \ldots, t_n) \Rightarrow \iota ) \rightsquigarrow
        \\ ( \mathrm{Decls}' ; \Gamma' \vdash f' (t'_1, \ldots, t'_n )
        \Rightarrow ( \beta , b_1 \wedge \ldots \wedge b_n \wedge f'_b (t'_1 , \ldots , t'_n) ) ) }
\]}

\section{Higher order predicates}

Currently, \kl{CN} has `predicates' which can access and affect both the
constraint context and the resource context, and `functions' which are pure and
can do neither.

The two not only have incompatible grammars, but also different auto-unfolding
behaviour, leading to confusion for the user.

Grammar for predicates is restricted to either straight-line code, or
one-top-level if-then-else, but has good auto-unfolding behaviour. Grammar for
functions is not just restricted, but has no auto-unfolding behaviour,
requiring manual `unfold' commands.

Though they have very different semantics, unifying the grammar and unfolding
behaviour is desirable, so long as there is some way to ensue pure expressions
do not refer to resourceful ones.

Additionally, the lack of a half-way ground where we can affect the constraint
context and access the SMT solver, but not the resourceful one is confusing.
This would allow the user to write morally partial functions and specifications
more naturally, and have \kl{CN} do the heavy lifting of inferring appropriate
preconditions, similar in spirit to the aforementioned translation from
integers to bit-vectors,

One way to solve this is by introducing the distinction at the basetype-level.
Pure basetypes are $\beta$, resourceful ones could be $\cnkw{res}\,\beta$, and
the SMT-related ones could be $\cnkw{smt}\,\beta$. Note that this can be merely
an internal representation to make it easier to specify type inference and
checking. If one is apprehensive about systems programmers seeing parameterised
types (despite the pervasiveness of C++ and popularity of Rust), it does not
need to affect the surface syntax that the user sees.

Terms of type $\cnkw{res}\,\beta$ can be interpreted as map into separation
logic propositions, $\beta \rightarrow \mathsf{Prop}^\mathsf{SL}$, as mentioned
in~\cref{fig:monad-sl}. Terms of $\cnkw{smt}\,\beta$ are instead interpreted as
a \emph{pairs} $\beta \times \mathsf{Prop}^\mathsf{SMT}$. Consider the
aforementioned term grammar extended with $\cnkw{assert}(b)$. A full treatment
would involve a typed translation similar to that for the integer-to-bit-vector
translation, particularly to get the type $\beta$ for translating
$\cnkw{assert}(b)$ to $\cnkw{default}\,\beta$, but a sketch is given below.

{\small%
\begin{align*}
    \llbracket \cnkw{assert}(b) \rrbracket &= ( \cnkw{default}\,\beta, b ) \\
    \llbracket x \rrbracket &= ( x, \cnkw{true} ) \\
    \llbracket t_1 \oplus t_2 \rrbracket &= ( t'_1 \oplus t'_2 , b_1 \wedge b_2 )
    \ \text{where}\ \llbracket t_i \rrbracket = ( t'_i , b_i ),\ i \in \{\, 1, 2 \,\} \\
    \llbracket \cnkw{if}\, (t_1) \,\{\, t_2 \,\}\, \cnkw{else} \,\{\, t_3 \,\} \rrbracket &=
        (\cnkw{if}\, (t'_1) \,\{\, t'_2 \,\}\, \cnkw{else} \,\{\, t'_3 \,\}, \\
        & \qquad b_1 \wedge (\cnkw{if}\, (t'_1) \,\{\, b_2 \,\}\, \cnkw{else} \,\{\, b_3 \,\})) \\
        & \quad \text{where}\ \llbracket t_i \rrbracket = ( t'_i , b_i ),\ i \in \{\, 1, 2, 3 \,\} \\
    \llbracket \cnkw{let}\ x = t_1\ \cnkw{in}\ t_2 \rrbracket &=
    (\cnkw{let}\ x = t'_1\ \cnkw{in}\ t'_2, b_1 \wedge (\cnkw{let}\ x = t'_1\ \cnkw{in}\ b_2)) \\
        & \quad \text{where}\ \llbracket t_i \rrbracket = ( t'_i , b_i ),\ i \in \{\, 1, 2 \,\} \\
    \llbracket f (t_1, \ldots, t_n) \rrbracket &=
    (f (t'_1, \ldots, t'_n), b_1 \wedge \ldots b_n \wedge f_b (t'_1, \ldots, t'_n)) \\
        & \quad \text{where}\ \llbracket t_i \rrbracket = ( t'_i , b_i ),\ i \in \{\, 1, \ldots, n \,\} \\
\end{align*}}

\section{Predicate definition checks}

\kl{CN}'s resource inference can diverge on certain inputs. Below is a small
example to illustrate the problem, and later, a solution.

\cfile[lastline=21,highlightlines={4-21}]{code/diverge.c} % chktex 8

First the code declares the simplest possible recursive structure:
\cinline{struct node} contains one \cinline{struct node*} called
\cinline{next}.

The first predicate asserts ownership of a points-to at \cninline{front}, binds
the pointee to \cninline{F}, and then calls \cninline{List_Next_Seg} with
\cninline{front}, \cninline{F.next}, and \cninline{back}. In turn, that
predicate checks whether \cninline{front} and \cninline{back} are equal,
returning if so, and recursing on \cninline{next} and \cninline{back} on the
original predicate if not. If not for \kl{CN}'s syntax limitations
(\nameref{sec:restriction-branching}), the two definitions could be combined
into one. Though the predicate is unusual, it is derived from an example
written by a user, and is a priori a sensible
definition (for non-empty list segments).\sidenote{\href{https://github.com/rems-project/cerberus/issues/451}{Cerberus\#451}}

\cfile[firstline=23,highlightlines={25-27}]{code/diverge.c} % chktex 8

The definitions are problematic because they can be used in a context which
causes \kl{CN} to diverge. Using \cinline{List_Seg(front, back)} % chktex 36
with \cinline{RW(back)} causes the auto-unfolding mechanism to loop. % chktex 36
This is because the \cinline{List_Seg} predicate included ownership of
\cinline{front}, and so by the time \kl{CN} symbolically evaluates the
condition in \cinline{List_Next_Seg}, it deduces
\cinline{!ptr_eq(front, back)} and recurses. This introduces new % chktex 36
ownership for \cinline{front->next}, the condition is guaranteed to always be
false in a context with ownership of \cinline{back}.

The root of the problem is the combination of automatically deriving logical
facts based on the resource context, and the auto-unfolding scheme for
recursive predicates. Intuitively, the problem is that the constraints on, and
behaviour of pointers changes after they have been owned. A potential remedy
for this is a monotonic dataflow analysis.

Informally, for a block of (mutually) recursive predicates, all parameters
start off marked as safe. The predicates are evaluated symbolically, and any
variables used in the argument of \cninline{RW}/\cninline{W} predicate are then
marked as unsafe. Those unsafe variables cannot then be used inside the
condition of an \cninline{if}-expression which is guarding a (mutually)
recursive call.

In the above example, at line 4, \cninline{front} is safe, but after line 5
\cninline{front} is unsafe. The call on line 6 then propogates that information
to the parameter on lines 11. Finally, line 15 has the condition which guards
the recursive call on line 18. Because \cninline{front} is unsafe, its use in
on line 15 is rejected.

\chapter{Conclusion}%
\label{chap:conclusion}

\margintoc{}

In this thesis, I argued that building a verification tool for C, suitable
for handling low-level systems programming idioms, is two parts engineering,
and one part theory.

In~\arefpart{formalisation}, I discussed the origin of \kl{CN} as a tool
to verify the \kl{pKVM} hypervisor, and \kl{CN}'s design goal of lowering the
cost of verification and being usable by `systems programmers who know
Haskell'. I showed how these engineering constraints informed the choice of
\kl{CN}'s many theoretical foundations. I showed that those these foundations
are well-established and individually well understood, its application to a
large calculus \kl{Core} with unique features presents new challenges and
solutions, such as closely linking resource contexts to heaps by representing
them as separation logic predicates, and using explicit proof terms to encode
transformations on both. I concluded by feeding back insights from the theory
into actual engineering questions we face in developing and extending \kl{CN}.

In~\arefpart{mem-model}, I described a complex, formal, memory object
model for C which accurately captures most of the behaviour exhibited by
existing optimsing compilers. I showed that we can simplify some of the
complexity by exploiting the fact that in verification, small and
easy-to-explain changes to the code are justified if they make the typing rules
easier to implement and explain. These engineering constraints defined a
flexible perimeter for the design space, which I decomposed, guided by the
theory of the model's definitions, features, purposes, and representations in
the \kl{CN} type system. I showed that there are multiple reasonable design
points for typing rules to capture, and that doing so soundly can come down to
very subtle technical details. After summarising the point I chose for
\kl{CN-VIP}, I showed that factoring out the definition and usage of the heap
allows for significantly more modular updates and structure to the
overall proof of soundness. I then explained how I implemented these rules in
\kl{CN} and their effect on the annotation and performance overhead of \kl{CN}.
At the end, I discussed how I fed back insights from the theory to improve performance,
and outlined the development of new features such as support for lemmas and
better foundations for \kl{CN}.

Lastly, in~\arefpart{engineering}, I showed that a verification tool for
real-world C, needs some way of taming the complexity of large repositories and
non-standard or unsupported aspects of C. I showed that whilst \kl{CN}'s first
sub-goal of proving \kl{pKVM} seems to be feasible, there are several concrete
ways in which needs to become more usable. I showed that the second sub-goal of
on-going proof maintenance still needs substantial work. I then compared
\kl{CN} to other proof tools by discussing my experience on verifying a small
but informative example in all of them, which indicated where it stands with
respect to the academic state-of-the-art. I then discussed industry feedback on
\kl{CN}, which indicated where \kl{CN} stands in relation to user
expectations. Lastly, I synthesised the large gap between them by outlining
concrete expectations and how to achieve them for \kl{CN} and similar tools.

There are of course, longer-term lines of research to pursue based on \kl{CN}.
I see at least four broad directions for this. I already covered in detail the
direction of more trustworthiness in \nameref{sec:better-foundations}; I will
discuss the remaining aspects here.

\section{More expressiveness}

% Increasing expressiveness of \kl{CN} means adding support for more C features,
% and in tow, a richer separation logic.
\begin{itemize}
    \item \textbf{User-defined variadic functions.} I believe the types of
        \kl{CN} are rich enough to support this, but supporting the
        \kl{Cerberus} constructs which enable this will require more work.
    \item \textbf{Function pointers.} Supporting this in the general
        case would require a higher-order separation logic, which would
        be considerable departure from the current design and require
        thinking deeply about reasonable schemes and expectations for
        automation.
    \item \textbf{Standard library support.} Full support for the above
        two features should allow this straightforwardly, though
        there may be considerable engineering involved.
    \item \textbf{Locks and relaxed-memory concurrency.} A first-order solution
        would be to provide locks as a primitive, and that should suffice in
        many cases, but at the very least would require fractional permissions
        to make usable. Relaxed memory models will require an even richer
        separation logic; as before, the impact on automation is unknown.
    \item \textbf{Inline assembly.} As seen in the early allocator in
        \kl{pKVM}, inline assembly is used in real-world C, and supporting
        this soundly would be a rather large integration with frameworks
        such as Islaris~\sidecite{sammler2022islaris}.
    \item \textbf{Binary verification.} Mete Polat did some exploratory
        work~\sidecite{polat2023automated} on translating \kl{CN}
        specifications and proofs to binary using the
        Islaris framework, and turning this into full-fledged support
        for direct binary verification (as opposed to verifying an
        industrial-grade optimising compiler) would be a large undertaking.
\end{itemize}

\section{More performance}

\begin{itemize}
    \item \textbf{Perform a `mem2reg' pass on \kl{Core}.} Representing stack
        variables whose address is not taken as resources is unnecessary.
        Transforming memory actions on such locations into immutable \kl{Core}
        variables and expressions, akin to the \kl{LLVM} memory-to-register
        pass (mem2reg), would reduce the amount of resource inference,
        bounds-checking, disjointness reasonsing required of the \kl{SMT}
        solver.
    \item \textbf{Use ownership typing.} Whilst low-level systems C code relies
        on aliasing pointers, most code and most pointers are not aliasing. Hence,
        a more restrictive discipline handled quickly for those common
        situaitons, for example using Rust-style borrow-checking as a type
        system,\sidenote{\url{https://smallcultfollowing.com/babysteps/blog/2024/03/04/borrow-checking-without-lifetimes}}
        could allow users to benefit from fast checking by default, and opt-in to a slower,
        more general scheme as needed. For this to work, the ownership
        discipline would need to be expressible in terms of \kl{CN}'s resource
        logic.
    \item \textbf{Optimising solvers for typical queries.} It seems
        likely that the types of queries \kl{CN} will typically send a
        solver are quite amenable to automation, especially related
        to pointers and the \kl{CN-VIP} memory model. If these can be
        made substantially faster, then this would have a large
        benefit on \kl{CN} performance.
    \item \textbf{Carefully incrementalisng verification.} For large projects,
        \kl{CN} should cache proofs it already knows and only check the parts
        which have changed.
\end{itemize}

\section{More user-friendliness}

\begin{itemize}
    \item \textbf{Iterating on syntax with user studies.} Syntax matters
        a lot to most people outside of programming language research,
        and the syntax of \kl{CN} right now is elegant but verbose. It
        needs to be made simultaneously more concise and intuitive, whilst
        retaining a clear denotation. C programmers may prefer an imperative
        specification language, and this may be more feasible than one might
        expect~\sidecite{memarian2024ghost}, accommodating which would be a new
        challenge for specification research.
    \item \textbf{Experimenting with graphical representations of resources.}
        It is difficult to understand failed resource lookups in \kl{CN}.
        On top of this, although currently \kl{CN}'s ownership model is simple,
        features such as fractional permissions will add further complexity to
        this. It would be useful to experiment with graphical explanations of
        resource manipulations, either representations of a symbolic heap, or
        resource change timelines à la
        RustViz\sidenote{\url{https://fplab.github.io/rustviz-tutorial/}}~\sidecite{almeida2022rustviz}.
        Of course a key difference between C and Rust is not tying aliasing
        to syntax, and so (potential) aliases would need to be explained
        appropriately.
    \item \textbf{Pedagogy and incremental benefits for professionals.}
        The target audience of \kl{CN} is people who are more invested in
        correctness and learning than the average programmer, but will still
        have limited time in which to become productive and show benefits
        to stakeholders. Even if the preceding two ideas are executed perfectly,
        verification is different enough from programming and test-based tooling
        to require structured pedagogy. On top of this, it will always be
        competing with the other two for attention. A sensible pedagogy,
        combined with incremental benefits (such as runtime checking of
        incomplete assertions~\sidecite{banerjee2025fulminate}), will
        be required for the substantial uptake.
    \item \textbf{Suggesting specifications, proofs and repairs.} Aside from
        being difficult, verification is often tedious. There is rich history
        of using techniques from artificial intelligence to improve the
        automation of both SAT solvers and interactive proof assistants, and
        there is much excitement about the potential for using machine
        learning, particularly large language models, for improving on the
        state-of-the art~\sidecite{first2023ml4tp}.
\end{itemize}


%----------------------------------------------------------------------------------------

\appendix % From here onwards, chapters are numbered with letters, as is the appendix convention
\pagelayout{wide} % No margins
\addpart{Appendix}
\setchapterstyle{plain}

%TC:ignore
{\small%
\renewcommand{\cngrammartabular}[1]{%
\begin{supertabular}{@{}p{0.12\textwidth}@{}l@{\ \ }c@{\ \ }ll@{}l}#1\end{supertabular}}
\renewcommand{\cnrulehead}[3]{\multirow[c]{2}{=}[1.5ex]{\begin{mathbreakcomma}#1\end{mathbreakcomma}} & & ${\Colon}{=}$ & \multicolumn{3}{l}{#3}}
\renewcommand{\cnprodline}[6]{%
& & $#1$ & $#2$ & $#3 #4$ & $#5$%
\ifthenelse{ \equal{}{#6} }%
{#6}%
{\\ & & & \multicolumn{3}{@{}p{0.8\textwidth}}{#6}}}
\chapter{Definitions}\label{chap:defns}
\section{Types and Patterns}

\subsection{Resource Related}
\cndefnsresXXjudge%

\subsection{Return Type Equality}
\cndefnsretXXjudge%

\subsection{Patterns}
\cndefnspatXXjudge%

\section{Explicit System}

\subsection{Pure Expressions}
\cndefnsexplXXpure%

\subsection{Resource Terms}
\cndefnsexplXXres%

\subsection{Return Terms}
\cndefnsexplXXspine%

\subsection{Effectful expressions}
\cndefnsexplXXaction%
\cndefnsexplXXmemop%
\cndefnsexplXXstmt%

% Elaboration rules are out of date, but still instructive
\section{Elaboration System}
\cndefnsinfXXres%
\cndefnselabXXaction%
\cndefnselabXXmemop%
\cndefnselabXXspine%
\cndefnselabXXexpr%
\cndefnselab%

\section{Operational Semantics}
\cndefnssubsXXjudge%
\cndefnspureXXopsem%
\cndefnsopsem%

\section{Miscellaneous}
\cndefnsproofXXdefns%
\cndefnsuserXXdefns%
\cndefnsheapXXsat%

\section{MiniCN}
\cndefnsmini%

\section{Metvars and Grammar}
\cnmetavars\\[\baselineskip]
\cngrammar%
}

\begingroup%
\newcommand{\lemmaref}[1]{lemma~\ref{#1} (\nameref{#1})}
\renewcommand{\cnprodline}[6]{%
& & $#1$ & $#2$ & $#3 #4$ & $#5$%
\ifthenelse{ \equal{}{#6} }%
{#6}%
{\\ & & & \multicolumn{3}{@{}p{0.8\textwidth}}{#6}}}
\include{soundness_proof.gen}
\endgroup%
%TC:endignore

%----------------------------------------------------------------------------------------
%	BIBLIOGRAPHY
%----------------------------------------------------------------------------------------

\backmatter% Denotes the end of the main document content

% The bibliography needs to be compiled with biber using your LaTeX editor, or on the command line with 'biber main' from the template directory

\defbibnote{bibnote}{Here are the references in citation order.\par\bigskip} % Prepend this text to the bibliography
\printbibliography[heading=bibintoc, title=Bibliography, prenote=bibnote] % Add the bibliography heading to the ToC, set the title of the bibliography and output the bibliography note

%----------------------------------------------------------------------------------------
%	INDEX
%----------------------------------------------------------------------------------------

% The index needs to be compiled on the command line with 'makeindex main' from the template directory

% \printindex % Output the index

\newpage \mbox{}
\cleardoubleevenemptypage%
% \includepdf[pages=2]{../MathSTICTemplate/main.pdf}

\end{document}
