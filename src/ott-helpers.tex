\input{cn_included.gen}    % auto-gen'd rules and grammars
\usepackage{ott/ottlayout} % customising layout
\input{cn_override.gen}    % intercepting premise, rule and defn blocks to ottlayout ones
\input{cn_drulenames.gen}  % auto-gen'd macros for rule names in small caps

\ottstyledefaults{%
numberpremises=yes,%
premisenamelayout=topright,%
rulelayout=nobreaks}

%%%%%%%%%%%%%%%%%%%
% Grammar helpers %
%%%%%%%%%%%%%%%%%%%

\ExplSyntaxOn%
\cs_new_eq:NN\StrIfInTF\str_if_in:nnTF
\cs_new_protected:Npn \MaybeSwallowCnProdNewline {
  \peek_remove_spaces:n
  \peek_catcode:NTF \cnprodnewline { \use_none:n } { }
}
\ExplSyntaxOff%

%%%%%%%%%%%%%%%%%%%%%%%%
% Grammar presentation %
%%%%%%%%%%%%%%%%%%%%%%%%

\renewcommand{\cnprodline}[6]{%
\StrIfInTF{#3}{M}{\MaybeSwallowCnProdNewline}{%     % hide meta prods
  \StrIfInTF{#5}{P}{\MaybeSwallowCnProdNewline}{%   % hide proof-related prods
    & & $#1$ & $#2$ & $#3$ & $#5$ & #6}}}           % no bind specs (#4)

\renewcommand{\cnbindspecprodline}[6]{} % hide extra bind specs

\NewDocumentCommand{\cngrammarcompressed}{mm}{
    \begingroup%
    \renewcommand{\cngrammartabular}[1]{\begin{align*}##1\end{align*}}
    \renewcommand{\cnrulehead}[3]{##1  &##2}
    \renewcommand{\cnfirstprodline}[6]{\begingroup\begin{minipage}[t]{#1}\raggedright\ $##2$}
    \renewcommand{\cnprodline}[6]{\StrIfInTF{##3}{M}{}{\StrIfInTF{##5}{P}{}{\ $##1$~$##2$}}}{}
    \renewcommand{\cnprodnewline}{}
    \renewcommand{\cninterrule}{\end{minipage}\endgroup\\}
    \renewcommand{\cnafterlastrule}{\end{minipage}\endgroup\\}
    \cngrammartabular{#2}
    \endgroup%
}

%%%%%%%%%%%%%%%%%%%%%%%
% Rule premise filter %
%%%%%%%%%%%%%%%%%%%%%%%

\ExplSyntaxOn%
\NewDocumentCommand{\onlyUseNthCnPremise}{m +m}
 {
   \group_begin:

   \seq_set_split:Nnn \l_indices { , } { #1 }
   % \seq_show:N \l_indices

   % Backup original \cndrule
   \cs_set_eq:NN \orig_cndrule: \cndrule

   % Locally override \cndrule
   \RenewDocumentCommand{\cndrule}{O{}mmm}
     {
       % Split on \cnpremise
       \seq_set_split:Nnn \l_premises { \cnpremise } { ##2 }
       % First one is empty
       \seq_pop_left:NN \l_premises \l_tmpa_tl%
       % \seq_show:N \l_premises

       \seq_clear_new:N \l_filtered
       \seq_map_indexed_inline:Nn \l_premises
         {
           \seq_if_in:NnTF \l_indices { ####1 }
             {
               \seq_put_right:Nn \l_filtered { \cnpremise { ####2 } }
               % \iow_term:x { Adding~premise~####1:~\tl_to_str:n {####2} }
             }
             {
               % \iow_term:x { Skipping~premise~####1:~\tl_to_str:n {####2} }
             }
         }
       % \seq_show:N \l_filtered

       \orig_cndrule: [##1]{\seq_use:Nn \l_filtered {}}{##3}{##4}
     }

   #2%
   \group_end:
 }
\ExplSyntaxOff%

%%%%%%%%%%%%%%%%%%%%%
% Defn block filter %
%%%%%%%%%%%%%%%%%%%%%

\ExplSyntaxOn%
% argument #1 calls \cnusedrule{\cmdX{}} with different \cmdX{}, and we want it
% to only have an effect if \cmdX{} is in the list #2.
\NewDocumentCommand{\onlyUseRules}{mm}
{
  \group_begin:%
  % split #2 on comma (clist = comma list) and put each element in a sequence
  \seq_clear:N \l_tmpa_seq
  \clist_map_inline:nn {#2}
  {
    \seq_gput_right:Nn \l_tmpa_seq {##1}
  }%

  %\typeout{Sequence}%
  %\seq_show:N \l_tmpa_seq {}%

  % save the original command
  \NewCommandCopy{\origcnusedrule}{\cnusedrule}%

  % intercept \cnusedrule to check if its argument is in the sequence,
  % and only call the original if so (otherwise do nothing)
  \renewcommand{\cnusedrule}[1]{%
    \seq_if_in:NnT \l_tmpa_seq {##1}
      {%
        %\typeout{Found!}
        \origcnusedrule{##1}%
      }%
  }
  #1
  \group_end:%
}
\ExplSyntaxOff%


%%%%%%%%%%%%%%%%%%%%%%%%%%%
% Defn block presentation %
%%%%%%%%%%%%%%%%%%%%%%%%%%%

% Put judgement comment on a new line, half-BLS below the judgement form (in a box)
\renewenvironment{cndefnblock}[3][]{\framebox{\mbox{#2}} \\[0.5\baselineskip] #3 \leavevmode\\[0pt]}{\leavevmode\\[0pt]}